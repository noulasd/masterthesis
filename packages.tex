
%%%%%%%%%%%%%%%%%%%%%%%%%%%%
\usepackage{amsthm}
\usepackage{amsmath}
\usepackage{amssymb}
%%%%%%%%%%%%%%%%%%%%%%%%%%%%%
\usepackage[greek,english]{babel}
\usepackage{alphabeta}
\usepackage[utf8]{inputenc}
\usepackage{mathtools}
\usepackage{blindtext}
\usepackage[T1]{fontenc}
\usepackage{titlesec}
\usepackage{sectsty}
\usepackage{verbatim}
\usepackage{multirow}
\chapternumberfont{\tiny} 
\chaptertitlefont{\Huge}
%ελληνικοι χαρακτηρες σε μαθ pdf utf-8
%%%%%%%%%%%%%%%%%%%%%%%%%%%%%%%%%
\usepackage{tikz-cd}
%\usepackage{hyperref}
\usepackage{xcolor}
\usepackage{framed}%frames
\usepackage{float}
\usepackage{array}
\usepackage{pbox}
\usepackage{caption}
%%%%%%%%%%%%%%%%%%%%%%%%
\usepackage{tikz}
%%%%%%%%%%%%%%%%%%%%%%%%%%
%\usepackage{sagetex} needs sagetex.sty file
%%%%%%%%περιθώρια%%%%%%%%%%%%
\usepackage[a4paper,margin=3.5cm]{geometry}

\usepackage{graphicx,import}

\usepackage{titlesec}    
\titleformat{\chapter}[display]
{\normalfont%
    \huge% %change this size to your needs for the first line
    \bfseries}{\chaptertitlename\ \thechapter}{20pt}{%
    \Huge %change this size to your needs for the second line
    }

\usepackage[n, % o r lambda
advantage,
operators,
sets,
adversary,
landau,
probability,
notions,
logic,
ff,
mm,
primitives,
events,
complexity,
oracles,
asymptotics,
keys ]{cryptocode}

\newtheorem{theorem}{Θεώρημα}
\newtheorem{lemma}{Λήμμα}
\newtheorem{example}{Παράδειγμα}
\newtheorem{defn}{Ορισμός}
\newtheorem{prop}{Πρόταση}
\newtheorem{cor}{Πόρισμα}
\newtheorem*{remark}{Παρατήρηση}

\newcommand {\tl}{\textlatin}
%%%%%%%%%αριθμηση%%%%%%%%%%%%%%
\renewcommand{\theenumi}{\arabic{enumi}}
\renewcommand{\labelenumi}{{\rm(\theenumi)}}
\renewcommand{\labelenumii}{\roman{enumii}) }
%%%%%%%%%%%% New theorems %%%%%%%%%%%%%%%%%%%%%%%%

%%%%%%%%%%%%%%%%%%%%%%%%%%%%%%%%%%%%%%%%%%%%%%%%%%%
\newcommand{\Z}{\mathbb{Z}}
\newcommand{\Q}{\mathbb{Q}}
\newcommand{\Co}{\mathbb{C}}
\newcommand{\So}{\mathcal{S}}
\newcommand{\C}{\mathcal{C}}
\newcommand{\Gal}{\operatorname{Gal}}
\newcommand{\p}{\mathfrak{p}}


\newcommand{\Sheaf}{(\So, \pi, X)}
\usepackage{listings}
\usepackage{color}

\newcommand{\el}{\lambda}
\newcommand{\eL}{\Lambda}
\newcommand{\presheaf}{(S(U),\rho^U_V)_{V\subseteq U\in \tau_X}}
\newcommand{\ds}{(A_{\lambda},\phi^{\lambda_1}_{\lambda_2})}
\newcommand{\dsb}{(B_{\lambda},\psi^{\lambda_1}_{\lambda_2})}
\newcommand{\dlimit}{(A,\phi_{\lambda})}
\newcommand{\openx}{\mathcal{N}^0_x}

\newcommand{\curly}{\mathrel{\leadsto}}

\newcommand{\defeq}{\mathrel{\stackrel{\makebox[0pt]{\mbox{\normalfont\tiny \text{ορσ}}}}{=}}}

\newcommand{\incfig}[1]{%
    \fontsize{12pt}{12pt}\selectfont %χρησιμοποιώ height 600 width 300 pixels
    \def\svgwidth{2in}
    \import{./figures/}{#1.pdf_tex}
}
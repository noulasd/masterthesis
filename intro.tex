\hspace{\parindent}Αρχικά, η θεωρία \tl{Iwasawa} είχε ως στόχο την 
μελέτη των μεθεθών των ομάδων κλάσεων ιδεωδών για τα κυκλοτομικά σώματα και σώματα που σχετίζονται 
με αυτά. Τα πιο πρόσφατα αποτελέσματα της θεωρίας αυτής πλέον διατυπώνονται ως διάφορες μορφές μιας ((κύριας εικασίας)) της θεωρίας 
\tl{Iwasawa}. Η πρώτη απλή μορφή που σχετίζεται με αβελιανές επεκτάσεις του $\Q$ αποδείχθηκε το 1984 από τους \tl{Mazur} και \tl{Wiles} 
και στην συνέχεια το 1990 για τα πλήρως πραγματικά σώματα από τον \tl{Wiles}. Μια κύρια εικασία αυτής της μορφής συσχετίζει τα 
μεγέθη των ομάδων κλάσεων, ή ειδικότερα των ομάδων \tl{Selmer}, στις $p$-αδικές $L$-συναρτήσεις. Σε αυτήν την μεταπτυχιακή εργασία 
θα εστιάσουμε στην κλασική θεωρία και στα βασικά αποτελέσματα με απώτερο στόχο να αποδείξουμε το θεώρημα του \tl{Iwasawa} 
για τα μεγέθη των ομάδων κλάσεων για τα σώματα που βρίσκονται ενδιάμεσα σε $\Z_p$-επεκτάσεις.

Η δομή της εργασίας είναι ως εξής. Στο επόμενο κεφάλαιο, δίνουμε τα απαραίτητα προαπαιτούμενα που θα χρειαζόταν ένας μεταπτυχιακός φοιτητής για 
να ακολουθεί εύκολα τα επιχειρήματα που θα ακολουθήσουν για την βασική θεωρία. Στην συνέχεια, δίνουμε τους απαραίτητους ορισμούς και εργαλεία 
για τα διάφορα αποτελέσματα της κλασικής θεωρίας \tl{Iwasawa} που σχετίζονται με τις τάξεις των ομάδων κλάσεων, όπου γίνεται εμφανής η σύνδεση 
των τάξεων αυτών με ειδικές τιμές των $L$-συναρτήσεων, των $p$-αδικών $L$-συναρτήσεων, των αριθμών \tl{Bernoulli} και των \tl{irregular} πρώτων αριθμών. 

Στο τρίτο κεφάλαιο αναπτύσσουμε τις ιδιότητες που έχει η άλγεβρα \tl{Iwasawa} $\Lambda = \Z_p[[T]]$ και κάθε πεπερασμένη επέκτασή της, βασιζόμενοι στην 
δομή των τοπικών σωμάτων $K$ που είναι επεκτάσεις του $\Q_p$ και τα εργαλεία που αποκτούμε με το να θεωρήσουμε δακτύλιους τυπικών δυναμοσειρών με συντελεστές από 
τους δακτύλιους ακεραίων $\mathcal{O}_K$. Με αυτά τα εργαλεία, δείχνουμε αρχικά ότι το $\Lambda$ επιδέχεται έναν αλγόριθμο διαίρεσης, ο οποίος με την σειρά του 
μας δίνει ένα πολύ ισχυρό θεώρημα δομής των $\Lambda$-προτύπων, όμοιο με το κλασικό θεώρημα δομής πάνω από περιοχές κυρίων ιδεωδών.

Στο τέταρτο κεφάλαιο χρησιμοποιούμε μια διαφορετική προσέγγιση για την άλγεβρα \tl{Iwasawa}, βλέποντας την ως το προβολικό όριο 
ομαδοδακτυλίων $\Z_p[\Gamma_n]$. Ταυτόχρονα, κοιτάμε την $p$-\tl{Sylow} υποομάδα $X_n$ της ομάδας κλάσεων σε κάθε πεπερασμένο στρώμα 
καθώς ανεβαίνουμε μια $\Z_p$-επέκταση, όπου γίνεται η συσχέτιση των $X_n$ ως $\Z_p[\Gamma_n]$-προτύπων. Ως συνέπεια, με τις πληροφορίες που μας δίνει η θεωρία κλάσεων σωμάτων 
για τα $X_n$, το ισχυρό θεώρημα δομής και τις δύο διαφορετικές όψεις της άλγεβρας $\Lambda$, παίρνουμε το επιθυμητό αποτέλεσμα για τον ρυθμό αύξησης των $|X_n|$ που είναι γνωστό ως θεώρημα \tl{Iwasawa}.

Τέλος, στο κεφάλαιο 5 αναπτύσσουμε περαιτέρω τις ιδέες του προηγούμενου κεφαλαίου ώστε να παρέχουμε το κατάλληλο υπόβαθρο για να διατυπωθεί η κύρια εικασία για τα πλήρως πραγματικά σώματα. Επιπλέον, 
χρησιμοποιώντας την κύρια εικασία παίρνουμε ένα ακόμα αποτέλεσμα για τα μεγέθη των ομάδων κλάσεων.

Δίνουμε παρακάτω μια γρήγορη περιγραφή των όρων και την αλληλεπίδρασή τους, όπως αυτή γίνεται εμφανής στα επόμενα κεφάλαια.

$ $\newline
Έστω $K_n = \Q(\zeta_{p^n})$ για $n \geq 1$ και $K_\infty = \Q(\zeta_{p^\infty}) = \cup K_n$. Έχουμε τον ισομορφισμό
$$\Gal(K_\infty/\Q) \cong \Z^\times_p$$
$$ \sigma \longmapsto a_\sigma \in \Z^\times_p$$ που καθορίζεται πλήρως από την σχέση
$$\sigma(\zeta_{p^n}) = \zeta_{p^n}^{a_\sigma}.$$ Υπενθυμίζουμε ότι έχουμε επιπλέον τον ισομορφισμό
$$\Z^\times_p \cong (\Z/p\Z)^\times \times \Z_p .$$ Θέτουμε $\Q_\infty = K_\infty^{(\Z/p\Z)^\times}$, έτσι ώστε 
$$\Gal(\Q_\infty/\Q) \cong \Z_p.$$ Η επέκταση $\Q_\infty/\Q$ είναι αυτό που εννοούμε ως $\Z_p$-επέκταση. Έστω $\gamma \in \Gal(K_\infty/\Q)$ να είναι τέτοιο ώστε 
$\gamma \longmapsto 1 + p \in \Z^\times_p$ στον παραπάνω ισομοριφσμό. Η εικόνα του $\gamma$ μέσα στην $\Gal(\Q_\infty/\Q)$ είναι ένας τοπολογικός γεννήτορας 
και θα συνεχίζουμε να τον συμβολίζουμε με $\gamma$.

Έστω $\chi: (\Z/N\Z)^\times \longrightarrow \overline{\Q}^\times$ ένας πρωταρχικός χαρακτήρας \tl{Dirichlet}. Βλέπουμε τον 
$\chi$ ως χαρακτήρα της $\Gal(\overline{\Q}/\Q)$ μέσα από την προβολή στο πηλίκο:
$$\chi : \Gal(\overline{\Q}/\Q) \longrightarrow \Gal(\Q(\zeta_N)/\Q) \cong (\Z/N\Z)^\times \longrightarrow \overline{\Q}^\times .$$ Έστω 
$\Q^\chi = \overline{\Q}^{\ker \chi}$ να είναι το σώμα διάσπασης του $\chi$. Δηλαδή, έχουμε ότι
$$\Gal(\overline{\Q}/\Q^\chi) \cong \ker \chi ,$$ άρα το $\chi$ μπορούμε να το βλέπουμε και σαν χαρακτήρα της $\Gal(\Q^\chi/\Q)$. Υποθέτουμε ότι $\Q^\chi \cap \Q_\infty = \Q$ και θέτουμε $F_\infty = \Q^\chi \Q_\infty$. Με αυτά, θα έχουμε ότι
$$\Gal(F_\infty/\Q) \cong \Gamma \times \Delta,$$ όπου
$$\Delta = \Gal(F_\infty/\Q_\infty) \cong \Gal(\Q^\chi/\Q),$$ άρα βλέπουμε το $\chi$ και σαν χαρακτήρα του $\Delta$ και σαν χαρακτήρα του $\Gamma$, εφόσον
$$\Gamma = \Gal(F_\infty/\Q_\chi) \cong \Gal(\Q_\infty/\Q) \cong \Z_p.$$ 


\noindent Έχουμε ότι υπάρχουν σώματα $F_n \subset F_\infty$ που αντιστοιχούν στις υποομάδες $\Gamma^{p^n}$ του $\Gamma$, όπως και $\Q_n$ να αντιστοιχούν στις υποομάδες $p^n\Z_p$ του $\Z_p$, έτσι ώστε 
$$\Gal(F_n/\Q^\chi) \cong \Gamma/\Gamma^{p^n} \cong \Z/p^n\Z .$$ Έστω $L_n$ να είναι η μέγιστη αδιακλάδιστη αβελιανή $p$-επέκταση του $F_n$. Η ομάδα $X_n = \Gal(L_n/F_n)$ είναι ισόμορφη με την $p$-\tl{Sylow} υποομάδα $A_n$ της ομάδας κλάσεων ιδεωδών του $F_n$. Για $L = \cup L_n$, θέτουμε 
$X = \Gal(L/F_\infty)$ και έτσι έχουμε το ακόλουθο διάγραμμα σωμάτων.

\begin{figure}[H]
    \centering
    \begin{tikzcd}
        & L                                                   &                                          \\
        & F_\infty \arrow[u, "X"', no head]                   &                                          \\
\Q^\chi \arrow[ru, "\Gamma", no head] &                                                     & \Q_\infty \arrow[lu, "\Delta"', no head] \\
        & \Q \arrow[lu, no head] \arrow[ru, "\Z_p"', no head] &                                         
\end{tikzcd}
\end{figure}

\noindent Συνεπώς, το $X$ είναι ένα $\Delta \times \Gamma$-πρότυπο. 
Ειδικότερα, είναι ένα $\Z_p [[\Gamma]]$-πρότυπο και αποδεικνύουμε ότι $\Z_p[[\Gamma]] \cong \Lambda := 
\Z_p [[T]]$. Δηλαδή, έχουμε ότι το $X$ είναι ένα $\Lambda$-πρότυπο και μάλιστα πεπερασμένα παραγόμενο $\Lambda$-στρέψης, οπότε υπάρχει ένας ομομορφισμός $\Lambda$-προτύπων
$$X \longrightarrow \left(\bigoplus\limits_{i=1}^r \Lambda/(p^{\mu_i})\right) \oplus \left(\bigoplus\limits_{j=1}^s 
\Lambda/(f_j(T)^{m_j})\right).$$ Έχουμε ότι ο πυρήνας και συνπυρήνας του παραπάνω ομομορφισμού είναι πεπερασμένοι 
και $\mu_i, m_j \geq 0$ και τα $f_j(T)$ είναι ανάγωγα μονικά πολυώνυμα στο $\Z_p[T]$. Επιπλέον, οι όροι $\mu_i$ και το πολυώνυμο $f_X(T) = \prod f_j(T)^{m_j}$ καθορίζονται μοναδικά από το $X$.

Τα προηγούμενα που αναφέραμε είναι λόγω του θεωρήματος δομής των $\Lambda$-προτύπων και το ακόλουθο είναι το θεώρημα του \tl{Iwasawa} που μας λέει ακριβώς τον ρυθμό αύξησης του $p$-μέρους της ομάδας κλάσεων ιδεωδών μέσα σε μια $\Z_p$-επέκταση.

\begin{theorem}[\tl{Iwasawa}]
        Έστω $p^{e_n}$ να είναι η δύναμη του $p$ που διαιρεί την τάξη της ομάδας κλάσεων του $F_n$. Τότε υπάρχουν ακέραιοι $\lambda \geq 0, \mu \geq 0$ και $\nu$, όλα ανεξάρτητα από το $n$, μαζί με $n_0 \in \mathbb{N}$ τέτοιο ώστε για κάθε $n\geq n_0$ να έχουμε
        $$e_n = \lambda n + \mu p^n + \nu .$$
\end{theorem}


\noindent Θέτουμε $V = X \otimes_{\Z_p} \overline{\Q}_p$. Το $V$ είναι ένας πεπερασμένης διάστασης διανυσματικός χώρος καθώς
$$V \cong \overline{\Q}_p(T)/(f_X(T)).$$ Επιπλέον, θέτουμε
$$V^\chi = \{v \in V: \ \sigma v = \chi(\sigma)v \ \forall \ \sigma \in \Delta \}.$$ Αυτό είναι το 
$\chi$-ισοτυπικό κομμάτι του $V$. Έστω $f_\chi(T)$ να είναι το χαρακτηριστικό πολυώνυμο της δράσης του $\gamma -1$ στο $V^\chi$, όπως το $f_X(T)$ είναι το χαρακτηριστικό πολυώνυμο της δράσης του $\gamma -1$ στο $V$. 
Άρα έχουμε και ότι $f_\chi (T) \mid f_X(T)$. Αν αντί να θεωρήσουμε το τανυστικό γινόμενο με το $\overline{\Q}_p$ και παίρναμε στην θέση του το $\mathcal{O}_\chi := \Z_p[\chi]$ θα παίρναμε έναν επιπλέον όρο $\mu_\chi$, που θα αντιστοιχούσε στην δύναμη που εμφανίζεται ο \tl{uniformizer} $\pi$ του $\mathcal{O}_\chi$.

Η κύρια εικασία συσχετίζει το χαρακτηριστικό πολυώνυμο $f_\chi(T)$ και τον όρο $\mu_\chi$ με μια $p$-αδική $L$-συνάρτηση και έναν αναλυτικό όρο $\mu$. Έστω $\psi$ ένας πρωταρχικός χαρακτήρας \tl{Dirichlet}. Θέτουμε

$$H_\psi (T) = \begin{cases}
        \psi(1+p)(1+T) -1 & \psi=1 \ \text{ ή έχει \tl{conductor} μια δύναμη του } p, \\
        1 & \text{διαφορετικά.}
\end{cases}$$ Υπάρχει $G_\psi(T) \in \mathcal{O}_\psi [[T]]$ τέτοια ώστε
$$\mathcal{L}_p(1-s,\psi) = G_\psi((1+p)^s -1)/H_\psi((1+p)^s -1) \quad (s \in \Z_p)$$ έτσι ώστε
$$\mathcal{L}_p(1-n,\psi) = (1-\psi\omega^{-n}(p)p^{n-1})L(1-n,\psi\omega^{-n}) \quad (n\geq 1).$$ Αυτή είναι η 
$p$-αδική $L$-συναρτήση για την οποία θα δώσουμε περισσότερες λεπτομέρειες αργότερα. Από το θεώρημα προπαρασκευής του \tl{Weierstrass} μπορούμε να γράψουμε

$$G_\psi(T) = \pi^{\mu^{\operatorname{an}}_\psi} g_\psi(T) u_\psi(T),$$ όπου $\mu^{\operatorname{an}}_\psi \geq 0$, το $g_\psi(T)$ είναι μονικό πολυώνυμο στο $\mathcal{O}_\psi[T]$ και το $u_\psi(T)$ είναι αντιστρέψιμο στο $\mathcal{O}_\psi[[T]]$.

\begin{theorem}[Κύρια Εικασία της Θεωρίας \tl{Iwasawa}] Έστω $\chi$ ένας περιττός χαρακτήρας τάξης σχετικά πρώτης με το $p$ για τον οποίο ισχύει ότι $\Q^\chi \cap \Q_\infty = \Q$. Τότε
        $$f_\chi(T) = g_{\chi^{-1}\omega}((1+p)(1+T)^{-1}-1)$$ και 
        $$\mu_\chi = \mu^{\operatorname{an}_{\chi^{-1}\omega}}.$$
\end{theorem}

\noindent Χρησιμοποιώντας την κύρια εικασία παίρνουμε ένα ακόμα αποτέλεσμα για τα μεγέθη των τάξεων των ομάδων κλάσεων.

\begin{theorem} Έστω $p$ ένας περιττός πρώτος και $F$ μια αβελιανή φανταστική επέκταση του
        $\Q$ βαθμού σχετικά πρώτου με το $p$. Έστω $\chi : \Gal(F/\Q) \longrightarrow \overline{\Q}^\times_p$ να είναι ένας περιττός χαρακτήρας. Υποθέτουμε ότι $\chi \neq \omega$, τότε 
        $$|A^\chi_F| = |\mathcal{O}_\chi / (\mathcal{L}_p(0,\chi^{-1}\omega))|,$$ όπου 
        $A^\chi_F$ είναι το $\chi$-ισοτυπικό κομμάτι του $p$-μέρους της ομάδας κλάσεων ιδεωδών του $F$.
\end{theorem}
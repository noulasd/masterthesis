\section{Θεώρημα Δομής $\Lambda_{\mathcal{O}}$-προτύπων}

Έχουμε δείξει στην προηγούμενη ενότητα ότι το $\Lo$ είναι περιοχή μοναδικής παραγοντοποίησης με ανάγωγα στοιχεία τo $\pi$ και τα ανάγωγα \tl{distinguished} πολυώνυμα. Υπενθυμίζουμε και ότι τα αντιστρέψιμα στοιχεία του $\Lo$ είναι αυτά που έχουν σταθερό όρο στο $\mathcal{O}^\times$.

Σε αυτή την ενότητα θα διατυπώσουμε ένα πολύ ισχυρό θεώρημα δομής για τα πεπερασμένα παραγόμενα $\Lo$-πρότυπα, το οποίο θα χρησιμοποιηθεί για την μελέτη ομάδων \tl{Galois} με απώτερο σκοπό το θεώρημα \tl{Iwasawa}.

\begin{lemma}\label{lemma4.7}
    Έστω $f,g \in \Lo$ σχετικά πρώτα. Τότε το ιδεώδες $(f,g)$ έχει πεπερασμένο δείκτη στο $\Lo$.
\end{lemma}

\begin{proof}
    Έστω $h\in(f,g)$ με ελάχιστο βαθμό. Από το θεώρημα προπαρασκευής του \tl{Weierstrass} μπορούμε να γράψουμε δίχως βλάβη γενικότητας $h(T)=\pi^n H(T)$ για κάποιον ακέραιο $n$ και όπου $H=1$ ή $H$ να είναι \tl{distinguished} πολυώνυμο. Έχουμε αυτή τη γραφή εφόσον θεωρούμε το $h$ ελάχιστου βαθμού, διαφορετικά από το $h(T) = \pi^n H(T)U(T)$ θα παίρναμε ως νέο $h$ το $hU^{-1}$.

    Αν $H\neq 1$ τότε μπορούμε να γράψουμε $f=qH+r$ με $\deg r < \deg H$. Αυτό μας δίνει
\begin{align*}
    \pi^n f &= q\pi^n H + \pi^n r \\
    &= qh + \pi^n r.
\end{align*}

\noindent Ωστόσο, αυτό μας δείχνει ότι $\pi^n r \in (f,g)$ και έχει μικρότερο βαθμό από το $h$, το οποίο είναι άτοπο. Υποθέτουμε τώρα ότι $H=1$ και άρα $h=\pi^n$. Μπορούμε να υποθέσουμε ότι $\pi \nmid f$, καθώς αν το διαιρεί μπορούμε να κάνουμε την ίδια υπόθεση για το $g$ εφόσον είναι σχετικά πρώτα. Επιπλέον, δίχως βλάβη γενικότητας μπορούμε πάλι με το θεώρημα προπαρασκευής του \tl{Weierstrass} να υποθέσουμε ότι το $f$ είναι \tl{distinguished}, διαφορετικά θα παίρναμε ως $f$ το $P(T) = f(T)/U(T)$ το οποίο παράγει το ίδιο ιδεώδες. Έχουμε ότι 

$$(\pi^n,f) \subseteq (f,g)$$ και άρα 
$$\Lo/(\pi^n,f) \supseteq \Lo/(f,g).$$

\noindent Από τον αλγόριθμο διαίρεσης παίρνουμε ότι όλα τα στοιχεία του $\Lo$ είναι ισοδύναμα \tl{modulo} $(\pi^n,f)$ σε πολυώνυμα βαθμού μικρότερου του $f$ με συντελεστές \tl{modulo} $\pi^n$. Ξεκάθαρα, έχουμε πεπερασμένες το πλήθος επιλογές για αυτά τα πολυώνυμα πεπερασμένου βαθμού, άρα το $(\pi^n,f)$ έχει πεπερασμένο δείκτη και συνεπώς ισχύει το ίδιο και για το $(f,g)$.
\end{proof}

\begin{lemma} Αν $R$ δακτύλιος της \tl{Noether} τότε είναι της \tl{Noether} και ο δακτύλιος $R[[X]]$.
\end{lemma}


\begin{proof}
    Μιμούμαστε την απόδειξη του θεωρήματος βάσης του \tl{Hilbert} ότι αν $R$ δακτύλιος της \tl{Noether}, τότε θα είναι και ο δακτύλιος 
    $R[X]$, ωστόσο με μια διαφοροποίηση. Καθώς ένα στοιχείο θα γράφεται ως $f = a_r X^r + a_{r+1}X^{r+1} + \cdots$ και δεν υπάρχει 
    μεγιστοβάθμιος όρος θα θεωρούμε για την συγκεκριμένη απόδειξη βαθμό του $f$ την μικρότερη δύναμη $r$ που εμφανίζεται με κύριο συντελεστή $a_r \neq 0$. Αν $f=0$, λέμε ότι ο βαθμός είναι άπειρος και ο κύριος συντελεστής να είναι το $0$.

    Έστω $I$ ένα ιδεώδες του $R[[X]]$, όπου θα δείξουμε ότι είναι πεπερασμένα παραγόμενο. Θα κατασκευάσουμε επαγωγικά μια ακολουθία 
    στοιχείων $f_i \in R[[X]]$ ως εξής. Έστω $f_1$ με ελάχιστο βαθμό ανάμεσα στα στοιχεία του $I$. Υποθέτουμε ότι έχουμε διαλέξει $f_1,\ldots, f_i$ με τα $f_j$ να έχουν βαθμό $d_j$ και κύριο συντελεστή $a_j$. Τότε διαλέγουμε το $f_{i+1}$ ώστε να ικανοποιεί τις ακόλουθες τρεις συνθήκες:

    \begin{enumerate}
        \item $f_{i+1} \in I$.
        \item $a_{i+1} \notin (a_1,\ldots,a_i)$.
        \item Από όλα τα στοιχεία που ικανοποιούν τις πρώτες δύο συνθήκες, το $f_{i+1}$ να έχει ελάχιστο βαθμό.
    \end{enumerate}

    \noindent Η δεύτερη συνθήκη αναγκάζει την διαδικασία να τερματίσει σε πεπερασμένα βήματα, διαφορετικά θα υπήρχε άπειρη γνήσια αύξουσα αλυσίδα
    $$(a_1) \subset (a_1,a_2) \subset (a_1,a_2,a_3) \subset \cdots$$ ιδεωδών στον δακτύλιο $R$ που είναι της \tl{Noether}. Υποθέτουμε ότι η παραπάνω ακολουθία τερματίζει στο βήμα $k$, έτσι θα δείξουμε ότι το $I$ παράγεται από τα $f_1,\ldots, f_k$.
    
    $ $\newline
    Έστω $g = aX^d + \cdots $ να είναι ένα στοιχείο του $I$ βαθμού $d$ με κύριο συντελεστή το $a$. Τότε $a \in (a_1,\ldots,a_k)$, αφού αν δεν άνηκε θα συνέχιζε την προηγούμενη διαδικασία.
    
    $ $\newline
    \textbf{Περίπτωση 1:} $d\geq d_k$. Έχουμε $d_i \leq d_{i+1}$ για κάθε $i$, εφόσον έχουν οριστεί τα $d_i$ με αυτή τη σειρά λόγω της συνθήκης $(3)$ και άρα $d\geq d_i$ για κάθε $i=1,\ldots, k$. Έστω ότι $a = \sum\limits_{i=1}^k c_{i0}a_i$ με $c_{i0} \in R$. Ορίζουμε
    $$g_0 = \sum\limits_{i=1}^k c_{i0} X^{d-d_i}f_i$$
    έτσι ώστε το $g_0$ να έχει βαθμό $d$ και κύριο συντελεστή $a$. Συνεπώς, το $g-g_0$ θα έχει βαθμό μεγαλύτερο του $d$. Έχοντας με την 
    ίδια διαδικασία ορίσει $g_0,\ldots,g_r \in (f_1,\ldots, f_k)$ έτσι ώστε το $g-\sum\limits_{i=0}^r g_i$ να έχει βαθμό μεγαλύτερο του 
    $d+r$, μπορούμε να υποθέσουμε ότι
    $$g-\sum\limits_{i=0}^r g_i = bX^{d+r+1} + \cdots $$
    όπου φυσικά το επιχείρημα παραμένει το ίδιο για βαθμό μεγαλύτερο του $d+r+1$. Έχουμε για τον ίδιο λόγο με πριν ότι $b \in (a_1,\ldots, a_k)$ και άρα 
    $$b = \sum\limits_{i=1}^k c_{i,r+1}a_i$$
    με τα $c_{i,r+1}$ να ανήκουν στο $R$. Ορίζουμε
    $$g_{r+1} = \sum\limits_{i=1}^k c_{i,r+1} X^{d+r+1-d_i} f_i$$
    έτσι ώστε το $g-\sum\limits_{i=0}^r g_i$ να έχει βαθμό μεγαλύτερο του $d+r+1$. Συνεπώς,
    $$g = \sum\limits_{r=0}^\infty g_r = \sum\limits_{r=0}^\infty \sum\limits_{i=1}^k c_{ir}X^{d+r-d_i} f_i$$

    \noindent και από αυτό έπεται ότι $g\in (f_1,\ldots,f_k)$, εφόσον μπορούμε να αλλάξουμε την σειρά άθροισης καθώς στην μια περίπτωση είναι πεπερασμένο το πλήθος.
    
    $ $\newline
    \textbf{Περίπτωση 2:} $d<d_k$. Όπως πριν, $a \in (a_1,\ldots,a_k)$ και άρα υπάρχει ένα ελάχιστο $m$ μεταξύ του $1$ και του $k$ έτσι ώστε $a \in (a_1,\ldots, a_m)$. Έπεται ότι $d\geq d_m$. Όπως στην πρώτη περίπτωση έχουμε $a = \sum\limits_{i=1}^m c_i a_i$ με $c_i \in R$. Ορίζουμε
    $$h = \sum\limits_{i=1}^m c_i X^{d-d_i} f_i \ \in (f_1,\ldots, f_k) \subseteq I.$$
    Ο κύριος συντελεστής του $h$ είναι το $a$, άρα ο βαθμός του $g-h$ είναι μεγαλύτερος του $d$. Αντικαθιστούμε το $g$ με το $g-h$ και επαναλαμβάνουμε την διαδικασία. Έτσι, μετά από το πολύ $d_k - d$ βήματα θα έχουμε κατασκευάσει ένα στοιχείο $g-\sum h_i$ στο $I$ βαθμού τουλάχιστον $d_k$, με όλα τα $h_i$ να είναι στο $(f_1,\ldots, f_k)$. Έτσι, από την ανάλυση που κάναμε στην πρώτη περίπτωση, θα έχουμε ότι $g\in (f_1,\ldots, f_k)$.
\end{proof}

\begin{cor} Ο δακτύλιος $\Lo$ είναι δακτύλιος της \tl{Noether}.
\end{cor}



\begin{proof}
Ο δακτύλιος $\mathcal{O}$ είναι περιοχή του \tl{Dedekind} και άρα δακτύλιος της \tl{Noether}, συνεπώς εφαρμόζεται το προηγούμενο λήμμα.
\end{proof}

\noindent Παρακάτω θα χρειαστούμε ότι για ένα \tl{distinguished} πολυώνυμο $P$, το $P^n$ θα τείνει στο $0$ στην τοπολογία του $\Lo$. Αυτό θα γίνει πιο ξεκάθαρο με το θεώρημα τομής του \tl{Krull}, οπότε εφόσον είδαμε ότι το $\Lo$ είναι δακτύλιος της \tl{Noether} θα κατηγοριοποιήσουμε στην συνέχεια τα πρώτα και μέγιστα ιδεώδη.
%Αν προσέξει κανείς, το θεώρημα του \tl{Iwasawa} είναι στην φύση του ασυμπτωτικό εφόσον το αποτέλεσμα ισχύει για κάθε $n$ πάνω από ένα $n_0$. Χρειαζόμαστε δηλαδή μια έννοια σύγκλισης. Την έννοια αυτή θα την πάρουμε όπως θα δούμε από το θεώρημα τομής του \tl{Krull}, άρα εφόσον δείξαμε ότι ο δακτύλιος $\Lo$ είναι της \tl{Noether} μένει να δείξουμε ότι είναι και τοπικός.

\begin{prop}
Οι πρώτοι του $\Lo$ είναι οι $0,(\pi,T),(\pi)$ και τα ιδεώδη $(P(T))$ όπου $P(T)$ είναι ανάγωγο \tl{distinguished} πολυώνυμο. Το ιδεώδες $(\pi,T)$ είναι το μοναδικό μέγιστο.
\end{prop}

\begin{proof}
    Έχουμε τους ισομορφισμούς:
    \begin{align*}
    \Lo/(\pi,T) &\cong \mathcal{O}/(\pi) \\
    \Lo/(\pi) &\cong (\mathcal{O}/(\pi))[[T]] \\
    \Lo/(P(T)) &\cong \mathcal{O}[T]/(P(T)) \\
    \Lo/0 &\cong \Lo ,
    \end{align*}

    \noindent όπου το πρώτο είναι πεπερασμένο σώμα και τα υπόλοιπα είναι ακέραιες περιοχές. Άρα αρκεί να δείξουμε ότι κάθε πρώτος στο $\Lo$ είναι σε μία από αυτές τις μορφές. Έστω $\q \subset \Lo$ ένα μη μηδενικό πρώτο ιδεώδες. Έστω $h \in \q$ με ελάχιστο βαθμό. Από το θεώρημα προπαρασκευής του \tl{Weierstrass} γράφουμε $h = \pi^n H$ με $H=1$ ή $H \in \q$. Αν $H=1$, τότε δεν μπορούμε να έχουμε $H \in \q$ καθώς τότε $\q = \Lo$. Υποθέτουμε ότι $H\neq 1$ και $H\in \q$. Έτσι, το $H$ πρέπει να είναι ανάγωγο από την υπόθεση ελαχίστου βαθμού που κάναμε για το $h$. Συνεπώς, $(H)\subseteq \q$. Αν $(H) = \q$ δεν έχουμε κάτι να δείξουμε. Άρα υποθέτουμε ότι $(H)\neq \q$, δηλαδή υπάρχει $g \in \q$ με $H\nmid g$. Καθώς το $H$ είναι ανάγωγο, έχουμε αναγκαστικά ότι τα $H,g$ είναι σχετικά πρώτα. Συνεπώς από το λήμμα \ref{lemma4.7} έχουμε ότι το $(H,g)$, άρα και το $\q$ έχουν πεπερασμένο δείκτη στο $\Lo$. Έχουμε ότι το $\Lo/\q$ είναι ένα πεπερασμένο $\mathcal{O}$-πρότυπο και άρα $\pi^N \in \q$ για αρκετά μεγάλο $N$. Καθώς το $\q$ είναι πρώτο έπεται ότι $\pi \in \q$. Για τον ίδιο λόγο έχουμε ότι θα υπάρχουν $i<j$ με $T^i \equiv T^j (\operatorname{mod}\q)$. Επιπλέον, το $1-T^{j-i}$ είναι αντιστρέψιμο εφόσον έχει αντιστρέψιμο σταθερό όρο. Άρα από την παρακάτω σχέση
    $$ T^i (1- T^{j-i}) \equiv 0 (\operatorname{mod}\q) $$
    παίρνουμε ότι $T^i \in \q$, άρα και $T \in \q$. Δείξαμε ότι $(\pi,T) \subseteq \q$, ωστόσο το $(\pi,T)$ είναι μέγιστο και άρα $\q = (\pi,T)$. Χρησιμοποιούμε το ίδιο επιχείρημα για την περίπτωση που $H=1$ και $\pi^n \in \q$. Έτσι, όλοι οι πρώτοι περιέχονται στο $(\pi,T)$, άρα ο δακτύλιος $\Lo$ είναι τοπικός.
\end{proof}

\noindent Από τα προηγούμενα, για τον τοπικό δακτύλιο $\Lo$ της \tl{Noether} με μέγιστο ιδεώδες $(\pi,T)$ ισχύει από το θεώρημα τομής του \tl{Krull} ότι
$$\bigcap \limits_{n=1}^\infty (\pi,T)^n = 0.$$


\noindent Έτσι, για ένα \tl{distinguished} πολυώνυμο
$$P(T) = T^n + a_{n-1}T^{n-1} + \cdots + a_1 T + a_0, \quad a_i \in (\pi)$$
έχουμε ότι $P^k \in (\pi,T)^k$ και η ακολουθία των ιδεωδών είναι προφανώς φθίνουσα, συνεπώς:

$$ \text{ καθώς } k\longrightarrow \infty, \quad P^k \in \bigcap\limits_{n=1}^\infty (\pi,T)^n = 0$$ δηλαδή, με αυτή την έννοια $P^k \longrightarrow 0$ στο $\Lo$.
Αυτό είναι ιδιαίτερα χρήσιμο όπως θα φανεί στο επόμενο λήμμα.

\begin{lemma}\label{lemma4.11}
    Έστω $f,g \in \Lo$ να είναι σχετικά πρώτα. Τότε
    \begin{enumerate}
        \item Η φυσική απεικόνιση
        $$\Lo/(fg) \longrightarrow \Lo/(f) \oplus \Lo /(g)$$ είναι μονομορφισμός με πεπερασμένο συνπυρήνα.
        \item Υπάρχει εμφύτευση
        $$\Lo/(f) \oplus \Lo/(g) \longrightarrow \Lo/(fg)$$ με πεπερασμένο συνπυρήνα.
    \end{enumerate}
\end{lemma}

\begin{proof}

    \begin{enumerate}
        %όταν το σκέφτομαι είναι καλό, να το γράψω καλύτερα 1)
        \item Έστω $h$ να ανήκει στον πυρήνα της απεικόνισης, τότε $f\mid h$ και $g\mid h$. Αυτά είναι σχετικά πρώτα και καθώς το $\Lo$ είναι περιοχή μοναδικής παραγοντοποίησης θα εμφανίζεται το $fg$ στην ανάλυση του $h$ σε πρώτους. Για τον συνπυρήνα, θεωρούμε ένα στοιχείο $(a(\operatorname{mod}f), b(\operatorname{mod}g))$. Υποθέτουμε ότι $a-b \in (f,g)$. Τότε υπάρχουν $c,d$ έτσι ώστε 
        $$a-b = fc + gd.$$
        Θέτουμε $\gamma = a-fc = b+gd$ και παρατηρούμε ότι 
        $$\gamma \equiv a(\operatorname{mod}f), \quad \gamma \equiv b (\operatorname{mod}g) .$$

        Άρα το $\gamma$ είναι στην εικόνα της απεικόνισης. Έστω $r_1,\ldots, r_n$ να είναι αντιπρόσωποι του πεπερασμένου όπως δείξαμε $\Lo/(f,g)$. Τότε έχουμε ότι το σύνολο
        $$\{(0 (\operatorname{mod}f), r_i (\operatorname{mod}g)): \ 1\leq i \leq n\}$$
        είναι ένα σύνολο αντιπροσώπων του συνπυρήνα της απεικόνισης.

        \item Θέτουμε $M = \Lo/(fg)$ και $N = \Lo/(f)\oplus \Lo/(g)$. Ξέρουμε ότι $M\subseteq N$ με πεπερασμένο δείκτη όπως δείξαμε στο $(1)$. Έστω $P$ ένα \tl{distinguished} πολυώνυμο στο $\Lo$ σχετικά πρώτο με το $fg$. Καθώς έχουμε $P^k \rightarrow 0$ στο $\Lo$, ισχύει ότι $P^k N \subseteq M$ για κάποιο $k$. Υποθέτουμε ότι $(P^k x, P^k y) = 0$ στο $N$ για κάποιο $(x,y)\in N$. Θα ισχύει ότι $f\mid P^kx$ και $g\mid P^ky$. Έχοντας διαλέξει $P$ έτσι ώστε $\gcd(P,fg)= 1$, έχουμε αναγκαστικά ότι $f\mid x$ και $g\mid y$. Συνεπώς, $(x,y) = 0 $ στο $N$. Άρα, η παρακάτω απεικόνιση είναι εμφύτευση:
        \begin{figure}[H]
            \centering
            \begin{tikzcd}
                N \arrow[r, "P^k"] & M
                \end{tikzcd}
        \end{figure}

        Η εικόνα αυτής της απεικόνισης περιέχει και το $(P^k \cdot 1, P^k \cdot 0) = (P^k,0) = (P^k,fg)$, το οποίο σαν ιδεώδες είναι πεπερασμένου δείκτη όπως έχουμε δείξει, εφόσον τα $P,fg$ είναι σχετικά πρώτα. Συνεπώς
        $$\operatorname{Im}(P^k) \supseteq (P^k,fg) $$
        $$\Lo/\operatorname{Im}(P^k) \subseteq \Lo/(P^k,fg)$$ και άρα ο συνπυρήνας έχει πεπερασμένο δείκτη.
    \end{enumerate}
\end{proof}

\begin{lemma} \label{lemma4.10}
    Έστω $f \in \Lo - \Lo^\times$. Τότε το $\Lo/(f)$ έχει άπειρη τάξη.
\end{lemma}

\begin{proof}
    Υποθέτουμε ότι $f\neq 0$. Από το θεώρημα προπαρασκευής του \tl{Weierstrass} γράφουμε $f= \pi^n H$ με $H=1$ ή \tl{distinguished}. Παρατηρούμε ότι $(f)\subseteq (\pi)$ ή $(f) \subseteq (H)$ και άρα αρκεί να θεωρήσουμε τις περιπτώσεις $f = \pi$ ή $f$ να είναι \tl{distinguished} πολυώνυμο. Στην δεύτερη περίπτωση παίρνουμε το αποτέλεσμα με εφαρμογή του αλγόριθμου διαίρεσης, εφόσον θα έχουμε για αντιπροσώπους πολυώνυμα βαθμού το πολύ $\deg f -1$ αλλά άπειρο το πλήθος επιλογές για τους συντελεστές. Για την άλλη περίπτωση, αν $f=\pi$ τότε $\Lo/(\pi) \cong (\mathcal{O}/\pi) [[T]]$ το οποίο είναι άπειρης τάξης.
\end{proof}

\begin{defn}
    Δύο $\Lo$-πρότυπα $M$ και $N$ θα λέγονται ψευδο-ισόμορφα και θα τα γράφουμε $M\sim N$, αν υπάρχει ακριβής ακολουθία:
    $$0 \longrightarrow A \longrightarrow M \longrightarrow N \longrightarrow B \longrightarrow 0$$ όπου τα $A,B$ είναι πεπερασμένα $\Lo$-πρότυπα.
\end{defn}

\noindent Ο παραπάνω ορισμός είναι ισοδύναμος με το να υπάρχει ομομορφισμός $\Lo$-προτύπων $M\longrightarrow N$ με πεπερασμένο πυρήνα 
και συνπυρήνα. Εδώ φαίνεται καλύτερα ότι η σχέση $M\sim N$ δεν συνεπάγεται στο ότι $N\sim M$. Πράγματι, έχουμε την ακριβή ακολουθία:

$$0 \longrightarrow (\pi,T) \longrightarrow \Lo \longrightarrow \mathcal{O}/(\pi)\longrightarrow 0$$ και άρα $(\pi,T) \sim \Lo$. Ωστόσο, δεν ισχύει το αντίστροφο. Αν είχαμε $\Lo \sim (\pi,T)$ δηλαδή έναν ομομορφισμό $\tau$ έτσι ώστε η παρακάτω ακολουθία να είναι ακριβής:

\begin{figure}[H]
    \centering
    \begin{tikzcd}
        0 \arrow[r] & \ker \tau \arrow[r] & \Lo \arrow[r, "\tau"] & {(\pi,T)} \arrow[r] & \operatorname{coker} \tau \arrow[r] & 0 \\
                    &                     & 1 \arrow[r, maps to]  & f(T)                &                                     &  
        \end{tikzcd}
\end{figure}
\noindent Τότε $\tau(\Lo) = (f) \subseteq (\pi,T)$, όπου το $f$ προφανώς δεν μπορεί να είναι αντιστρέψιμο μέσα στο $(\pi,T)$. Άρα το $\Lo/(f)$ είναι άπειρης τάξης και συνεπώς το υποπρότυπό του $(\pi,T)/(f) = \operatorname{coker} \tau$ είναι άπειρης τάξης. Άρα δεν μπορούμε να έχουμε $\Lo \sim (\pi,T)$. Ωστόσο, η σχέση $\sim$ είναι συμμετρική όταν μιλάμε για $\Lo$-πρότυπα $\Lo$-στρέψης. 


\begin{theorem}[Δομής Πεπερασμένα Παραγόμενων $\Lo$-Προτύπων]
    Έστω $M$ ένα πεπερασμένα παραγόμενο $\Lo$-πρότυπο. Τότε
    $$M \sim \Lo^r \oplus \left(\bigoplus\limits_{i=1}^s \Lo/(\pi^{n_i})\right) \oplus \left( \bigoplus\limits_{j=1}^t \Lo/(f_j(T)^{m_j})\right)$$ όπου τα $r,s,t,n_i$ και $m_j$ ανήκουν στο $\Z$ και τα $f_j(T)$ είναι \tl{distinguished} και ανάγωγα πολυώνυμα. Αυτή η διάσπαση καθορίζεται πλήρως από το $M$.
\end{theorem}

\begin{proof}
Το αποτέλεσμα αυτό είναι ίδιο με την διάσπαση προτύπων πάνω από περιοχές κυρίων ιδεωδών, με την διαφοροποίηση ότι εδώ έχουμε ψευδο-ισομορφισμούς. Η απόδειξη θα είναι μια γενίκευση των τεχνικών που χρησιμοποιήθηκαν σε εκείνο το θεώρημα.

$ $\newline
Έστω $M$ με γεννήτορες $u_1,\ldots,u_n$ με διάφορες σχέσεις
$$\lambda_1 u_1,\ldots, \lambda_n u_n = 0, \quad \lambda_i \in \Lo$$
\noindent Καθώς οι σχέσεις $R$ είναι υποπρότυπο του $\Lo^n$ και το $\Lo$ είναι της \tl{Noether}, το $R$ είναι πεπερασμένα παραγόμενο. Άρα μπορούμε να γράφουμε το $M$ σαν πίνακα με γραμμές της μορφής $(\lambda_1,\ldots, \lambda_n)$, όπου $\sum \lambda_i u_i = 0$ να είναι μια σχέση. Υπερφορτώνοντας τον συμβολισμό, θα λέμε αυτόν τον πίνακα $R$.

\noindent Ξεκινάμε με τις βασικές πράξεις γραμμών και στηλών, οι οποίες αντιστοιχούν στην αλλαγή των γεννητόρων των $R$ και $M$.

\noindent \textbf{Πράξη \tl{A}.} Μπορούμε να εναλλάσουμε τις γραμμές μεταξύ τους ή να εναλλάσουμε τις στήλες μεταξύ τους.

\noindent \textbf{Πράξη \tl{B}.} Μπορούμε να προσθέσουμε ένα πολλαπλάσιο μιας γραμμής (ή στήλης) σε μια άλλη γραμμή (ή στήλη). Ειδική περίπτωση αν $\lambda^\prime = q\lambda +r$ τότε
$$\begin{pmatrix}
    \vdots & & \vdots & \\
    \lambda & \cdots & \lambda^\prime & \cdots \\
    \vdots & & \vdots & 
\end{pmatrix} \longrightarrow \begin{pmatrix}
    \vdots & & \vdots & \\
    \lambda & \cdots & r & \cdots \\
    \vdots & & \vdots &
\end{pmatrix}$$

\noindent \textbf{Πράξη \tl{C}.} Μπορούμε να πολλαπλασιάσουμε μια γραμμή ή στήλη με στοιχείο του $\Lo^\times$.

\noindent Οι παραπάνω πράξεις είναι αυτές που χρησιμοποιούνται στις περιοχές κυρίων ιδεωδών. Ωστόσο, έχουμε άλλες τρεις επιπλέον πράξεις που βασίζονται στους ψευδο-ισομορφισμούς.


\noindent \textbf{Πράξη 1.}  Αν το $R$ περιέχει γραμμή $(\lambda_1,\pi \lambda_2,\ldots, \pi \lambda_n)$ με $\pi \nmid \lambda_1$, τότε μπορούμε να αλλάξουμε τον $R$ στον $R^\prime$, του οποίου η πρώτη γραμμή είναι $(\lambda_1,\lambda_2,\ldots,\lambda_n)$ και οι υπόλοιπες γραμμές είναι οι γραμμές του $R$, όπου η πρώτη στήλη είναι πολλαπλασιασμένη με $\pi$. Εικονικά:

$$\begin{pmatrix}
    \lambda_1 & \pi \lambda_2 & \cdots \\
    a_1 & a_2 & \cdots \\
    b_1 & b_2 & \cdots
\end{pmatrix} \longrightarrow \begin{pmatrix}
    \lambda_1 & \lambda_2 & \cdots \\
    \pi a_1 & a_2 & \cdots \\
    \pi b_1 & b_2 & \cdots \\
\end{pmatrix}$$
\noindent Ως ειδική περίπτωση, αν $\lambda_2 = \cdots = \lambda_n = 0$ τότε μπορούμε να πολλαπλασιάσουμε τα $a_1,b_1,\ldots$ με οποιαδήποτε δύναμη του $\pi$.
\begin{proof}
    Στο $R$ έχουμε την σχέση
    $$\lambda_1 u_1 + \pi(\lambda_2 u_2 + \cdots + \lambda_n u_n) = 0.$$
    Έστω $M^\prime = M\oplus \nu \Lo$, με έναν καινούργιο γεννήτορα $\nu$ όπου ισχύουν οι σχέσεις
    $$(-u_1,\pi \nu) = 0, \quad (\lambda_2 u_2 + \cdots + \lambda_n u_n, \lambda_1 \nu) = 0.$$
    Υπάρχει φυσική απεικόνιση $M\rightarrow M^\prime$. Υποθέτουμε ότι $m\mapsto 0$. Τότε το $m$ θα ανήκει στο πρότυπο των σχέσεων, οπότε
    $$(m,0) = a(-u_1,\pi\nu) + b(\lambda_2 u_2 + \cdots + \lambda_n u_n, \lambda_1 \nu)$$ με τα $a,b$ να είναι στο $\Lo$. Συνεπώς
    $$a\pi = -b \lambda_1.$$

    \noindent Καθώς $\pi \nmid \lambda_1$ από την υπόθεση, έχουμε ότι $\pi \mid b$. Επιπλέον, $\lambda_1 \mid a$. Άρα στην $M$-συνιστώσα έχουμε
    \begin{align*}
    m &= -\frac{a}{\lambda_1} (\lambda_1 u_1) - \frac{a}{\lambda_1} \pi(\lambda_2 u_2 + \cdots + \lambda_n u_n) \\
    &= -\frac{a}{\lambda_1}(0) = 0.
    \end{align*}

    \noindent Καθώς οι εικόνες των $\pi \nu$ και $\lambda_1 \nu$ στο $M^\prime$ είναι στην εικόνα του $M$, το ιδεώδες $(\pi,\lambda_1)$ 
    μηδενίζει το $M^\prime/M$. Εφόσον το $\Lo/(\pi,\lambda_1)$ είναι πεπερασμένο και το $M^\prime$ πεπερασμένα παραγόμενο, το $M^\prime/M$ θα είναι πεπερασμένο. Συνεπώς
    $$M \sim M^\prime .$$

    \noindent Το νέο πρότυπο $M^\prime$ έχει γεννήτορες $\nu,u_2,\ldots, u_n$. Κάθε σχέση $a_1 u_1 + \cdots a_n u_n = 0$ γίνεται $\pi a_1 \nu + \cdots + a_n u_n = 0$. Άρα η πρώτη στήλη έχει πολλαπλασιαστεί με το $\pi$. Επιπλέον, έχουμε την σχέση $\lambda_1 \nu + \cdots + \lambda_n u_n$. Άρα ο νέος πίνακας $R^\prime$ έχει την μορφή που αναφέραμε.
\end{proof}

\noindent \textbf{Πράξη 2.} Αν όλα τα στοιχεία στην πρώτη στήλη του $R$ διαιρούνται από το $\pi^k$ και αν υπάρχει 
γραμμή $(\pi^k\lambda_1,\ldots, \pi^k \lambda_n)$ με $\pi \nmid \lambda_1$, τότε μπορούμε να αλλάξουμε τον πίνακα στον 
$R^\prime$ που είναι ο ίδιος με τον $R$, αλλά στην θέση της γραμμής $(\pi^k\lambda_1,\ldots, \pi^k \lambda_n)$ μπαίνει η 
γραμμή $(\lambda_1,\ldots, \lambda_n)$. Εικονικά:
$$\begin{pmatrix}
    \pi^k \lambda_1 & \pi^k \lambda_2 & \cdots \\ 
    \pi^k a_1 & a_2 & \cdots 
\end{pmatrix} \longrightarrow \begin{pmatrix}
    \lambda_1 & \lambda_2 & \cdots \\
    \pi^k a_1 & a_2 & \cdots 
\end{pmatrix}.$$

\begin{proof} Έστω $M^\prime = M\oplus \Lo \nu$ όπου ισχύουν οι σχέσεις
    $$(\pi^k u_1, -\pi^k \nu) = 0, \quad (\lambda_2 u_2 + \cdots \lambda_n u_n, \lambda_1 \nu) = 0.$$
    Όπως και πριν, το ότι $\pi \nmid \lambda_1$ μας επιτρέπει να πούμε ότι το $M$ εμφυτεύεται στο $M^\prime$. Επιπλέον, το ιδεώδες $(\pi^k,\lambda_1)$ μηδενίζει το $M^\prime /M$, οπότε το πηλίκο αυτό είναι πεπερασμένο. Συνεπώς $M\sim M^\prime$.
    
    Χρησιμοποιώντας το γεγονός ότι $\pi^k (u_1 - \nu) = 0$ και το ότι το $\pi^k$ διαιρεί τον πρώτο συντελεστή από όλες τις σχέσεις που περιέχουν το $u_1$, παίρνουμε ότι
    $$M^\prime = M^{\prime \prime} \oplus (u_1 - \nu)\Lo$$
    όπου το $M^{\prime \prime}$ παράγεται από τα $\nu,u_1,\ldots, u_n$ και έχει σχέσεις που παράγονται από τα $(\lambda_1,\ldots, \lambda_n)$ και $R$. Συνεπώς το $M^{\prime \prime}$ έχει το $R^\prime$ ως τις σχέσεις του. Παρατηρούμε ότι
    $$(u_1 - \nu)\Lo \cong \Lo /(\pi^k)$$ το οποίο είναι ήδη στην μορφή που θέλουμε. Άρα αρκεί να δουλέψουμε με τα $M^{\prime \prime}$ και $R^\prime$.
\end{proof}

\noindent \textbf{Πράξη 3.} Αν το $R$ περιέχει γραμμή $(\pi^k \lambda_1,\ldots, \pi^k \lambda_n)$ και για κάποιο $\lambda \in \Lo$ με $\pi\nmid \lambda$ η γραμμή $(\lambda \lambda_1,\ldots, \lambda \lambda_n)$ να είναι σχέση (όχι απαραίτητα γραμμή του $R$), τότε μπορούμε να αλλάξουμε το $R$ με το $R^\prime$ που είναι το ίδιο εκτός από την γραμμή $(p^k \lambda_1,\ldots, \pi^k \lambda_n)$ που θα αντικατασταθεί με την $(\lambda_1,\ldots, \lambda_n)$.

\begin{proof} Θεωρούμε τον επιμορφισμό:
    $$M \longrightarrow M^{\prime} = M/(\lambda_1 u_1 + \cdots + \lambda_n u_n) \Lo , $$

    \noindent του οποίου ο πυρήνας μηδενίζεται από το ιδεώδες $(\lambda, \pi^k)$. Καθώς το $M$, άρα ως \tl{Noether} και ο πυρήνας, είναι πεπερασμένα παραγόμενα μαζί με το πεπερασμένο $\Lo/(\lambda, \pi)$, έχουμε ότι ο πυρήνας είναι πεπερασμένος. Άρα $M\sim M^\prime$ και το $M^\prime$ έχει το $R^\prime$ ως πίνακα σχέσεων.
\end{proof}

\noindent Αυτές οι έξι \tl{A,B,C},1,2,3 είναι οι επιτρεπτές πράξεις μας, οι οποίες διατηρούν το μέγεθος του πίνακα. Είμαστε σε θέση να ξεκινήσουμε. Έστω $f\neq 0 \in \Lo$, τότε
$$f(T) = \pi^\mu P(T)U(T) ,$$ με $P$ \tl{distinguished} πολυώνυμο και $U \in \Lo^\times$. Έστω 
$$\deg_w f = \begin{cases}
    \infty, \quad \mu > 0 \\
    \deg P(T), \quad \mu = 0
\end{cases}$$
το οποίο αναφέρεται ως ο βαθμός \tl{Weierstrass} του $f$. Δεδομένου ενός πίνακα $R$, ορίζουμε
$$\deg^{(k)}(R) = \min \deg_w (a^\prime_{ij}) \quad \text{ για } i,j \geq k ,$$ όπου τα $(a^\prime_{ij})$ διατρέχουν όλους τους πίνακες σχέσεων που παίρνουμε από το $R$ μέσω των επιτρεπτών πράξεων, οι οποίες αφήνουν αναλλοίωτες τις πρώτες $(k-1)$ γραμμές. Αν ο πίνακας $R$ έχει τη μορφή

$$\begin{pmatrix}
    \lambda_{11} & & 0 & & 0 & \cdots & 0 \\ 
    & \ddots& & & & &  \\
    0 & & \lambda_{r-1, r-1} & & 0 & \cdots & 0 \\
    * & \cdots & * & & * & \cdots & * \\ 
    * & \cdots & * & & * & \cdots & * 
\end{pmatrix} = \begin{pmatrix}
    D_{r-1} & 0 \\ 
    A & B
\end{pmatrix}$$ με $\lambda_{kk}$ να είναι \tl{distinguished} και 
$$\deg \lambda_{kk} = \deg_w \lambda_{kk} = \deg^{(k)}(R) \quad \text{ για } 1\leq k \leq r-1$$
τότε λέμε ότι o $R$ είναι σε $(r-1)$-κανονική μορφή.

$ $\newline
\textbf{Ισχυρισμός.} Αν ο υποπίνακας $B$ δεν είναι ο μηδενικός, τότε το $R$ μπορεί μέσω των επιτρεπτών πράξεων να μετατραπεί στον $R^\prime$ ο οποίος είναι σε $r$-κανονική μορφή και έχει τα ίδια πρώτα $(r-1)$ διαγώνια στοιχεία.
\begin{proof}
    Η ειδική περίπτωση στην πράξη 1 μας επιτρέπει να υποθέτουμε, όταν χρειάζεται ότι μια μεγάλη δύναμη του $\pi$ διαιρεί κάθε $\lambda_{ij}$ με $i\geq r$ και $j\leq r-1$. Δηλαδή, $\pi^N \mid A$, με $N$ αρκετά μεγάλο ώστε $\pi^N\nmid B$. Χρησιμοποιώντας την πράξη 2, υποθέτουμε ότι $\pi \nmid B$. Μπορούμε να υποθέσουμε ότι το $B$ περιέχει στοιχείο $\lambda_{ij}$ τέτοιο ώστε
    $$\deg_w \lambda_{ij} = \deg^{(r)}(R)<\infty .$$

    \noindent Αν $\lambda_{ij} = P(T)U(T)$, τότε πολλαπλασιάζουμε την $j$-οστή στήλη με το $U^{-1}$. Συνεπώς, μπορούμε να υποθέσουμε ότι το $\lambda_{ij}$ είναι \tl{distinguished}, εφόσον οι πρώτες $r-1$ γραμμές έχουν το 0 στην $j$-οστή στήλη και άρα δεν θα αλλάξουν. Λόγω της πράξης \tl{A} μπορούμε να υποθέσουμε ότι $\lambda_{ij}  = \lambda_{rr}$ (όπου πάλι τα μηδενικά μας βοηθάνε). Από τον αλγόριθμο διαίρεσης (ειδική περίπτωση της πράξης \tl{B}), μπορούμε να υποθέσουμε ότι το $\lambda_{rj}$ έιναι πολυώνυμο με
    $$\deg \lambda_{rj} < \deg \lambda_{rr}, \quad j\neq r$$ και 
    $$\deg \lambda_{rj} < \deg \lambda_{jj}, \quad j < r$$

    \noindent Καθώς το $\lambda_{rr}$ έχει ελάχιστο βαθμό \tl{Weierstrass} στο $B$, έχουμε αναγκαστικά ότι $\pi \mid \lambda_{rj}$ για τα $j > r$. Από την πράξη 1, υποθέτουμε ότι $\pi^N \mid \lambda_{rj}, j<r$, για κάποιο μεγάλο $N$. Υποθέτουμε ότι $\lambda_{rj} \neq 0$ για κάποιο $j > r$. Η πράξη 1 μας επιτρέπει εδώ να διώξουμε την δύναμη του $\pi$ από κάποιο μη μηδενικό $\lambda_{rj}$ με $j>r$ (με τα μηδενικά από πάνω να μείνουν αναλλοίωτα). Τότε 
    $$\deg_w \lambda_{rj} = \deg \lambda_{rj} < \deg \lambda_{rr} = \deg_w \lambda_{rr}$$ το οποίο είναι άτοπο. Συνεπώς, $\lambda_{rj} = 0$ για $j>r$. Αν κάποιο $\lambda_{rj}$ δεν είναι $0$ για $j<r$, χρησιμοποιούμε την πράξη 1 για να πάρουμε ότι $\pi \nmid \lambda_{rj}$ για κάποιο $j$. Ωστόσο, τότε θα έχουμε
    $$\deg_w \lambda_{rj} \leq \deg \lambda_{rj} < \deg \lambda_{jj} = \deg_w \lambda_{jj}$$
    και καθώς 
    $$\deg_w \lambda_{jj} = \deg^{(j)}(R)$$
    αυτό έρχεται σε αντίθεση με τον ορισμό του $\deg^{(j)}(R)$. Συνεπώς, $\lambda_{rj} = 0$ για κάθε $j\neq r$. Αυτό αποδεικνύει τον ισχυρισμό.
\end{proof}

\noindent Αν ξεκινήσουμε με έναν πίνακα $R$ και $r=1$, μπορούμε μέσω των πράξεων να αλλάξουμε τον $R$ μέχρι να πάρουμε έναν πίνακα
$$\begin{pmatrix}
    \lambda_{11} & & & 0 \\ 
    & \ddots & & & \\
    & & \lambda_{rr} &  \\
    A & & & 0 \\
\end{pmatrix},$$
με κάθε $\lambda_{jj}$ να είναι \tl{distinguished} και $\deg\lambda_{jj} = \deg^{(j)}(R)$ για κάθε $j\leq r$. Από τον αλγόριθμο διαίρεσης, μπορούμε να υποθέσουμε ότι τo $\lambda_{ij}$ είναι πολυώνυμο και 
$$\deg \lambda_{ij} < \deg \lambda_{jj}, \quad \text{ για } i \neq j.$$

\noindent Υποθέτουμε ότι $\lambda_{ij}\neq 0$ για κάποια $i\neq j$. Καθώς το $\deg_w \lambda_{jj}$ είναι ελάχιστο, το $\pi$ διαιρεί το $\lambda_{ij}$. Άρα έχουμε μια μη μηδενική σχέση $(\lambda_{i1},\ldots, \lambda_{ir},0,\ldots,0)$ την οποία διαιρεί το $\pi$. Θέτουμε $\lambda = \lambda_{11}\cdots \lambda_{rr}$. Τότε $\pi \nmid \lambda$, εφόσον τα $\lambda_{jj}$ είναι \tl{distinguished} και το παρακάτω
$$\left(\lambda \frac{1}{\pi} \lambda_{i1} \ ,\ldots, \lambda\frac{1}{\pi} \lambda_{ir} \ , 0 ,\ldots , 0 \right)$$ είναι μια σχέση, 
καθώς $\lambda_{jj} u _j = 0$. Από την πράξη 3 μπορούμε να υποθέσουμε ότι το $\pi$ δεν διαιρεί το $\lambda_{ij}$ για κάποιο $j$, οπότε
$$\deg_w \lambda_{ij} \leq \deg \lambda_{ij} < \deg \lambda_{jj} = \deg^{(j)}(R)$$
Το οποίο είναι άτοπο. Συνεπώς $\lambda_{ij}=0$ για κάθε $i$ και $j$ με $i\neq j$. Αυτό σημαίνει ότι $A=0$. Δηλαδή, σε όρους $\Lo$-προτύπων έχουμε
$$\Lo/(\lambda_{11}) \oplus \cdots \oplus \Lo /(\lambda_{rr}) \oplus \Lo^{n-r}.$$

\noindent Ξαναβάζοντας στην παραπάνω γραφή και τους όρους $\Lo/(p^k)$ που αφαιρέσαμε λόγω της πράξης 2, παίρνουμε το επιθυμητό αποτέλεσμα. Μια λεπτομέρεια είναι ότι τα $\lambda_{ii}$ δεν είναι απαραίτητα ανάγωγα, ωστόσο εδώ έρχεται το λήμμα \ref{lemma4.11} και τακτοποιεί αυτό το πρόβλημα. Εδώ τελειώνει η απόδειξη του θεωρήματος δομής.
\end{proof}
\section{$P$-\tl{adic} $L$-συναρτήσεις}

Σε αυτήν την ενότητα θα διατυπώσουμε την θεωρία γύρω από τις $p$-αδικές εκδόσεις των \tl{Dirichlet} $L$-συναρτήσεων. Στον ορισμό των 
$L$-συναρτήσεων που υπάρχει το $1/n^s$, ξέρουμε ότι αυτό θα παίρνει αυθαίρετα μεγάλες τιμές στην $p$-αδική νόρμα, καθώς όταν το $n$ θα 
τείνει στο άπειρο  θα το διαιρούν μεγάλες δυνάμεις του $p$. Άρα οι $p$-αδικές $L$-συναρτήσεις δεν μπορούν εδώ να οριστούν με τον ίδιο 
τρόπο, καθώς οι σειρές τους θα αποκλείνουν στο $\Q_p$. Ωστόσο, οι τιμές της $L(s,\chi)$ στους αρνητικούς ακέραιους είναι αλγεβρικές 
και άρα μπορούμε να θεωρήσουμε ότι βρίσκονται σε κάποια επέκταση του $\Q_p$. Συνεπώς, ψάχνουμε κάποια $p$-αδική συνάρτηση που να 
συμφωνεί με την $L(s,\chi)$ στους αρνητικούς ακεραίους. 

Με τις συναρτήσεις που θα κατασκευάσουμε θα δείξουμε ισοτιμίες μεταξύ των γενικευμένων αριθμών \tl{Bernoulli}, από όπου θα καταλήξουμε 
στο κριτήριο του \tl{Kummer} για το \tl{irregularity} των πρώτων. Αρχικά, θα κοιτάξουμε τις $p$-αδικές $L$-συναρτήσεις στο $s=1$ και 
θα βρούμε έναν τύπο όμοιο με την κλασική περίπτωση. Αυτό με την σειρά του θα μας δώσει μια $p$-αδική έκδοση του τύπου \tl{Dirichlet} 
για αριθμούς κλάσεων, από το οποίο θα πάρουμε το αποτέλεσμα του \tl{Kummer}: $p|h^+_p \implies p |h^-_p$.

Χρειαζόμαστε βασικά αποτελέσματα της ανάλυσης στους $p$-αδικούς αριθμούς, ξεκινώντας από το $\Q_p$. Καθώς θα χρειαστούμε όπως αναφέραμε 
αλγεβρικές επεκτάσεις του $\Q_p$ που παράγονται από τιμές των χαρακτήρων \tl{Dirichlet}. Επομένως, επεκτείνουμε στην αλγεβρική θήκη 
$\overline{\Q}_p$ του $\Q_p$. Η κατασκευή της γίνεται παίρνοντας την ένωση όλων των πεπερασμένων επεκτάσεων $K/\Q_p$, όπου σε κάθε 
τέτοια επέκταση η $p$-αδική νόρμα επεκτείνεται μοναδικά σε μια μη-αδχιμηδιανή απόλυτη τιμή στο $K$ ως εξής.

$$|\cdot |: K \longrightarrow \mathbb{R}_{\geq 0}$$
$$|x| =  \sqrt[n]{|N_{K/\Q_p}(x)|_p}$$
όπου $n=[K:\Q_p]$, όπως αποδεικνύεται στο \cite{Gouv}. Επιπλέον, δείχνεται ότι η τιμή του $|x|$ για $x \in K/\Q_p$ δεν επηρεάζεται 
από τα μεγαλύτερα σώματα που περιέχουν το $K$. Άρα κοιτώντας σε ένα $x \in \overline{\Q}_p$, έχουμε την πεπερασμένη επέκταση 
$\Q_p(x)$ και την μοναδική επέκταση της $p$-αδικής απόλυτης τιμής σε αυτό το σώμα. Έτσι, επεκτείνεται μοναδικά η $p$-αδική απόλυτη 
τιμή σε
$$|\cdot |: \overline{\Q}_p \longrightarrow \mathbb{R}_{\geq 0}$$
%F GOUVEA σελίδα 187

\begin{prop}
	Το $\overline{\Q}_p$ δεν είναι πλήρες.
\end{prop}
\begin{proof} Έστω 
	$$a = \sum\limits_{n=1}^\infty \zeta_{n^\prime} p^n$$ όπου $n^\prime = n$ αν $\gcd(n,p)=1$ και $n^\prime = 1$ διαφορετικά. Η ακολουθία των μερικών αθροισμάτων της παραπάνω σειράς τείνει στο $0$ ως προς την $p$-αδική νόρμα, άρα έχουμε ότι η σειρά συγκλίνει και ως επακόλουθο αν το $\overline{\Q}_p$ ήταν πλήρες θα ίσχυε ότι $a \in \overline{\Q}_p$. Οπότε, το $a$ θα άνηκε σε μια πεπερασμένη επέκταση $K$ του $\Q_p$. Υποθέτουμε ότι $\zeta_{n^\prime} \in K$ για όλα τα $n < m$ για κάποιο $m$. Επιπλέον, υποθέτουμε ότι $p\nmid m$. Τότε

	$$b = p^{-m} \left( a - \sum\limits_{n=0}^{m-1} \zeta_{n^\prime} p^n\right) \in K$$ και $b\equiv \zeta_m (\operatorname{mod}p)$. 
	Συνεπώς η εξίσωση $X^m -1 \equiv 0 (\operatorname{mod}p)$ έχει ρίζα στο $K$. Από την γενικότερη έκδοση του λήμματος του \tl{Hensel} που βρίσκεται στο \cite{AM}, 
	το $K$ περιέχει ρίζα του πολυωνύμου $X^m-1$ η οποία είναι ισοϋπόλοιπη με το $b (\operatorname{mod}p)$, άρα και με το $\zeta_m (\operatorname{mod}p)$. Καθώς οι $m$-οστές ρίζες της μονάδας είναι διακεκριμένες \tl{modulo} $p$, εφόσον ισχύει ότι 
	$$m = \prod\limits_{\substack{\zeta^m = 1 \\ \zeta \neq 1}}(1-\zeta)$$ παίρνουμε ότι $\zeta_m \in K$. Από επαγωγή, έχουμε ότι $\zeta_m \in K$ για κάθε $m$ με $p\nmid m$. Όπως παραπάνω, οι ρίζες της μονάδας τάξης σχετικά πρώτης ως προς το $p$ είναι διακεκριμένες \tl{modulo} $p$, άρα ο δακτύλιος των ακεραίων του $K$ περιέχει άπειρες τέτοιες. Φυσικά, αυτό είναι άτοπο καθώς η επέκταση $K/\Q_p$ είναι πεπερασμένη. Άρα το $\overline{\Q}_p$ δεν είναι πλήρες.
\end{proof}


\noindent Για τα εργαλεία της ανάλυσης, είναι πιο βολικό να εργαζόμαστε σε πλήρες σώμα. 
Θέτουμε $\mathbb{C}_p$ να είναι η πλήρωση του $\overline{\Q}_p$ ως προς την $p$-αδική απόλυτη τιμή. 
Αυτή με την σειρά της επεκτείνεται φυσιολογικά στο $\mathbb{C}_p$ ως

$$|\cdot | : \mathbb{C}_p \longrightarrow \mathbb{R}_{\geq 0}$$
$$|z| = \lim_{n\rightarrow \infty} |x_n|$$ για $(x_n)$ οποιαδήποτε ακολουθία στο $\overline{\Q}_p$ που τείνει στο $x$, 
εφόσον από κατασκευή το $\overline{\Q}_p$ είναι πυκνό στο $\mathbb{C}_p$. Ξεκινήσαμε από ένα πλήρες σώμα το $\Q_p$, 
το οποίο δεν είναι αλγεβρικά κλειστό. Για να δείξουμε ότι δεν συνεχίζεται άλλο αυτή η διαδικασία, δηλαδή 
ότι το $\mathbb{C}_p$ είναι αλγεβρικά κλειστό, χρειαζόμαστε το ακόλουθο λήμμα:

\begin{lemma}[\tl{Krasner}]
	Έστω $K$ ένα πλήρες σώμα ως προς μια μη-αρχημηδιανή εκτίμηση. Έστω $a,b \in \overline{K}$, η αλγεβρική θήκη του $K$, με $a$ διαχωρίσιμο πάνω από το $K(b)$. Επιπλέον, υποθέτουμε ότι για κάθε $a_i \neq a$ συζυγή με το $a$ έχουμε:
	$$|b-a| < |a_i - a|$$
	Τότε $K(a)\subseteq K(b)$ (εδώ $|x|$ είναι η μοναδική επέκταση της απόλυτης τιμής του $K$).
\end{lemma}

\noindent Με άλλα λόγια, αν το $b$ είναι αρκετά κοντά στο $a$ τότε το $a$ θα ανήκει στο $K(b)$.
\begin{proof}
	Θεωρούμε την επέκταση $K(a,b)/K(b)$ και έστω $L/K(b)$ να είναι η \tl{Galois} θήκη. Έστω $\sigma \in \Gal(L/K(b))$. Τότε $\sigma(b-a) = b -\sigma(a)$. Καθώς $|\sigma(x)| = |x|$ από την μοναδικότητα της επέκτασης της απόλυτης τιμής, έχουμε
	$$|b-\sigma(a) | = |b-a| < |a_i - a|$$ για όλα τα $a_i \neq a$. Συνεπώς 

	$$|a-\sigma(a)| \leq \max \{ |a-b|, |b-\sigma(a)|\} < |a_i-a|$$ Από το οποίο έπεται ότι $\sigma(a) =a$, άρα $a \in K(b)$ όπως θέλαμε.

\end{proof}

\begin{prop}
	Το $\mathbb{C}_p$ είναι αλγεβρικά κλειστό.
\end{prop}

\begin{proof}
	Έστω $a$ να είναι αλγεβρικό πάνω από το $\mathbb{C}_p$ και έστω $f(X)$ να είναι το ελάχιστό του πολυώνυμο στο $\mathbb{C}_p[X]$. Καθώς το $\overline{\Q}_p$ είναι πυκνό στο $\mathbb{C}_p$, μπορούμε να διαλέξουμε ένα πολυώνυμο $g(X) \in \overline{\Q}_p[X]$ του οποίου οι συντελεστές να είναι κοντά στους συντελεστές του $f(X)$. Τότε το $g(a) - f(a)$ θα είναι πολύ μικρό. Γράφοντας $g(X) = \prod (X-b_j)$, βλέπουμε ότι το $|a-b|$ είναι μικρό για κάποια ρίζα $b$ του $g(X)$. Ειδικότερα, μπορούμε να διαλέξουμε το $g(X)$ και μετά το $b$ έτσι ωστε
	$$|b-a| < |a_i -a|$$ για όλα τα συζυγή $a_i \neq a$ του $a$. Συνεπώς, από το λήμμα του \tl{Krasner} έχουμε ότι $a \in \mathbb{C}_p(b) = \mathbb{C}_p$, διότι $b \in \overline{\Q}_p \subset \mathbb{C}_p$.
\end{proof}


\noindent Ουσιαστικά, το $\mathbb{C}_p$ είναι η $p$-αδική έκδοση των μιγαδικών $\mathbb{C}$. Αυτά τα δύο σώματα είναι ισόμορφα με την 
αλγεβρική έννοια, αλλά όχι τοπολογικά. Και τα δύο έχουν μη-αριθμήσιμο βαθμό υπερβατικότητας πάνω από το $\Q$ και για να τα πάρουμε 
ξεκινάμε από το $\Q$ επισυνάπτοντας μια υπερβατική βάση και μετά παίρνουμε την αλγεβρική θήκη. Για αυτό, για τεχνικούς λόγους 
όπου είναι βολικό μπορούμε να εμφυτεύουμε το $\mathbb{C}$ στο $\mathbb{C}_p$ και αντίστροφα.

$ $\newline
Αν σταθεροποιήσουμε μια εμφύτευση $\overline{\Q} \xhookrightarrow{} \mathbb{C}_p$, μπορούμε να βλέπουμε τις τιμές που παίρνουν 
οι \tl{Dirichlet} χαρακτήρες μέσα στο $\mathbb{C}_p$. Είδικότερα, βλέπουμε τον χαρακτήρα \tl{Teichmüller} ως

$$\omega: \mathbb{F}_p^\times \longrightarrow \mathbb{C}_p$$ με $\omega(a)$ να είναι μια $(p-1)$-οστή ρίζα της μονάδας 
με $\omega(a) \equiv a (\operatorname{mod}p)$.


\begin{theorem}[Θεώρημα 5.11 στο \cite{Wash}] \label{wash5.11}
	Έστω $\chi$ ένας χαρακτήρας \tl{Dirichlet}. Υπάρχει μια $p$-αδική μερόμορφη απεικόνιση (αναλυτική αν $\chi \neq 1$) $\mathcal{L}_p(s,\chi)$ που ορίζεται στο $\{s \in \mathbb{C}_p: |s| < p^{1-1/(p-1)}\}$ έτσι ώστε:
	$$\mathcal{L}_p(1-n,\chi) = -(1-\chi \omega^{-n}(p)p^{n-1}) \frac{B_{n,\chi\omega^{-n}}}{n}, \quad n \geq 1.$$ Αν $\chi = 1$ τότε η $\mathcal{L}_p(s,1)$ είναι αναλυτική εκτός από έναν πόλο στο $s=1$ με ολοκληρωτκό υπόλοιπο $1-1/p$.
\end{theorem}

\noindent Το $\chi\omega^{-n}$ είναι ο πρωταρχικός χαρακτήρας που επάγει τον $a\mapsto \chi(a)\omega^{-n}(a)$. Γενικά, αυτοί οι δύο δεν ταυτίζονται καθώς για $\chi = \omega^n$ έχουμε
$$\omega^n \omega^{-n}(p) = p \neq 0 = \omega^n(p) \omega^{-n}(p)$$

\noindent Μπορεί να βλέπει κανείς την $p$-αδική $L$-συνάρτηση $\mathcal{L}_p(s,\chi)$ ως μια παρεμβολή, για τα διαφορετικά $p$, 
στην συνήθη $L(s,\chi)$. Συγκεκριμένα, για $n\geq 1$ έχουμε
$$\mathcal{L}_p (1-n,\chi) = (1-\chi\omega^{-n}(p)p^{n-1})L(1-n,\chi\omega^{-n})$$
Άρα, αν $n\equiv 0(\operatorname{mod}p-1)$ τότε παίρνουμε
$$\mathcal{L}_p(1-n,\chi) = (1-\chi(p)p^{n-1})L(1-n,\chi)$$

\noindent Θέλουμε να καταργήσουμε τον $p$-οστό παράγοντα στο γινόμενο \tl{Euler}, καθώς όπως αναφέραμε πριν αν το $p$ μπορεί να διαρεί 
το $1/n^s$, τότε η $p$-αδική απόλυτη τιμή θα παίρνει αυθαίρετα μεγάλες τιμές και έτσι το άθροισμα των στοιχείων δεν θα συγκλίνει.
Επιπλέον, αν το $\chi$ είναι περιττός χαρακτήρας τότε θα ισχύει ότι $B_{n,\chi\omega^{-n}}= 0$. Άρα η $p$-αδική $L$-συνάρτηση 
ενός περιττού χαρακτήρα είναι ταυτοτικά $0$. Ωστόσο, αν ο $\chi$ είναι άρτιος χαρακτήρας τότε το ίδιο δεν ισχύει. 
Οι ρίζες της $p$-αδικής $L$-συνάρτησης δεν μας είναι ακόμα πλήρως κατανοητές.

\begin{theorem}[Θεώρημα 5.12 στο \cite{Wash}] 
	Έστω $\chi$ ένας μη τετριμμένος χαρακτήρας και $p^2\nmid m_x$. (Υπενθυμίζουμε ξανά ότι αποφεύγουμε την περίπτωση $p=2$ αν και είναι εύκολα διαχειρίσιμη). Τότε
	$$\mathcal{L}_p(s,\chi) = a_0 + a_1(s-1) + a_2(s-1)^2 + \cdots$$ όπου $|a_0|_p \leq 1$ και $p | a_i$ για κάθε $i\geq 1$.
\end{theorem}

\begin{cor}
	Έστω $m,n$ ακέραιοι με $m\equiv n(\operatorname{mod}p-1)$ και $n\not\equiv 0(\operatorname{mod}p-1)$. Τότε ισχύει ότι
	$$\frac{B_m}{m} \equiv \frac{B_n}{n}(\operatorname{mod}p).$$
\end{cor}

\begin{proof} Καθώς $m\equiv n (\operatorname{mod}p-1)$ έχουμε ότι $\mathcal{L}_p(s,\omega^m)=\mathcal{L}_p (s,\omega^n)$. Ξέρουμε ότι
	$$\mathcal{L}_p (1-m,\omega^m) = -(1-p^{m-1})\frac{B_m}{m}$$ και όμοια για το $\mathcal{L}_p(s,\omega^n)$. Από το προηγούμενο θεώρημα και αφού $m,n\not\equiv 0 (\operatorname{mod}p-1)$ μπορούμε να γράψουμε
	$$\mathcal{L}_p(s,\omega^m) = a_0 + a_1(s-1)+a_2(s-1)^2 + \cdots$$ με $|a_0|_p\leq 1$ και $p|a_i$ για κάθε $i\geq 1$. Συνεπώς
	$$\mathcal{L}_p(1-m,\omega^m) = a_0 + a_1(-m) + a_2(-m^2)^2 + \cdots \equiv a_0 (\operatorname{mod}p)$$ και όμοια για το 
	$\mathcal{L}_p(1-n,\omega^n)$. Άρα, $\mathcal{L}_p(1-m,\omega^m) \equiv \mathcal{L}_p(1-n,\omega^n)(\operatorname{mod}p)$ δηλαδή έπεται το αποτέλεσμα.
\end{proof}


\noindent Λέμε ότι ένα στοιχείο $a/b \in \mathbb{C}_p$ είναι $p$-ακέραιο αν $|a/b|_p \geq 0$, δηλαδή όταν το $p$ δεν διαιρεί τον παρονομαστή 
στην απλοποιημένη μορφή. Άμεσο πόρισμα του προηγούμενου θεωρήματος είναι το ακόλουθο:

\begin{cor}\label{cor3.27}
	Έστω $m,n \in \mathbb{Z}$ και $\chi$ ένας μη τετριμένος χαρακτήρας με $p^2 \not| m_{\chi}$. Τότε το $\mathcal{L}_p (m,\chi)$ είναι $p$-ακέραιο και 
	$$\mathcal{L}_p(m,\chi) \equiv \mathcal{L}_p(n,\chi)(\operatorname{mod}p).$$
\end{cor}


\begin{cor}
Έστω $n$ ένας περιττός ακέραιος και $n\not\equiv -1 (\operatorname{mod}p-1)$. Τότε έχουμε
$$B_{1,\omega^n} \equiv \frac{B_{n+1}}{n+1}(\operatorname{mod}p)$$
\end{cor}


\begin{proof}
	Παρατηρούμε ότι $\omega^{n+1}\neq 1$, καθώς $n\neq -1 (\operatorname{mod}p-1)$. Επιπλέον, έχουμε ότι $\omega^n \neq 1$ 
	αφού το $n$ είναι περιττό και $\omega^n(p) = 0$. Από το θεώρημα \ref{wash5.11} έχουμε ότι
	$$\mathcal{L}_p(-n,\omega^{n+1}) = -(1-p^n)\frac{B_{n+1}}{n+1}$$ και 
	$$\mathcal{L}_p(0,\omega^{n+1}) = -(1-\omega^n(p))B_{1,\omega^n} = -B_{1,\omega^n}$$
	και καθώς τα $\mathcal{L}_p(-n,\omega^{n+1})$ και $\mathcal{L}_p(0,\omega^{n+1})$ είναι και τα δύο $p$-ακέραια 
	και ισότιμα $\operatorname{mod}p$ από το πόρισμα \ref{cor3.27}, έπεται το αποτέλεσμα.
\end{proof}


\begin{theorem}
	Έστω $p$ ένας περιττός πρώτος. Τότε $p|h_p^-$ αν και μόνο αν $p|B_j$ για κάποιο $j=2,4,\ldots,p-3$. Με το $p|B_j$ εννοούμε ότι το $p$ διαιρεί τον αριθμητή του $B_j$.
\end{theorem}

\begin{proof} Υπενθυμίζουμε την σχέση \ref{3.1}
	$$h^-_p = 2p \prod\limits_{\substack{j = 1 \\ j \text{ περιττός}}} \left(-\frac{1}{2} B_{1,\omega^j}\right)$$
	και ξεκινάμε με το
	$$2p \left(-\frac12 B_{1,\omega^{p-2}}\right) = -pB_{1,\omega^{p-2}}$$
	Έχουμε ότι $B_{1,\omega^{p-2}} = B_{1,\omega^{-1}}$. Από τον ορισμό του $B_{1,\omega^{-1}}$ και τον ορισμό των \tl{Bernoulli} πολυωνύμων στο \cite{Wash}, βλέπουμε ότι
	$$B_{1,\omega^{-1}} = \frac1p \sum\limits_{a=1}^{p-1} a \omega^{-1}(a)$$ Άρα έχουμε
	$$2p\left(-\frac12 B_{1,\omega^{p-2}}\right) = - \sum\limits_{a=1}^{p-1}a\omega^{-1}(a)$$ και καθώς $a \equiv \omega(a) \operatorname{mod}p$ παίρνουμε ότι
	$$2p \left(-\frac12 B_{1,\omega^{p-2}}\right) \equiv -(p-1) \equiv 1(\operatorname{mod}p)$$
	Συνεπώς,
	$$h^-_p \equiv \prod\limits_{\substack{j=1 \\ j \text{ περιττός }}}^{p-4} \left(-\frac12 B_{1,\omega^j}\right) (\operatorname{mod}p).$$ και εφαρμόζοντας το προηγούμενο πόρισμα παίρνουμε
	$$ \prod\limits_{\substack{j=1 \\ j \text{ περιττός }}}^{p-4} \left(-\frac12 B_{1,\omega^j}\right) \equiv  \prod\limits_{\substack{j=1 \\ j \text{ περιττός }}}^{p-4} \left(-\frac12 \frac{B_{j+1}}{j+1}\right)(\operatorname{mod}p).$$ από το οποίο έπεται το θεώρημα.
\end{proof}


\begin{defn}
	Λέμε ότι ένας πρώτος $p$ είναι \tl{irregular} αν $p|B_j$ για κάποιο $j=2,4,\ldots,p-3$. Διαφορετικά, τον λέμε κανονικό.
\end{defn}

\noindent Με το προηγούμενο θεώρημα έχουμε ότι το $p$ είναι \tl{irregular} πρώτος αν και μόνο αν $p|h^-_p$. Όπως θα δούμε στη 
συνέχεια ότι $p|h_p$ αν και μόνο αν $p|B_j$ για κάποιο $j=2,4,\ldots,p-3$. Άρα, το $p$ είναι \tl{irregular} αν και μόνο αν $p|h_p$. Αυτό χρησιμοποείται και ως εναλλακτικός ορισμός.

\begin{theorem}
	Υπάρχουν άπειροι \tl{irregular} πρώτοι.
\end{theorem}

\begin{proof} Έστω $p_1,\ldots,p_r$ είναι όλοι οι \tl{irregular} πρώτοι και έστω $m=N(p_1-1)\cdots(p_r-1)$ όπου το $N$ διαλέγεται αρκετά 
	μεγάλο ώστε $|B_m/m|>1$. Μπορούμε να το κάνουμε αυτό, εφόσον από το \cite{Wash} παίρνουμε ότι $|B_n/n|\rightarrow \infty$ 
	καθώς $n\rightarrow \infty$. Έτσι, υπάρχει πρώτος $p$ που διαιρεί τον αριθμητή του $B_m/m$. Καθώς τα $p_i$ βρίσκονται 
	στον παρονομαστή του $B_m$ για $i=1,\ldots,r$ από το θεώρημα \tl{Von Staudt-Clausen} (Θεώρημα 5.10 στο \cite{Wash}), δεν μπορούμε να έχουμε $p=p_i$ για κάποιο $i$. Επιπλέον, για τους ίδιους λόγους $m\not\equiv 0 (\operatorname{mod}p-1)$. Έστω $m^{\prime}\equiv m (\operatorname{mod}p-1)$ με $0<m^{\prime}<p-1$. Τότε 
	$$\frac{B_{m^{\prime}}}{m^{\prime}} \equiv \frac{B_m}{m} (\operatorname{mod}p)$$ και άρα $p|B_{m^{\prime}}$. Συνεπώς, το $p$ είναι \tl{irregular}. Έπεται ότι υπάρχουν άπειροι \tl{irregular} πρώτοι.
\end{proof}

\begin{conj} Υπάρχουν άπειροι κανονικοί πρώτοι.
\end{conj}

\noindent Στην συνέχεια θα δώσουμε την $p$-αδική έκδοση του τύπου του \tl{Dirichlet} για τον τύπο του αριθμού κλάσεων. Εκτός 
από την αναλογία του με τον κλασικό τύπο, αυτό χρησιμοποιείται για να δωθεί ένα ακόμα αποτέλεσμα στους αριθμούς κλάσεων. 
Το αποτέλεσμα αυτό στο οποίο δεν θα επικεντρωθούμε είναι το ακόλουθο:

\begin{theorem} Έστω $\chi \neq 1$ ένας άρτιος χαρακτήρας \tl{Dirichlet}. Τότε $\mathcal{L}_p(1,\chi) \neq 0$.
\end{theorem}

\noindent Θα διατηρήσουμε ως στόχο το παρακάτω θεώρημα.

\begin{theorem} \label{3.33}
	Αν $p|h^+_p$, τότε $p|h^-_p$. Ειδικότερα, $p|h_p$ αν και μόνο αν $p|B_j$ για κάποιο $j=2,4,\ldots, p-3$.
\end{theorem}


\noindent Για να το αποδείξουμε χρειαζόμαστε τον $p$-αδικό τύπο του αριθμού κλάσεων του \tl{Dirichlet} μαζί με τον ορισμό 
του $p$-αδικού \tl{regulator} $R_p$, καθώς και μια επιπλέον πρόταση. Για τις αποδείξεις και τον πλήρη ορισμό του $p$-αδικού 
\tl{regulator} μπορεί να ανατρέξει κανείς στο κεφάλαιο $5$ στο \cite{Wash} ή στο \cite{Milne1}.

\begin{theorem} Έστω $K$ ένα πλήρως πραγματικό αβελιανό σώμα αριθμών με $[K:\Q]=n$. Έστω $X$ η ομάδα χαρακτήρων που αντιστοιχεί στο $K$. Τότε έχουμε
	$$\prod\limits_{\substack{\chi \in X \\ \chi \neq 1}} \left(1-\frac{\chi(p)}{p}\right)^{-1} \mathcal{L}_p(1,\chi) = \frac{2^{n-1}h_K R_p(K)}{\sqrt{\Delta_K}}$$
\end{theorem}


\begin{prop}[Πρόταση 5.33 στο \cite{Wash}]\label{3.35}
	Έστω $K$ πλήρως πραγματικό \tl{Galois} σώμα. Αν υπάρχει μόνο ένας πρώτος $\mathfrak{p}$ ετσι ώστε $\mathfrak{p}|p$ και αν 
	$e(\mathfrak{p} \mid p)\leq p-1$, τότε
	$$\left|\frac{[K:\Q] R_p(K)}{\sqrt{\Delta_K}}\right|_p \leq 1$$
\end{prop}

%\begin{proof}
	%Δες \tl{reference washington proposition 5.33}.
%\end{proof}

\begin{proof} (Του θεωρήματος \ref{3.33})

	$ $\newline
	Υπενθυμίζουμε ότι η ομάδα χαρακτήρων που αντιστοιχεί στο $\Q(\zeta_p)^+$ είναι η $$\{1,\omega^2,\omega^4,\ldots,\omega^{p-3}\}$$
	δηλαδή οι άρτιοι χαρακτήρες του $\Q(\zeta_p)$, όπου όλοι τους δίνουν την τιμή $0$ στο $p$. Εφόσον το $\Q(\zeta_p)^+$ είναι πλήρως 
	πραγματικό, παίρνουμε από τον $p$-αδικό τύπο του \tl{Dirichlet}
	\begin{equation} \label{p-int}
		\prod\limits_{\substack{j=2 \\ j \text{ άρτιο }}}^{p-3} \mathcal{L}_p(1,\omega^j) = \frac{2^{n-1}h_p^+ R_p^+}{\sqrt{\Delta_p^+}}
	\end{equation}

	\noindent και το $\Q(\zeta_p)^+$ ικανοποιεί τις υποθέσεις της πρότασης \ref{3.35}, εφόσον γνωρίζουμε ότι
	$$[\Q(\zeta_p)^+:\Q] = \frac{p-1}2$$ δηλαδή, έχουμε υποχρεωτικά ότι $e \leq p-1$ και στην μεγαλύτερη επέκταση υπάρχει 
	μοναδικός πρώτος πάνω από το $p$, εφόσον ξέρουμε από την αλγεβρική θεωρία αριθμών ότι
	
	$$p\mathcal{O}_{\Q(\zeta_p)} = (1-\zeta_p)^{p-1}$$ Συνεπώς:

	$$\left| \frac{[\Q(\zeta_p)^+ :\Q]R_p^+}{\sqrt{\Delta^+_p}}\right|_p = \left|\frac{R_p^+}{\sqrt{\Delta^+_p}} \right|_p \leq 1$$

	\noindent Άρα, αυτή η ποσότητα είναι $p$-ακέραια, οπότε αν υποθέσουμε ότι $p |h^+$ τότε δεν μπορεί το $p$ να 
	διαιρεί τον παρονομαστή $\sqrt{\Delta^+_p}$ και έτσι μπορούμε να καταλήξουμε στο ότι το $p$ θα διαιρεί και το άλλο μέλος της 
	ισότητας (\ref{p-int}). Δηλαδή, έχουμε ότι $p|\mathcal{L}_p(1,\omega^{j})$ για κάποιο $j \in \{2,4,\ldots,p-3\}$. Από το 
	πόρισμα \ref{cor3.27} παίρνουμε ότι
	\begin{align*}
		-B_{1,\omega^{j-1}} &= -\left(1- \omega^{j-1}(p)\right)B_{1,\omega^{j-1}} \\
		&= \mathcal{L}_p(0,\omega^j) \\
		&\equiv \mathcal{L}_p(1,\omega^j)(\operatorname{mod}p) \\
		&\equiv 0(\operatorname{mod}p)
	\end{align*}

	\noindent και καθώς τα $B_{1,\omega^i}$ είναι $p$-ακέραια και 
	$$h^-_p \equiv \prod\limits_{\substack{i=1 \\ i \text{ περιττό}}}^{p-4} \left(-\frac12 B_{1,\omega^i}\right) 
	(\operatorname{mod}p)$$ έπεται ότι 
	$$h^-_p \equiv 0 (\operatorname{mod}p)$$
\end{proof}


	\noindent Κλείνουμε την ενότητα με μια εικασία, ότι οι συνθήκη για το θεώρημα \ref{3.33} δεν ικανοποιείται στην πραγματικότητα ποτέ.
	\begin{conj}[\tl{Vandiver}]
		Για κάθε $p$ πρώτο αριθμό έχουμε $p\nmid h^+_p$. 
	\end{conj}
\section{Κυκλοτομικά Σώματα}


\begin{defn}Μια πρωταρχική $n$-οστή ρίζα της μονάδας είναι ένας αριθμός $\zeta_n \in \mathbb{C}$ τέτοιος ώστε $\zeta_n^n = 1$ και 
	$\zeta_n^m \neq 1$ για κάθε $0<m<n$. Το σώμα $\Q(\zeta_n)$ λέγεται το $n$-οστό κυκλοτομικό σώμα. 
\end{defn}

\noindent Ορίζουμε το $n$-οστό κυκλοτομικό πολυώνυμο $\Phi_n(x)$ ως εξής:
$$\Phi_n(x) = \prod\limits_{\substack{ 0 < m < n \\ \gcd(m,n)=1}} (x - \zeta_n^m) \ \in \Z[x]$$

\noindent Οι ρίζες του πολυωνύμου είναι ακριβώς οι πρωταρχικές $n$-οστές ρίζες της μονάδας. Έχουμε ότι $\deg (\Phi_n) = \phi(n)$. 
Επιπλέον, ισχύει ότι $\Phi_n(x) \in \Q[x]$. Αυτό φαίνεται καλύτερα από την παρακάτω σχέση

\begin{equation} \label{eq2.2}
	x^n-1 = \prod\limits_{d \mid n} \Phi_d(x)
\end{equation}

\noindent και κάνοντας επαγωγή στο $n$. Εφόσον $\Phi_n(\zeta_n) = 0$, έχουμε ότι $[\Q(\zeta_n):\Q] \leq \phi(n)$. Επιπλέον, έχουμε 
ότι η επέκταση $\Q(\zeta_n)/\Q$ είναι \tl{Galois}, καθώς το $\Phi_n$ διασπάται πλήρως στο $\Q(\zeta_n)$. Εφαρμόζοντας τον μετασχηματισμό 
\tl{Möbius} στην εξίσωση (\ref{eq2.2}) παίρνουμε:
$$\Phi_n(x) = \prod\limits_{d\mid n} (x^d-1)^{\mu(n/d)}$$

\begin{lemma}Έστω $n=p^r$ όπου $p$ πρώτος. Τότε:
	\begin{enumerate}
	\item $[\Q(\zeta_{p^r}):\Q] = \phi(p^r) = p^r-p^{r-1}$.
	%ως ιδεώδη:
	\item $p \mathcal{O}_{\Q(\zeta_{p^r})} = (1-\zeta_{p^r})^{\phi(p^r)}$ και το $(1-\zeta_{p^r})$ είναι πρώτο ιδεώδες του $\mathcal{O}_{\Q(\zeta_{p^r})}$.
	\item $\mathcal{O}_{\Q(\zeta_{p^r})} = \Z[\zeta_{p^r}]$.
	\item $\Delta_{\Q(\zeta_{p^r})} = \pm p^{p^{r-1}(pr-r-1)}$. 
	\end{enumerate}
\end{lemma}

%\begin{proof}
	%Άρχικά έχουμε $\Z[\zeta_{p^r}]\subseteq \mathcal{O}_{\Q(\zeta_{p^r})}$ αφού τα στοιχεία του πρώτου είναι άθροισμα ακεραίων της μορφής $\sum\limits_{i=0}^{p^r-1} a_i \zeta_{p^r}^i$ και τα ακέραια στοιχεία αποτελούν δακτύλιο.
	%Αν $\zeta_{p^r}^{\prime}$ είναι μια άλλη $p^r$ ρίζα της μονάδας, τότε υπάρχουν $s,t\in\Z$ με $p\not\mid st$ και $\zeta_{p^r} = (\zeta_{p^r}^{\prime})^t, \zeta_{p^r}^{\prime} = \zeta_{p^r}^s$. Έτσι, $\Q(\zeta_{p^r}) = \Q(\zeta_{p^r}^{\prime})$ και $\Z[\zeta_{p^r}] = \Z[\zeta_{p^r}^{\prime}]$. Επιπλέον,
	
	%$$\frac{1-\zeta^{\prime}_{p^r}}{1-\zeta_{p^r}} = \frac{1-\zeta^s_{p^r}}{1-\zeta_{p^r}} = 1 + \zeta_{p^r} + \cdots + \zeta_{p^r}^{s-1} \in \Z[\zeta_{p^r}]$$ και όμοια, $(1-\zeta_{p^r})/(1-\zeta_{p^r}^{\prime}) \in \Z[\zeta_{p^r}]$. Αρα το $(1-\zeta^{\prime}_{p^r})$ είναι αντιστρέψιμο στο $\Z[\zeta_{p^r}]$ και άρα και στο $\mathcal{O}_{\Q(\zeta_{p^r})}$.

	%$$\Phi_{p^r} (x) = \frac{x^{p^r}-1}{x^{p^{r-1}}-1} = \frac{t^p - 1}{t-1} = 1+t+\cdots + t^{p-1}, \ t = x^{p^{r-1}}$$ και $\Phi_{p^r}(1) = p$. Από τους ορισμούς φαίνεται ότι:
	%\begin{align*}
		%\Phi_{p^r}(1) &= \prod (1-\zeta^{\prime}_{p^r}) \\
		%&= \prod \frac{1-\zeta^{\prime}_{p^r}}{1-\zeta_{p^r}}(1-\zeta_{p^r}) \\
		%&= u (1-\zeta_{p^r})^{\phi(p^r)} 
	%\end{align*}
	%με $u$ αντιστρέψιμο στοιχείο του $\Z[\zeta_{p^r}]$. Άρα παίρνουμε ισότητα στα ιδεώδη του $\mathcal{O}_{\Q(\zeta_{p^r})}$, δηλαδή $p\mathcal{O}_{\Q(\zeta_{p^r})} = (1-\zeta_{p^r})^{\phi(p^r)}$. Συνεπώς, το ιδεώδες $p\mathcal{O}_{\Q(\zeta_{p^r})}$ έχει τουλάχιστον $\phi(p^r)$ πρώτους παράγοντες στο $\mathcal{O}_{\Q(\zeta_{p^r})}$. Άρα (?) παίρνουμε $[\Q(\zeta_{p^r}):\Q] \geq \phi(p^r)$ και συνεπώς 
	%$$[\Q(\zeta_{p^r}):\Q] = \phi(p^r) = p^r - p^{r-1}$$
	%Επιπλέον, το $(1-\zeta_{p^r})$ παράγει πρώτο ιδεώδες αλλιώς θα είχαμε παραπάνω από $\phi(p^r)$ πρώτους στην παραγοντοποίηση του $p\mathcal{O}_{\Q(\zeta_{p^r})}$. Για την διακρίνουσα, χρησιμοποιούμε τον τύπο με την παράγωγο από την βιβλιογραφία (π.χ. \tl{Milne ANT prop 2.33})
	
	%$$disc(\Z[\zeta_{p^r}]/\Z) = \pm N_{\Q(\zeta_{p^r})/\Q} (\Phi^{\prime}_{p^r}(\zeta_{p^r}))$$. Έχουμε
	%$$\Phi^{\prime}_{p^r}(\zeta_{p^r}) = \frac{p^r \zeta_{p^r}^{p^r-1}}{\zeta_{p^r}^{p^{r-1}}-1}$$ και
	%$$N(\zeta_{p^r}) = \pm 1$$ αρα
	%$$N(p^r) = (p^r)^{\phi(p^r)} = p^{r\phi(p^r)}$$ και ισχυριζόμαστε ότι:
	%$$N(1-\zeta_{p^r}^{p^s}) = p^{p^s}, \ 0\leq s < r$$
	%Πράγματι, το ελάχιστο πολυώνυμο του $1-\zeta_{p^r}$ είναι το $\Phi_{p^r}(1-x)$ που έχει σταθερό όρο $\Phi_{p^r}(1) = p$. Αρα $N(1-\zeta_{p^r})= \pm p$. Έστω $s<r$, το $\zeta_{p^r}^{p^s}$ είναι πρωταρχική $p^{r-s}$-οστή ρίζα της μονάδας, άρα ο ίδιος υπολογισμός για $r-s$ αντί για $r$ δίνει $N_{\Q(\zeta_{p^r}^{p^s})/\Q}(1-\zeta^{p^s}_{p^r}) = \pm p$. Χρησιμοποιώντας την προσεταιριστικότητα  της νόρμας, μαζί με $N_{M/L}(a) = a^{[M:L]}$ για σώματα $M\supset L$, παίρνουμε ότι:
	%$$N_{\Q(\zeta_{p^r})/\Q} (1-\zeta^{p^s}_{p^r}) = p^a$$ όπου
	%$$a = [\Q(\zeta_{p^r}):\Q(\zeta^{p^s}_{p^r})] = \phi(p^r)/\phi(p^{r-s}) = p^s$$

	%Συνεπώς, $N(\Phi^{\prime}_{p^r}(\zeta_{p^r})) =\pm p^c$ όπου $c = p^{r-1}(pr-r-1)$. Άρα η διακρίνουσα του $\Z[\zeta_{p^r}]$ πάνω από το $\Z$ είναι δύναμη του $p$. Άρα και η διακρίνουσα του $\mathcal{O}_{\Q(\zeta_{p^r})}$ πάνω από το $\Z$ είναι δύναμη του $p$ από τον τύπο:
	%$$disc(\mathcal{O}_{\Q(\zeta_{p^r})}/\Z) [\mathcal{O}_{\Q(\zeta_{p^r})}: \Z[\zeta_{p^r}]]^2 = disc(\Z[\zeta_{p^r}/\Z])$$
	%(\tl{Milne remark 2.24})

	%τι πηλίκο είναι το παρακάτω; το p^M κάνει annihilate
	%Επιπλέον, έχουμε ότι το $[\mathcal{O}_{\Q(\zeta_{p^r})}: \Z[\zeta_{p^r}]]$ είναι δύναμη του $p$, άρα $p^M (\mathcal{O}_{\Q(\zeta_{p^r})}/\Z[\zeta_{p^r}]) = 0$ για κάποιο $M$. Δηλαδή, $p^M \mathcal{O}_{\Q(\zeta_{p^r})} \subseteq \Z[\zeta_{p^r}]$. Το χρησιμοποιούμε αυτό για το ιδεώδες $\p = (1-\zeta_{p^r})$ και έχουμε $f(\p/p) =1$ και άρα η παρακάτω απεικόνιση είναι ισομορφισμός:

	%$$\Z / p\Z \longrightarrow \mathcal{O}_{\Q(\zeta_{p^r})/(1-\zeta_{p^r})}$$
	%Άρα $\Z + (1-\zeta_{p^r})\mathcal{O}_{\Q(\zeta_{p^r})} = \mathcal{O}_{\Q(\zeta_{p^r})}$ και άρα επίσης:
	%\begin{equation} \label{eq2.3}
		%\Z[\zeta_{p^r}] + (1-\zeta_{p^r})\mathcal{O}_{\Q(\zeta_{p^r})} = \mathcal{O}_{\Q(\zeta_{p^r})} 
	%\end{equation}
	%η οποία δίνει:
	%\begin{equation} \label{eq2.4}
		%(1-\zeta_{p^r}) \Z[\zeta_{p^r}] + (1-\zeta_{p^r})^2 \mathcal{O}_{\Q(\zeta_{p^r})} = (1-\zeta_{p^r}) \mathcal{O}_{\Q(\zeta_{p^r})}
	%\end{equation}

	%Έστω $a \in \mathcal{O}_{\Q(\zeta_{p^r})}$. Τότε από την εξίσωση \ref{eq2.3} παίρνουμε ότι $a=a^{\prime} + \gamma$ με $a^{\prime} \in (1-\zeta_{p^r})\mathcal{O}_{\Q(\zeta_{p^r})}$ και $\gamma \in \Z[\zeta_{p^r}]$. Η εξίσωση \ref{eq2.4} δίνει $a^{\prime} = a^{\prime\prime} + \gamma^{\prime}$ με $a^{\prime\prime} \in (1-\zeta_{p^r})^2 \mathcal{O}_{\Q(\zeta_{p^r})}$ και $\gamma^{\prime} \in \Z[\zeta_{p^r}]$. Άρα $a = (\gamma+\gamma^{\prime}) + a^{\prime\prime}$. Συνεπώς:
	%$$\Z[\zeta_{p^r}]+ (1-\zeta_{p^r})^2 \mathcal{O}_{\Q(\zeta_{p^r})} = \mathcal{O}_{\Q(\zeta_{p^r})}$$

	%Με επανάληψη, μπορούμε να πάρουμε $\Z[\zeta_{p^r}] + (1-\zeta_{p^r})^m \mathcal{O}_{\Q(\zeta_{p^r})} = \mathcal{O}_{\Q(\zeta_{p^r})}$ για $m \in \mathbb{N}$. Καθώς $(1-\zeta_{p^r})^{\phi(p^r)} = p\cdot u$, $u$ αντιστρέψιμο, έχουμε $\Z[\zeta_{p^r}]+ p^m \mathcal{O}_{\Q(\zeta_{p^r})} = \mathcal{O}_{\Q(\zeta_{p^r})}$ για κάθε $m\in\mathbb{N}$. Ωστόσο, για αρκετά μεγάλο $m$ έχουμε δείξει ότι $p^m \mathcal{O}_{\Q(\zeta_{p^r})} \subseteq \Z[\zeta_{p^r}]$. Άρα πράγματι $\Z[\zeta_{p^r}] = \mathcal{O}_{\Q(\zeta_{p^r})}$. Αυτό μαζί με τον υπολογισμό του $disc(\Z[\zeta_{p^r}]/\Z)$ ολοκληρώνουν την απόδειξη.  
%\end{proof}

\noindent Χρησιμοποιώντας το ακόλουθο λήμμα, το παραπάνω αποτέλεσμα γενικεύεται για τα κυκλοτομικά σώματα $\Q(\zeta_n)$ όπου 
$n \in \mathbb{N}$.
%milne lemma 6.5
\begin{lemma}
	Έστω $K,L$ πεπερασμένες επεκτάσεις του $\Q$ με
	$$[KL:\Q] = [K:\Q]\cdot [L:\Q]$$ και έστω $d = \gcd \left( \operatorname{disc}(\mathcal{O}_K/\Z), \operatorname{disc}(\mathcal{O}_L/\Z)\right)$. Τότε
	$$O_{KL} \subset d^{-1} \mathcal{O}_K \mathcal{O}_L$$
\end{lemma}

\begin{prop}
	Έστω $\zeta_n$ μια πρωταρχική $n$-οστή ρίζα της μονάδας και $K= \Q(\zeta_n)$. Ισχύουν τα ακόλουθα:
	\begin{enumerate}
		\item $[K:\Q] = \phi(n)$.
		\item $\mathcal{O}_K = \Z[\zeta_n]$.
		\item Ο πρώτος $p$ διακλαδίζεται στο $K$ αν και μόνο αν $p\mid n$ (εκτός αν $n=2\cdot$περιττός και $p=2$). Ειδικότερα, αν $n=p^r$ με $\gcd(p,m)=1$, τότε 
		$$p \mathcal{O}_K = (\p_1 \cdots \p_s)^{\phi(p^r)}$$ στο $K$, με τα $\p_i$ να είναι διακεκριμένοι πρώτοι του $K$. %πρώτα ιδεώδη στο O_K αλλά το ίδιο είναι αν δω τα ιδεώδη που παράγονται από τους πρώτους
	\end{enumerate}
\end{prop}

%\begin{proof}

	%Με επαγωγή στο πλήθος των πρώτων που διαιρούν το $n$. Θεωρούμε τα σώματα:
	%\begin{figure}[H]
		%\centering
	%\begin{tikzcd}
		%& K = \Q(\zeta_n)                            &                                   \\
%E=\Q(\zeta_{p^r}) \arrow[ru, no head] &                                            & F=\Q(\zeta_m) \arrow[lu, no head] \\
		%& \Q \arrow[lu, no head] \arrow[ru, no head] &                                  
%\end{tikzcd}
%\end{figure}
%και κοιτάμε πώς το $p$ παραγοντοποιείται στα $E,F$.
%$p\mathcal{O}_E = \p^{\phi(p^r)}$ διακλαδίζεται πλήρως όπως δίνεται από το προηγούμενη πρόταση.
%$p\mathcal{O}_F = \p_1\cdots \p_r$ δεν διακλαδίζεται καθώς το $p$ είναι σχετικά πρώτο με την διακρίνουσα.

%Τώρα, κοιτάμε την παραγοντοίηση
%\end{proof}

\begin{remark}
	Εφόσον $\phi(p^r) = p^{r-1}(p-1)$, αν το $n$ είναι $2$ επί κάποιον περιττό αριθμό και $p=2$, τότε θα ισχύει η περίπτωση που το 
	$2$ διαιρεί το $n$ αλλά δεν διακλαδίζεται. Επιπλέον, σε αυτήν την περίπτωση έχουμε ότι $\Q(\zeta_n) = \Q(\zeta_{2n})$.
\end{remark}

\begin{remark} Έστω $K = \Q(\zeta_p)$. Οι μόνες ρίζες της μονάδας που βρίσκονται στο $K$ είναι οι $\zeta_p^s$ για τα $1\leq s \leq p-1$. 
	Αυτό έπεται από το αποτέλεσμα ότι $\Q(\zeta_m)\cap \Q(\zeta_n) = \Q$ αν $\gcd(m,n) =1 $.
\end{remark}


\noindent Μπορούμε να πούμε κάποια παραπάνω αποτελέσματα σχετικά με το πως ένας πρώτος $p$ διασπάται στο $\Q(\zeta_n)$ αν $p\nmid n$.

\begin{lemma} Έστω $p$ ένας πρώτος τέτοιος ώστε $p\nmid n$. Έστω $\p$ πρώτος του $\Q(\zeta_n)$ που στέκεται πάνω από το $p$. Τότε οι $n$-οστές ρίζες της μονάδας είναι διακεκεριμένες \tl{modulo} $\p$.
\end{lemma}

\begin{lemma} Έστω $p$ ένας πρώτος τέτοιος ώστε $p\nmid n$. Έστω $f$ να είναι ο μικρότερος θετικός ακέραιος τέτοιος ώστε $p^f \equiv 1 (\operatorname{mod}n)$. Τότε ο $p$ διασπάται σε $\phi(n)/f$ διακεκριμένους πρώτους του $\Q(\zeta_n)$, όπου για τον καθένα η αντίστοιχη τάξη του σώματος υπολοίπων είναι $f$. Ειδικότερα, το $p$ διασπάται πλήρως αν και μόνο αν $p\equiv 1 (\operatorname{mod}n)$.
\end{lemma}
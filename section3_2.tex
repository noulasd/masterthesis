\section{$L$-συναρτήσεις}

Σε αυτό το σημείο θα αναφέρουμε κάποια βασικά αποτελέσματα για τις $L$-συναρτήσεις που συνδέονται με τους χαρακτήρες. 
Αυτά χρειάζονται ώστε αργότερα να είναι ξεκάθαρες οι αναλογίες όταν θα κοιτάμε τοπικά στις $p$-αδικές $L$-συναρτήσεις. Όπως αναφέραμε πιο πριν, 
οι αποδείξεις τους είναι αναλυτικής φύσεως και για αυτές μπορεί κάποιος να ανατρέξει στο \cite{Wash}. Γενικά, για ένα αντικείμενο $A$ αριθμο-θεωρητικού 
ενδιαφέροντος υπάρχει μια $L$-συνάρτηση που του αντιστοιχεί, δηλαδή ένα αντικείμενο το οποίο επεξεργαζόμαστε με εργαλεία της μιγαδική ανάλυσης. 
Με βάση αυτά, οι τιμές που παίρνει η $L$-συνάρτηση επιστρέφουν πίσω πληροφορία για το αντικείμενο $A$. Έτσι, με το σύνηθες \tl{Hasse principle} 
κοιτάμε αυτές τις τιμές τοπικά στις ανάλογες $\mathcal{L}_p$ που είναι οι $p$-αδικές εκδόσεις των $L$-συναρτήσεων. Το αντικείμενο εδώ για το οποίο μας 
ενδιαφέρει να πάρουμε πληροφορίες είναι ο αριθμός κλάσεων $h_n$ του σώματος αριθμών $\Q(\zeta_n)$. 

Υπενθυμίζουμε ότι αν έχουμε έναν χαρακτήρα \tl{Dirichlet} με \tl{conductor} $k$
$$\chi^{\prime}: (\Z/k\Z)^\times \longrightarrow \mathbb{C}^\times$$ μπορούμε να τον βλέπουμε ως αριθμητική συνάρτηση
$$\chi:\Z\longrightarrow \mathbb{C}$$
$$\chi(n) = \chi^{\prime}([n]_k)$$ που πληρεί τις ιδιότητες:
\begin{enumerate}
	\item $\chi(mn) = \chi(m)\chi(n)$ για κάθε $m,n \in \Z$.
	\item $|\chi(n)| = \begin{cases} 1, \quad \text{ αν } (n,k)=1 \\ 0, \quad \text{ αλλιώς}\end{cases}$.
	\item $\chi(n+km) = \chi(n)$ για κάθε $n,m \in \Z$.
	\item $\chi^{\phi(k)}(n) = 1$ για $(n,k)=1$
\end{enumerate}

\begin{defn} Έστω $\chi$ ένας χαρακτήρας \tl{Dirichlet} με \tl{conductor} $k$ που επεκτείνεται στο $\Z$ όπως παραπάνω. Η $L$-συνάρτηση που αντιστοιχεί στο $\chi$ είναι η εξής
	$$L(s,\chi) = \sum\limits_{n=1}^\infty \chi(n)n^{-s}, \quad \operatorname{Re}(s)>1.$$
\end{defn}

\noindent Είναι γνωστό ότι η $L(s,\chi)$ δέχεται αναλυτική συνέχιση σε όλο το $\mathbb{C}$ αν $\chi\neq 1$. Για $\chi = 1$ ειναι η γνωστή 
συνάρτηση ζήτα του \tl{Riemann}, που έχει μερόμορφη συνέχιση σε όλο το $\mathbb{C}$, εκτός από έναν απλό πόλο με ολοκληρωτικό υπόλοιπο 1 στο $s=1$. 
Η $L$-συνάρτηση $L(s,\chi)$ έχει γινόμενο \tl{Euler} που συγκλίνει:
$$L(s,\chi) = \prod\limits_{p} (1-\chi(p)p^{-s})^{-1},\quad \operatorname{Re}(s)>1.$$

\noindent Υπενθυμίζουμε ότι οι αριθμοί \tl{Bernoulli} $B_m$ ορίζονται ως:
$$\frac{t}{e^t-1} = \sum\limits_{m=0}^\infty B_m \frac{t^m}{m!}$$ και όμοια ορίζονται οι γενικευμένοι αριθμοί \tl{Bernoulli} $B_{m,\chi}$ 
για έναν χαρακτήρα με \tl{conductor} $n$:
$$\sum\limits_{j=1}^n \frac{\chi(j) t e^{jt}}{e^{nt}-1} = \sum\limits_{m=0}^\infty B_{m,\chi} \frac{t^m}{m!}$$

\noindent Οι γενικευμένοι αριθμοί \tl{Bernoulli} έχουν σημαντικό ρόλο καθώς σχετίζονται με ειδικές τιμές που παίρνουν οι $L$-συναρτήσεις. 
Πιο συγκεκριμένα, ισχύει το ακόλουθο θεώρημα:

%ref wash theorem 4.2
\begin{theorem}[Θεώρημα 4.2 στο \cite{Wash}] \label{wash4.2}
	Για $m\geq 1$ έχουμε
	$$L(1-m,\chi) = - \frac{B_{m,\chi}}{m}.$$
\end{theorem}

$ $\newline
Έστω $K$ ένα αβελιανό σώμα αριθμών και $X$ η ομάδα χαρακτήρων που του αντιστοιχεί. Έχουμε ότι υπάρχουν $r_1 + 2r_2$ εμφυτεύσεις του $K$ στο 
$\mathbb{C}$, όπου $r_1$ είναι το πλήθος των πραγματικών εμφυτεύσεων και $r_2$ είναι το πλήθος των ζευγαριών των συζυγών μιγαδικών 
εμφυτεύσεων. Επιπλέον, έχουμε $R_K$ να είναι το \tl{regulator} του $K$, $\omega_K$ ο αριθμός των ριζών της μονάδας που περιέχει το $K$, $D_K$ 
η διακρίνουσα του $K$ και $h_K$ ο αριθμός κλάσεων του $K$. Οι ακριβείς ορισμοί των παραπάνω μεγεθών υπάρχουν στο \cite{Milne1}.

Με τα παραπάνω, είμαστε σε θέση να γράψουμε τον τύπο του \tl{Dirichlet} για τους αριθμούς κλάσεων. Όπως είναι ξεκάθαρο, ο τύπος αυτός ακολουθεί την φιλοσοφία 
των $L$-συναρτήσεων που αναφέραμε προηγουμένως.

\begin{theorem}[\tl{Dirichlet's Class Number Formula}]
	Έστω $K$ ένα αβελιανό σώμα αριθμών και $X$ η ομάδα χαρακτήρων που του αντιστοιχεί. Για τα μεγέθη όπως παραπάνω έχουμε ότι
	$$\prod\limits_{\substack{\chi \in X \\ \chi \neq 1} } L(1,\chi) = \frac{2^{r_1} (2\pi)^{r_2} R_K}{\omega_K \sqrt{|D_K|}} \cdot h_K.$$
\end{theorem}

$ $\newline
Με αυτό το θεώρημα έχουμε μια θεωρητική μέθοδο για να υπολογίζουμε τον αριθμό κλάσεων ενός αβελιανού σώματος αριθμών, ωστόσο χρειάζεται να γίνει υπολογισμός 
του \tl{regulator}. Εκείνος απαιτεί την εύρεση μιας βάσης της $\mathcal{O}_K^\times$. Συνήθως, η εύρεση μιας τέτοιας βάσης απαιτεί πολλούς υπολογισμούς 
για να είναι πρακτική η μέθοδος. Για αυτό, θα χωρίσουμε τον αριθμό κλάσεων σε δύο παράγοντές του και θα δουλέψουμε με τον έναν εκ των οποίων είναι ευκολότερο. 

Θα αναφέρουμε το επόμενο θεώρημα εστιάζοντας στο $K = \Q(\zeta_n)$, ωστόσο ισχύει το γενικότερο όχι μόνο για τα 
αβελιανά σώματα αριθμών, αλλά και για τα λεγόμενα \tl{CM}-σώματα (\tl{complex multiplication}). Ένα σώμα λέγεται {\em πλήρως πραγματικό} αν όλες του 
οι εμφυτεύσεις στο $\mathbb{C}$ βρίσκονται μέσα στο $\mathbb{R}$, ενώ αντίθετα {\em πλήρως φανταστικό} αν καμία τέτοια εμφύτευση δεν βρίσκεται 
ολόκληρη μέσα στο $\mathbb{R}$. Ένα \tl{CM}-σώμα είναι μια πλήρως φανταστική τετραγωνική επέκταση ενός πλήρως πραγματικού σώματος. 
Μπορούμε να πάρουμε ένα τέτοιο σώμα $K$ ξεκινώντας από ένα πλήρως πραγματικό $F$ και επισυνάπτοντας μια ρίζα $\sqrt{a}$, για κάποιο στοιχείο 
$a \in F \subseteq \mathbb{R}$ όπου το $Irr(a,\Q)$ να έχει μόνο μιγαδικές ρίζες. Ισοδύναμα, για κάθε εμφύτευση $\sigma$ του $F$ στους πραγματικούς 
να ισχύει ότι $\sigma(a)<0$. Όλα τα σώματα $\Q(\zeta_n)$ είναι \tl{CM}-σώματα, αφού έχουν ως μέγιστο πραγματικό υπόσωμα το $\Q(\zeta_n+\zeta_n^{-1}) 
= \Q(\cos 2\pi /n)$ και παίρνουμε το $\Q(\zeta_n)$ επισυνάπτοντας την ρίζα του $\zeta_n^2 + \zeta_n^{-2} - 2$, η οποία είναι η διακρίνουσα 
του αναγώγου πολυωνύμου $x^2 -(\zeta_n + \zeta_n^{-1})x + 1$ με μιγαδικές ρίζες $\zeta_n,-\zeta_n$. Αν $h_n$ είναι ο αριθμός κλάσεων του $\Q(\zeta_n)$ 
και $h^+_n$ είναι ο αριθμός κλάσεων του $\Q(\zeta_n+\zeta_n^{-1})$, τότε έχουμε το ακόλουθο θεώρημα:

\begin{theorem}
	Για κάθε θετικό ακέραιο $n$ ισχύει ότι $h_n^+ | h_n$.
\end{theorem}

\noindent Για την απόδειξη του θεωρήματος χρειάζεται η ακόλουθη πρόταση που βασίζεται στην θεωρία κλάσεων σωμάτων.

\begin{prop}
	Έστω $K/E$ μια επέκταση σωμάτων αριθμών που δεν έχει μη τετριμμένη ενδιάμεση αδιακλάδωτη αβελιανή επέκταση $F/E$. Τότε $h_E|h_K$.
\end{prop}

\noindent Υπενθυμίζουμε ότι μια επέκταση $F/E$ είναι αδιακλάδωτη αν κάθε επέκταση πραγματικής εμφύτευσης του $E$ σε εμφύτευση του $F$ παραμένει πραγματική και κάθε πρώτο ιδεώδες του $\mathcal{O}_E$ παραμένει αδιακλάδωτο στο $\mathcal{O}_F$.

\begin{proof}
	Από την θεωρία κλάσεων σωμάτων, υπάρχει το σώμα κλάσεων \tl{Hilbert} $H_E$ του $E$ που είναι η μέγιστη αδιακλάδωτη αβελιανή επέκταση του $E$ τέτοιο ώστε 
	$$C_E \cong \Gal(H_E/E),$$ όπου $C_E$ είναι η ομάδα κλάσεων του $E$. Η υπόθεση μας λέει ότι $K\cap H_E = E$ και άρα
	$$h_E = [H_E:E]=[H_EK:K],$$ όπου χρησιμοποιούμε ότι $\Gal(H_EK/K) \cong \Gal(H_E/H_E\cap K) = \Gal(H_E/E)$. Αυτός ο ισομορφισμός μας λέει ότι και η $H_EK/K$ είναι αβελιανή. Επιπλέον, εφόσον η $H_E/E$ είναι αδιακλάδωτη θα παραμένει αδιακλάδωτη και η μεταφορά $H_EK/K$. Πράγματι, αν θεωρήσουμε έναν πρώτο $q$ του $HK$ που βρίσκεται πάνω από έναν πρώτο $p$ του $K$ τότε

	$$e=e(q \mid p) = |I_q| = \#\{ \sigma \in \Gal(H_EK/K): \ \sigma(a) \equiv a \mod q\}$$ και έχουμε ότι το $q\cap H_E$ είναι πρώτος πάνω από τον πρώτο 
	$p\cap H_E$ στην αδιακλάδιστη επέκταση $H_E/E$, δηλαδή
	$$1 = |I_{q\cap H_E}| = \#\{\sigma \in \Gal(H_E/E): \ \sigma(a) \equiv a \mod q\cap H\}$$ Άρα ξεκινώντας με ένα $\sigma \in I_q$ ο ισομορφισμός μας δίνει:
	$$\sigma(a) \equiv a \mod q \ \forall a \in \mathcal{O}_{H_EK}$$
	$$\sigma|_{H_E}(a) \equiv a \ \forall a \in \mathcal{O}_{H_E}$$
	$$\sigma|_{H_E} = 1 \implies \sigma = 1$$ οπότε $e=1$, δηλαδή το $q$ θα παραμείνει αδιακλάδιστο. Όμοια, για έναν αρχιμήδειο πρώτο 
	$w$ του $H_EK$ πάνω από το $v$ του $K$ θα έχουμε ότι
	$$I_w = \{\sigma \in \Gal(H_EK/K): \ w \circ \sigma = w \}$$
	$$I_{w|_{H_E}} = \{\sigma \in \Gal(H_E/K): \ w|_{H_E} \circ \sigma = w|_{H_E}\}$$
	και $|I_w|=|I_{w|_{H_E}}| = 1$.
	%https://math.stackexchange.com/questions/507671/the-galois-group-of-a-composite-of-galois-extensions
	
	$ $\newline
	Άρα η αδιακλάδιστη επέκταση $H_EK/K$ θα στέκεται μέσα στην μέγιστη αδιακλάδιστη επέκταση $H_K/K$. Συνεπώς,

	 $$h_K = [H_K:K] = [H_K:H_EK][H_EK:E] = [H_K:H_EK]\cdot h_E$$ δηλαδή
	 $$h_E|h_K$$
\end{proof}
%https://math.stackexchange.com/questions/1412636/what-are-the-restrictions-in-the-ramification-behavior-of-a-galois-extension-of

\begin{proof}(του Θεωρήματος)

	$ $\newline
	Η επέκταση $\Q(\zeta_n)/\Q(\zeta_n+\zeta_n^{-1})$ έχει βαθμό $2$ και άρα δεν υπάρχει ενδιάμεση επέκταση. Επιπλέον, αυτή διακλαδίζεται στον πραγματικό πρώτο, 
	αφού το $\Q(\zeta_n)$ είναι \tl{CM}-σώμα και το $\Q(\zeta_n+\zeta_ν^{-1})$ είναι πλήρως πραγματικό. Άρα ικανοποιούνται οι προϋποθέσεις της προηγούμενης 
	πρότασης.
\end{proof}


\begin{defn}
	Ορίζουμε τον σχετικό αριθμό κλάσεων του $\Q(\zeta_n)$ να είναι $h_n^- := h_n/h^+_n$.
\end{defn}

\noindent Με εργαλεία της ανάλυσης και χρησιμοποιώντας τον τύπο του \tl{Dirichlet} για τους αριθμούς κλάσεων δείχνεται ότι
\begin{equation*}
	h_n^- = 2^a 2n \prod\limits_{ \substack{ \chi \in X \\ \chi(-1)=-1}} \left(-\frac{1}{2} B_{1,\chi}\right)
\end{equation*}
όπου το $X$ είναι η ομάδα χαρακτήρων που αντιστοιχεί στο $\Q(\zeta_n)$ και το $a$ είναι $0$ αν το $n$ είναι πρώτος ή $1$ διαφορετικά. Ειδικότερα, για $p$ πρώτο έχουμε
\begin{equation}
	h^-_p = 2p \prod\limits_{\substack{j = 1 \\ j \text{ περιττός}}} \left(-\frac{1}{2} B_{1,\omega^j}\right) \label{3.1}
\end{equation}
και αυτός ο τύπος θα χρησιμοποιηθεί για να αποδειχτεί το ότι $p|h^-_p$ αν και μόνο αν $p|B_j$ για κάποιο $j=2,4,\ldots,p-3$.
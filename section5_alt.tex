


%Kubota Leopoldt original p-adic L-functions (analytic)
Απόδειξη από \tl{Iwasawa} για τους πρώτους που ικανοποιούν την εικασία του \tl{Vandiver}, απόδειξη από \tl{Mazur} και \tl{Wiles} για κάθε πρώτο το 1984 και απόδειξη από \tl{Wiles} το 1990 για κάθε πλήρως πραγματικό σώμα $F$, δηλαδή ξεκινώντας από την ομάδα $\Gal(F(\zeta_p)/F)$ στις $\Z_p$-επεκτάσεις.
%κατασκευή p-adic L function με stickelberger?



Για λόγους απλότητας, υποθέτουμε ότι $p\neq 2$. Για έναν $\chi$ χαρακτήρα \tl{Dirichlet} ο \tl{Iwasawa} κατασκεύασε χρησιμοποιώντας τα \tl{Stickelberger} στοιχεία την $p$-αδική $L$-συνάρτηση $\mathcal{L}_p(s,\chi)$, όπως αυτό μπορεί να ορίζεται καλά. Όπως είδαμε έχουμε την ακόλουθη απεικόνιση με πεπερασμένο συνπυρήνα, δηλαδή τον ψευδοισομορφισμό

$$X \sim \prod\limits_{i} \Lambda / (p^{n_i}) \oplus \prod\limits_{j}\Lambda /(f^{m_j}_j)$$

το οποίο στην ουσία μας λέει ότι το $\prod\limits_{i} p^{n_i} \prod\limits_{j} f^{m_j}_j$ μηδενίζει το $X$. Ας θυμηθούμε το αντίστοιχο ότι το στοιχείο \tl{Stickelberger} για το σώμα $K_n$ μηδενίζει την ομάδα κλάσεων ιδεωδών. Έτσι προκύπτει φυσιολογικά το ερώτημα για την συσχέτιση που έχουν τα \tl{Stickelberger} στοιχεία πάνω από $\Z_p$-πύργους σωμάτων και του μηδενιστή του $X$ που αναφέραμε. Προκύπτει ότι το όριο των \tl{Stickelberger} στοιχείων πάνω στην $\Z_p$-επέκταση θα ανήκει στην \tl{Iwasawa} άλγεβρα $\Lambda$ και η αναλυτική συνάρτηση που αντιστοιχεί σε αυτή τη τυπική δυναμοσειρά είναι η αρχική $p$-αδική $L$-συνάρτηση όπως κατασκεύασαν με εργαλεία ανάλυσης οι \tl{Kubota} και \tl{Leopoldt}. Τυπικά, η κύρια εικασία της θεωρίας \tl{Iwasawa} λέει ότι αυτές οι δύο μέθοδοι δίνουν το ίδιο στοιχείο στην \tl{Iwasawa} άλγεβρα, ως προς συντροφικότητα. Οπότε, θα εκφράσουμε την χαρακτηριστική συνάρτηση του $X$ σαν μια $p$-αδική $L$-συνάρτηση, το οποίο μάλιστα έρχεται σε αναλογία με τις εικασίες του \tl{Weil} για καμπύλες. 

Έστω $q=p$ αν $p\neq 2$ και $q=4$ αν $p=2$. Επιπλέον, έστω $d$ θετικός ακέραιος με $(p,d)=1$, όπου υποθέτουμε $d\not\equiv 2 (\operatorname{mod}4)$. Θέτουμε $q_n = qp^n d$ και $K_n=\Q(\zeta_{q_n})$ με $K_\infty = \bigcup_{n\geq 0}K_n$. Κοιτώντας στα επίπεδα που οι επεκτάσεις είναι πεπερασμένες έχουμε την βραχεία ακριβή ακολουθία
$$0\longrightarrow \Gal(K_n/K_0) \longrightarrow \Gal(K_n/\Q) \longrightarrow \Gal(K_0/\Q) \longrightarrow 0$$ 
η οποία καθώς είνα διασπώμενη έχουμε την ταύτιση
$$\Gal(K_n/\Q) \simeq \Delta \times \Gamma_n$$ με 
$$\Delta \simeq (\Z/p\Z)^\times, \quad \Gamma_n \simeq \Z/p^n \Z$$

και όπως έχουμε δει στο αντίστροφο όριο έχουμε
$$\Gal(K_\infty/\Q) \simeq \Delta \times \Gamma $$ 
με $\Gamma = \Gal(K_\infty/K_0) \simeq \Z_p$.


Έστω $\chi$ χαρακτήρας με \tl{conductor} της μορφής $dp^j$. Βλέποντας τον ως χαρακτήρα της $\Gal(K_n/\Q)$, τότε γράφουμε $\chi = \theta \psi$ όπου $\theta$ είναι χαρακτήρας του $\Delta$ και $\psi$ χαρακτήρας του $\Gamma_n$. Οι $\theta,\psi$ αναφέρονται στην βιβλιογραφία ως χαρακτήρες πρώτου και δεύτερου τύπου αντίστοιχα. Σημειώνουμε ότι ο $\psi$ είναι άρτιος χαρακτήρας με τάξη δύναμη του $p$ και \tl{conductor} $qp^i$ για $i\geq 1$, ενώ ο $\theta$ έχει \tl{conductor} $d$ ή $qd$. Οι χαρακτήρες του πρώτου τύπου αντιστοιχούν στο σώμα $K_0$, ενώ οι χαρακτήρες δεύτερου τύπου αντιστοιχούν στο υπόσωμα του $\Q(\zeta_{qp^n})$ με βαθμό $p^n$ πάνω από το $\Q$. Έχουμε λοιπόν ότι αν ο $\chi$ είναι άρτιος χαρακτήρας θα είναι και ο $\theta$. 

Έστω $\mathcal{O}_{\theta}$ ο δακτύλιος ακεραίων του $\Q_p(\theta)$, δηλαδή επισυνάπτοντας τις τιμές που παίρνει ο $\theta$. Η αλγεβρική κατασκευή των $p$-αδικών $L$-συναρτήσεων γίνεται με το ακόλουθο θεώρημα. Υπενθυμίζουμε ότι για τους περιττούς χαρακτήρες η $\mathcal{L}_p(s,\chi)$ είναι ταυτοτικά 0, οπότε κοιτάμε την περίπτωση των άρτιων.

\begin{theorem} Έστω $\chi$ ένας άρτιος χαρακτήρας. Υπάρχουν τυπικές δυναμοσειρές $f_\theta(T)$ (αν $\theta \neq 1$), $g_{\theta}(T)$ και $h_\theta(T) \in \mathcal{O}_\theta [[T]]$ με 
	$$h_\theta (T) = 1-(1+q_0)/(1+T)$$
	$$f_\theta(T) = \frac{g_\theta(T)}{h_\theta(T)}, \quad \theta \neq 1$$
	Αν $\theta =1$, τότε παίρνουμε το παραπάνω τυπικά σαν ορισμό του $f_\theta(T)$ στο σώμα πηλίκων του $\mathcal{O}_\theta[[T]]$. Επιπλέον, στον δίσκο $D = \{ s \in \mathbb{C}_p: \ |s| < qp^{-1/(p-1)}\}$ ισχύει ότι
	$$\mathcal{L}_p (s,\chi) = f_{\theta}(\psi(1+q_0)^{-1}(1+q_0)^s -1)$$
\end{theorem}
\begin{proof}
	Δες \tl{ reference Wash theorem 7.10}
\end{proof}
%να γράψω τα remark 
Υπενθυμίζουμε, για έναν χαρακτήρα $\theta$ του $\Delta$, έχουμε ως $\varepsilon_{\theta}$ την προβολή στον ομαδοδακτύλιο $\Z_p[\Delta]$
$$\varepsilon_{\theta} = \frac{1}{p-1}\sum\limits_{\delta \in \Delta}\theta(\delta)\delta^{-1}$$ και για την άλγεβρα \tl{Iwasawa}
$$\Lambda  = \Z_p[\Gamma] = \varprojlim \Z_p[\Gamma_n] \simeq \Z_p[[T]]$$
όπως έχουμε δει, υπάρχει απεικόνιση $\Lambda$-προτύπων
$$\varepsilon_{\theta}X \longrightarrow \bigoplus \limits_{i}\Lambda / (p^{n_i(\theta)}) \oplus \bigoplus\limits_j \Lambda /(g_{\theta}^j(T)^{m_j})$$
με πεπερασμένο συνπυρήνα. Ορίζουμε ως χαρακτηριστική συνάρτηση του $X$ εκείνη που μας δίνει τα ιδεώδη
$$\operatorname{char}_{\Lambda}(X(\theta)) = \left( \prod\limits_{i} p^{n_i}\prod\limits_j g^j_{\theta}(T)^{m_j}\right)$$
και έχουμε μια δυναμοσειρά $f_\theta(T) \in \Lambda$ που ικανοποιεί για κάθε $n\geq 1$ και $\chi = \theta\psi$ άρτιο χαρακτήρα του $\Delta \times \Gamma_n$ την σχέση
$$f_{\theta}(\psi(1+q_0)^{-1}(1+q_0)^{1-n} -1 ) = \mathcal{L}_p(1-n,\chi) = -(1-\chi\omega^{-n}(p)p^{n-1})\frac{B_{n,\chi\omega^{-n}}}{n}$$

Έχουμε ό,τι χρειάζεται για να αναφέρουμε την κύρια εικασία \tl{Iwasawa}. Σε αναλογία με τις εικασίες του \tl{Weil}, είναι φυσιολογικό να ρωτήσει κανείς αν η χαρακτηριστική συνάρτηση του \tl{pro} $p$-μέρους της ομάδας κλάσεων ιδεωδών $X$ μιας $\Z_p$-επέκτασης έχει μορφή συνάρτησης ζήτα. Βλέπουμε ότι η χαρακτηριστική συνάρτηση στην άλγεβρα \tl{Iwasawa} είναι καλά ορισμένη ως προς αντιστρέψιμα στοιχεία. Είδαμε επίσης ότι τα \tl{Stickelberger} στοιχεία σχηματίζουν μια κυκλοτομική $\Z_p$-επέκταση, που επάγει μια δυναμοσειρά που περιμβάλλει τις ειδικές τιμές των $L$-σειρών \tl{Dirichlet}. Η κύρια εικασία \tl{Iwasawa} είναι πράγματι αυτό που θα περίμενε κανείς, ότι οι δύο αυτές δυναμοσειρές θα πρέπει να είναι ίσες ως προς αντιστρέψιμα στοιχεία μέσα στην άλγεβρα \tl{Iwasawa}. Υπάρχουν πολλοί ισοδύναμοι τρόποι να διατυπωθεί το προγηγούμενο. Αναφέρουμε τον εξής.

\begin{theorem}[Κύρια Εικασία της Θεωρίας \tl{Iwasawa}]
	Για κάθε μη τετριμμένο άρτιο χαρακτήρα $\theta$ του $\Delta$ ισχύει ότι
	$$\operatorname{char}_{\Lambda}(X(\theta)) = f_{\theta}(T)\Lambda$$
\end{theorem}
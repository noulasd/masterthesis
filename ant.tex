%\noindent Σε αυτό το κεφάλαιο θα υπενθυμίσουμε διάφορα αποτελέσματα από την αλγεβρική θεωρία αριθμών, τα κυκλοτομικά σώματα, την θεωρία \tl{Galois} σε 
%άπειρες επεκτάσεις και την θεωρία κλάσεων σωμάτων. Για τους ακριβείς ορισμούς και για να δει κανείς αυτήν την θεωρία πιο αναλυτικά με αποδείξεις 
%μπορεί να ανατρέξει στα βιβλία του \tl{Milne} \tl{reference}.

\noindent Σε αυτό το κεφάλαιο θα υπενθυμίσουμε διάφορα αποτελέσματα που χρειαζόμαστε, ξεκινώντας από την αλγεβρική θεωρία αριθμών. Στην συνέχεια, θα αναφέρουμε τα 
βασικά στοιχεία των κυκλοτομικών σωμάτων, καθώς σε αυτά επικεντρώνεται κυρίως η κλασική θεωρία \tl{Iwasawa}. Επιπλέον, θα είναι αναγκαίο να εργαζόμαστε 
με επεκτάσεις σωμάτων άπειρου βαθμού και τις αντίστοιχες ομάδες \tl{Galois}, οπότε διατυπώνουμε το θεμελιώδης θεώρημα για τις άπειρες επεκτάσεις που 
γενικεύει με την χρήση της τοπολογίας την πεπερασμένη περίπτωση. Θα κλείσουμε το κεφάλαιο με κάποια βασικά στοιχεία της θεωρίας κλάσεων σωμάτων, η οποία έχει σκοπό να περιγράψει τις αβελιανές επεκτάσεις των σωμάτων αριθμών. 
Για να δει κανείς τα αποτελέσματα από όσα αναφέραμε παραπάνω πιο λεπτομερώς και με αποδείξεις μπορεί να ανατρέξει στα βιβλία του \tl{Milne} \cite{Milne1}, \cite{Milne2} ή στα \cite{kontogar}, \cite{CF}.


\section{Άλγεβρική Θεωρία Αριθμών}

\noindent Έστω $L/K$ να είναι μια πεπερασμένη επέκταση σωμάτων αριθμών με δακτύλιους ακεραίων $\mathcal{O}_L$ και $\mathcal{O}_K$ αντίστοιχα.

\begin{theorem}
	Κάθε γνήσιο μη-μηδενικό ιδεώδες $\mathfrak{a} \subset \mathcal{O}_K$ έχει μοναδική παραγοντοποίηση:
	$$\mathfrak{a} = \mathfrak{p}_1^{e_1} \cdots \mathfrak{p}_r^{e_r}$$ με $e_i > 0$ και τα $\mathfrak{p}_i$ να είναι πρώτα ιδεώδη.
\end{theorem}

\noindent Έχοντας ένα πρώτο ιδεώδες $\mathfrak{p} \subset \mathcal{O}_K$, μπορούμε να θεωρήσουμε το ιδεώδες $\mathfrak{p} \mathcal{O}_L$ στον 
δακτύλιο $\mathcal{O}_L$. Οπότε, με βάση το προηγούμενο θεώρημα μπορούμε να το παραγοντοποιήσουμε σε γινόμενο πρώτων ιδεωδών:
\begin{equation}
	\label{eq2.1}
	\mathfrak{p}\mathcal{O}_L = \mathfrak{p}_1^{e_1} \cdots \mathfrak{p}_r^{e_r}
\end{equation}  με τα $\mathfrak{p}_i$ να είναι πρώτα ιδεώδη του $\mathcal{O}_L$.


\begin{defn}
	Σε μια παραγοντοποίηση όπως στην εξίσωση (\ref{eq2.1}), λεμε το $e_i = e(\mathfrak{p}_i \mid \mathfrak{p})$ δείκτη διακλάδωσης του $\mathfrak{p}$ στο 
	$\mathfrak{p}_i$. Θα λέμε ότι το πρώτο ιδεώδες $\mathfrak{p}$ διακλαδίζεται στο $L$ αν ισχύει ότι $e_i >1$ για κάποιο $i$. Επιπλέον, λέμε βαθμό αδράνειας 
	$f_i = f(\mathfrak{p}_i \mid \mathfrak{p})$ την διάσταση του διανυσματικού χώρου $\mathcal{O}_L/\mathfrak{p}_i$ πάνω από το πεπερασμένο σώμα 
	$\mathcal{O}_K/\mathfrak{p}$. 
\end{defn}%residue class field degree f_i

%από εδώ και πέρα μη μηδενικό πρώτο ιδεώδες... ή κατευθείαν σκέτος πρώτος
\begin{prop}
	Ένα πρώτο ιδεώδες $\mathfrak{p}$ στο $\mathcal{O}_K$ διακλαδίζεται στο $\mathcal{O}_L$ αν και μόνο αν $\mathfrak{p} \mid \operatorname{disc} 
	(\mathcal{O}_L/\mathcal{O}_K)$.
\end{prop}
%!Τι σημαίνει $disc(\mathcal{O}_L/\mathcal{O}_K)$; η διακρίνουσα ορίζεται για σώματα αριθμών. Λογικά:
%$$disc_{\mathcal{O}_K} (\mathcal{O}_L) = \det (T_{L/K}(a_ia_j))$$ όπου $a_i$ βάση του $\mathcal{O}_L$ ως $\mathcal{O}_K$-πρότυπο, που σημαίνει τα $a_i$ είναι βάση του $L$ υπεράνω του $K$ (σωστό με βάση \tl{Milne})
%απόδειξη;
%theorem 24 number fields σελίδα 50 (περίπου)
%solid Milne: the primes that ramify σελίδα 53
%τα ορίζει σελίδα 26 ο Milne


\begin{theorem}
	Με βάση τα παραπάνω έχουμε:
	\begin{equation}
		\sum\limits_{i=1}^r e(\mathfrak{p}_i\mid \mathfrak{p}) f(\mathfrak{p}_i \mid \mathfrak{p}) = \sum\limits_{i=1}^r e_i f_i = [L:K].
	\end{equation}
\end{theorem}

\noindent Στο εξής θα θεωρούμε ότι η επέκταση $L/K$ είναι \tl{Galois}. Έτσι, μπορούμε να απλοποιήσουμε αρκετά το προηγούμενο θεώρημα. Ξεκινάμε με την ακόλουθη πρόταση.

\begin{prop} \label{prop2.5}
	Η ομάδα $\Gal(L/K)$ δρα μεταβατικά στο σύνολο των πρώτων ιδεωδών $\mathfrak{p_i}$ του $\mathcal{O}_L$ που στέκονται πάνω από το $\mathfrak{p}$.
\end{prop}

%\begin{proof} Προς άτοπο, έστω ότι $\sigma (\p_i) \neq \p_j$ για κάθε $\sigma \in \Gal (L/K)$. Υπενθυμίζουμε ότι το $\sigma (\p_i)$ θα είναι και αυτό πρώτο ιδεώδες που θα στέκεται πάνω από το $\p$. Καθώς είμαστε σε περιοχές \tl{Dedekind} τα $\p_i$ και $\sigma(p_i)$ θα είναι μεγιστικά. Άρα $\p_i \not\subseteq \sigma(\p_i)$. Από το αντιθετοαντίστροφο του λήμματος αποφυγής πρώτων παίρνουμε ότι 
	%$$\p_i \not\subseteq \bigcup\limits_{\sigma \in \Gal (L/K)} \sigma(\p_i)$$ δηλαδή, υπάρχει $x \in \p_i$ που αποφεύγει όλα τα $\sigma (\p_i)$. Για την νόρμα, παρατηρούμε ότι:
	%$$N_{L/K}(x) = \prod\limits_{\sigma \in \Gal (L/K)} \sigma(x)$$ βρίσκεται μέσα στο $\p = \mathcal{O}_K \cap \p_i$, διότι η νόρμα θα βρίσκεται μέσα στο $\mathcal{O}_K$ καθώς και στο παραπάνω γινόμενο εμφανίζεται το $x$ που ανήκει στο ιδεώδες $\p_i$. Έχουμε ότι $x \not \in \sigma(p_i)$ και άρα $\sigma^{-1}(x) \not \in \p_i$ για κάθε $\sigma \in \Gal (L/K)$. Άρα $\prod \sigma^{-1}(x) = \prod \sigma(x) \not \in \p_i\cap \mathcal{O}_K = \p$, το οποίο είναι άτοπο.
%\end{proof}
%Υπενθυμίζουμε: robert ash proposition 8.1.1 
%prime avoidance robert ash section 3.1 problems

\begin{cor}
	Έστω $L/K$ \tl{Galois} επέκταση και $0 \neq \p \subset \mathcal{O}_K$ ένα πρώτο ιδεώδες. Τότε, $e(\p_i \mid \p) = e(\p_j \mid \p) = e$ και $f(\p_i \mid \p) = f(\p_j \mid \p)=f$ για κάθε $i,j$ της εξίσωσης (\ref{eq2.1}). Ειδικότερα, έχουμε ότι $[L:K] = ref$.
\end{cor}
%\begin{proof}
	%Ο αυτομορφισμός $\sigma$ διατηρεί τις αλγεβρικές σχέσεις:
	%$$\sigma (\p \mathcal{O}_L) = \prod\limits_{i=1}^r \sigma(p_i)^{e_i} = \prod\limits_{i=1}^r \p_i^{e_i} = \p \mathcal{O}_L$$ και συγκρίνουμε τους εκθέτες για να πάρουμε ότι είναι ίδιοι. Αν $\sigma(\p_i) = \p_j$ τότε παίρνουμε $f_i = f_j$ από τον ισομορφισμό πεπερασμένων σωμάτων:
	%$$\mathcal{O}_L / \p_i \simeq \mathcal{O}_L / \p_j$$ από τον επιμορφισμό που επάγει ο $\sigma$:
	%$$\mathcal{O}_L \longrightarrow \mathcal{O}_L / \p_j $$
	%$$x \longmapsto \sigma(x) + \p_j$$
%\end{proof}


\noindent Για $[L:K]=n$, υπενθυμίζουμε την σχετική ορολογία:
\begin{center}
\begin{tabular}{ |c|c|c|c|c| } 
	\hline
	 & $e$ & $f$ & $r$ \\
	\hline
	 αδρανές $\quad \quad \quad \quad \quad \quad$ & 1 & $n$ & 1\\ 
	 πλήρως διακλαδιζόμενο & $n$ & 1 & 1 \\ 
	 πλήρως διασπώμενο $\quad $& 1 & 1 & $n$ \\ 
	 \hline
	 \end{tabular}
	\end{center}

\begin{defn}
	Έστω $\q$ ένα πρώτο ιδεώδες του $\mathcal{O}_L$. Η υποομάδα $$D_{\q} = \{\sigma \in \Gal (L/K): \sigma(\q) = \q\}$$ λέγεται η ομάδα διάσπασης του $\q$ υπεράνω του $K$.
\end{defn}

\noindent Από την πρόταση \ref{prop2.5} και το θεώρημα \tl{orbit-stabilizer} παίρνουμε το ακόλουθο πόρισμα.

\begin{cor} \label{cor2.8}
	Για $L/K$ επέκταση όπως παραπάνω και $\p$ πρώτο ιδεώδες του $\mathcal{O}_K$ έχουμε:
	\begin{enumerate}
		\item $[\Gal(L/K) : D_{\q}] = r$ για κάθε $\q \mid \p$.
		\item $D_{\q} = 1$ αν και μόνο αν το $\p \mathcal{O}_L$ διασπάται πλήρως.
		\item $D_{\q} = \Gal(L/K)$ αν και μόνο αν το $\p \mathcal{O}_L$ διακλαδίζεται πλήρως, δηλαδή $\p \mathcal{O}_L = \q^{n}$ με $n=[L:K]$.
		\item $|D_{\q}| = ef$. 
	\end{enumerate}
\end{cor}

$ $\newline
Έχουμε μια φυσιολογική απεικόνιση:
$$D_{\q} \longrightarrow \Gal \left( (\mathcal{O}_L/\q) / (\mathcal{O}_K/\p)\right),$$ όπου ένα $\sigma \in D_{\q}$, εφόσον κρατάει 
σταθερό το $\q$, επάγει έναν $\mathcal{O}_L/\q$-αυτομορφισμό $\overline{\sigma}$. Ο αυτομορφισμός $\overline{\sigma}$ με την σειρά του 
κρατάει σταθερό το υπόσωμα $\mathcal{O}_K/\p$, διότι ο $\sigma$ κρατάει σταθερό το $K$. Όπως αποδεικνύεται στην πρόταση 14 του βιβλίου 
του \tl{Lang} \cite{Lang1}, αυτή η απεικόνιση είναι επί. %8.1.8 robert b ash

\begin{defn}
	Ο πυρήνας $I_{\q} \subseteq D_{\q}$ του παραπάνω ομομορφισμού λέγεται ομάδα αδράνειας του $\q$ υπεράνω του $K$. Ισχύει ότι:
	$$I_{\q} = \{s \in D_{\q}: \ \sigma(x) = x \mod \q \ \forall x \in L\}.$$
\end{defn}

\noindent Από το πόρισμα 2.8 έχουμε ότι:
\begin{cor}
	Για $L/K$ επέκταση όπως παραπάνω έχουμε ότι $|I_{\q}| = e$.
\end{cor}


%θεωρία πεπερασμένων σωμάτων Μαλιάκας Galois κεφ 9.
\noindent Από την θεωρία πεπερασμένων σωμάτων γνωρίζουμε ότι η ομάδα \tl{Galois} ενός πεπερασμένου σώματος είναι κυκλική και ένας 
γεννήτορας της είναι ο $\sigma(x) = x^q$, όπου $q$ είναι η τάξη του υποσώματος. Αυτός ο γεννήτορας είναι γνωστός ως ο αυτομορφισμός του 
\tl{Frobenius}. Στην περίπτωσή μας με $q = |\mathcal{O}_K/\p|$ και $\q \mid \p$, υπάρχει ένας αυτομορφισμός $\overline{\sigma}_{\q}$ 
του $\mathcal{O}_L/\q$ που σταθεροποιεί το $\mathcal{O}_K/\p$ και δίνεται από την σχέση $\overline{\sigma}_{\q}(x + \q) = x^q + \q$. 
Άρα από τον ισομορφισμό:
$$D_{\q}/ I_{\q} \simeq \Gal \left( (\mathcal{O}_L/\q) / (\mathcal{O}_K/\p) \right)$$

\noindent Έχουμε ότι κάποιο σύμπλοκο $\sigma_{\q} + I_{\q}$ θα αντιστοιχεί στον αυτομορφισμό του \tl{Frobenius}. Θα λέμε κάθε στοιχείο του 
συμπλόκου ως αυτομορφισμό του \tl{Frobenius} στο $\q$ και θα το συμβολίζουμε με $\Frob_{\q}$. Αν η ομάδα αδράνειας $I_{\q}$ είναι τετριμμένη, 
δηλαδή $e=1$ και το $\p$ δεν διακλαδίζεται, τότε υπάρχει καλά ορισμένο στοιχείο $\Frob_{\q} \in D_{\q}$. Είναι σημαντικό να μπορούμε να 
συσχετίσουμε τα $\Frob_{\q_1}$ και το $\Frob_{\q_2}$ για τα διαφορετικά πρώτα ιδεώδη $\q_i \mid \p$. Ξέρουμε ότι υπάρχει $\tau \in \Gal(L/K)$ με 
$\tau(\q_1) = \q_2$ και εύκολα φαίνεται ότι $D_{\q_2} = \tau D_{\q_1} \tau^{-1}$, καθώς και ότι $\Frob_{\q_2} = \tau \Frob_{\q_1} \tau^{-1}$. 
Αν η $\Gal(L/K)$ είναι αβελιανή και το $\p$ δεν διακλαδίζεται στο $L$, τότε μπορούμε να ξεχωρίσουμε το μοναδικό στοιχείο της 
$\Gal(L/K)$ που βρίσκεται στην $D_{\q}$ για κάθε $\q \mid \p$. Αυτό το στοιχείο θα το λέμε $\Frob_{\p}$.

Επιπλέον, είναι σημαντικό το ακόλουθο αποτέλεσμα για τις ομάδες διάσπασης.


%εύκολα: κοιτάω σ \in D_{\q_1} ότι τστ^{-1}(q_2) = ... = q_2 
%Frob: σ(χ) = χ^τάξη O_K/p mod q
% στ^{-1}(χ) = τ^{-1}(χ)^τάξη = (ομομορφισμός) τ^{-1}(χ^τάξη) και παίρνω τ στα δύο μέλη

\begin{prop}
	Έστω $L/K$ επέκταση \tl{Galois}, $\p$ πρώτο ιδεώδες του $\mathcal{O}_K$ και $\q$ πρώτο ιδεώδες του $\mathcal{O}_L$ με $\q \mid \p$. 
	Υπάρχει ισομορφισμός $D_{\q} \cong \Gal(L_\q/K_\p)$ όπου με $L_\q$ και $K_\p$ συμβολίζουμε τις πληρώσεις των σωμάτων ως προς τις 
	αντίστοιχες νόρμες που επάγουν τα πρώτα ιδεώδη.
\end{prop}





Σε αυτό το κεφάλαιο θα δώσουμε μια πληθώρα αποτελεσμάτων για τα μεγέθη των τάξεων που παίρνουν οι ομάδες κλάσεων ιδεωδών των κυκλοτομικών σωμάτων. 
Καθώς πολλά αποτελέσματα είναι αναλυτικής φύσεως, δεν θα δώσουμε τις αποδείξεις τους, εφόσον μας ενδιαφέρουν περισσότερο οι εφαρμογές τους. Ωστόσο, 
όλα αποδεικνύονται λεπτομερώς στο βιβλίο του \tl{Washington} \cite{Wash} στα κεφάλαια 3-6.

Ένας δεύτερος στόχος που έχει αυτό το κεφάλαιο, είναι να δώσουμε μια εισαγωγή στα εργαλεία που χρειαζόμαστε για την κύρια εικασία στο κεφάλαιο 5. Αυτά 
είναι οι χαρακτήρες, οι $L$-συναρτήσεις και οι $p$-αδικές $L$-συναρτήσεις όπου οι τελευταίες όπως θα δούμε έχουν κυρίαρχο ρόλο.

\section{Χαρακτήρες}

\begin{defn}
	Ένας χαρακτήρας \tl{Dirichlet} \tl{modulo} $m$ είναι ένας ομομορφισμός μεταξύ των πολλαπλασιαστικών ομάδων $\chi:(\Z/m\Z)^{\times} \longrightarrow \mathbb{C}^{\times}$, όπου $m$ είναι θετικός ακέραιος.
\end{defn}

\noindent Για έναν χαρακτήρα \tl{Dirichlet} και $m$ όπως παραπάνω, μπορούμε να πάρουμε νέους χαρακτήρες $(\Z/n\Z)^\times \longrightarrow 
\mathbb{C}^\times$ για κάθε ακέραιο $n$ με $m|n$ συνθέτοντας με την φυσιολογική απεικόνιση $(\Z/n\Z)^\times \longrightarrow 
(\Z/m\Z)^\times \longrightarrow \mathbb{C}^\times$. Το ελάχιστο $m$ έτσι ώστε το $\chi$ να είναι ομομορφισμός $(\Z/m\Z)^\times \longrightarrow 
\mathbb{C}$ λέγεται \tl{conductor} του $\chi$ και συμβολίζεται με $m_{\chi}$. 

Θα λέμε έναν χαρακτήρα $\chi$ πρωταρχικό αν τον ορίζουμε να έχει υπόλοιπο τον \tl{conductor} του, δηλαδή
$$\chi : (\Z/m_{\chi}\Z)^{\times} \longrightarrow \mathbb{C}^\times.$$ Με την παραπάνω παρατήρηση, ξέρουμε ότι ο $\chi$ θα επάγει χαρακτήρες 
που θα είναι με υπόλοιπο τα πολλαπλάσια του $m_\chi$. Και αντίστροφα, 
για κάθε χαρακτήρα $\chi$ υπάρχει ο αντίστοιχος πρωταχικός χαρακτήρας από τον οποίο επάγεται. Επιπλέον, θα ισχύει ότι $\chi(-1) =\pm 1$. Έτσι οι χαρακτήρες χωρίζονται 
στους {\em άρτιους} που ικανοποιούν την σχέση $\chi(-1)=1$ και αντίστοιχα στους {\em περιττούς} με $\chi(-1)=-1$.

Έστω $\chi,\psi$ δύο χαρακτήρες με \tl{conductors} $m_{\chi}$ και $m_{\psi}$ αντίστοιχα. Για να οριστεί το γινόμενό τους ορίζεται πρώτα 
ο ομομορφισμός $$\phi: (\Z/ \operatorname{εκπ}(m_{\chi},m_{\psi}) \Z)^{\times} \longrightarrow \mathbb{C}^{\times}$$ ως $$\phi(n)=\chi(n)\psi(n).$$ 
Καθώς ο χαρακτήρας $\phi$ δεν θα είναι απαραίτητα πρωταχρικός, ορίζουμε ως γινόμενο $\chi \psi$ τον πρωταρχικό χαρακτήρα που αντιστοιχεί στον $\phi$. 
Επιπλέον, ορίζουμε ως κύριο χαρακτήρα $\chi_0$ αυτόν που απεικονίζει κάθε κλάση στο $1$. Μαζί με αυτό, για κάθε $\chi$ ορίζουμε τον αντίστροφο του 
$\overline{\chi}$, όπου $\overline{\chi}(a) = \chi(a)^{-1} = \overline{\chi(a)}$ με το τελευταίο να είναι ο μιγαδικός συζυγής. Με όλα τα προηγούμενα, 
οι χαρακτήρες \tl{Dirichlet} αποτελούν πολλαπλασιαστική ομάδα.
%gcd(m_x,m_y) =1 άσκηση m_{xy}|m_x m_y εξ ορισμού, για το ανάποδο CRT στο Z/m_xm_yZ

Χρησιμοποιώντας τον ισομορφισμό $$\Gal(\Q(\zeta_n)/\Q) \cong (\Z/n\Z)^{\times},$$ μπορούμε να θεωρήσουμε τους χαρακτήρες \tl{Dirichlet} ως χαρακτήρες ομάδων \tl{Galois}. Ο ισομορφισμός αυτός είναι ο $\sigma \longmapsto a_{\sigma}$, όπου $a_{\sigma}$ είναι τέτοιο ώστε $\sigma(\zeta_n) = 
\zeta_n^{a_{\sigma}}$. Έστω $\chi$ ένας χαρακτήρας με \tl{conductor} $n$ και πεδίο ορισμού το $\Gal(\Q(\zeta_n)/\Q)$. Ο πυρήνας του $\chi$ είναι υποομάδα της $\Gal(\Q(\zeta_n)/\Q)$ και άρα το σταθερό σώμα του $\ker \chi$ είναι υπόσωμα του $\Q(\zeta_n)$. Θα αναφερόμαστε σε αυτό ως το σταθερό σώμα του $\chi$ και το συμβολίζουμε με $\Q^{\chi}$.

%παράδειγμα;
Γενικότερα, έχοντας ως $X$ μια πεπερασμένη ομάδα χαρακτήρων \tl{Dirichlet} και $n$ το ελάχιστο κοινό πολλαπλάσιο των \tl{conductor} των χαρακτήρων στο $X$, 
τότε το $X$ θα είναι μια υποομάδα της ομάδας των χαρακτήρων του $\Q(\zeta_n)$. Θέτουμε ως $\mathcal{K}$ την τομή των πυρήνων όλων των χαρακτήρων που 
βρίσκονται στο $X$ και $\Q^X$ το σταθερό σώμα του $\mathcal{K}$. Το $\Q^X$ θα λέγεται το σώμα που αντιστοιχεί στην ομάδα χαρακτήρων $X$. 

Ακολουθούν κάποια βασικά αποτελέσματα για τις ομάδες χαρακτήρων, τα οποία μπορούμε να τα δούμε γενικότερα στο πλαίσιο των πεπερασμένων αβελιανών ομάδων. Έστω $G$ μια πεπερασμένη αβελιανή ομάδα, θα συμβολίζουμε με $G^{\wedge}$ την ομάδα των πολλαπλασιαστικών χαρακτήρων της $G$. Σαν $G$ θα εννούμε μια τυχαία πεπερασμένη αβελιανή ομάδα μέχρις ότου να αναφερθεί διαφορετικά.

\begin{lemma} $G\cong G^{\wedge}$ (όχι με φυσιολογικό τρόπο).
\end{lemma}
\begin{proof} Η $G$ γράφεται ως ευθύ άθροισμα κυκλικών ομάδων της μορφής $\Z/m\Z$. Συνεπώς, η $G^{\wedge}$ γράφεται ως ευθύ γινόμενο ομάδων της μορφής 
	$(\Z/m\Z)^{\wedge}$. Για ένα $\chi \in (\Z/m\Z)^{\wedge}$, εφόσον η δομή εδώ είναι προσθετική και το 1 είναι γεννήτορας, το $\chi(1)$ θα είναι οποιαδήποτε $m$-οστή ρίζα της μονάδας. Συνεπώς, ισχύει το λήμμα για την $\Z/m\Z$ και άρα για την $G$. 
\end{proof}

\begin{theorem} $G\cong G^{\wedge \wedge}$ (με φυσιολογικό τρόπο). \label{chars_natural_theorem}
\end{theorem}


\begin{proof} Ορίζουμε:
	$$G \longrightarrow G^{\wedge\wedge}$$
	$$g \longmapsto (\chi \mapsto \chi(g))$$ και έχοντας από το προηγούμενο λήμμα ότι $|G| = |G^{\wedge}| = |G^{\wedge \wedge}|$, 
	αρκεί να δείξουμε ότι η απεικόνιση είναι μονομορφισμός. Έστω $g \in G$ τέτοιο ώστε $\chi(g) = 1$ για κάθε $\chi \in G^{\wedge}$. 
	Θέτουμε ως $H=\langle g \rangle$ και τότε οι χαρακτήρες της $G$ αντιστοιχούν στους χαρακτήρες της $G/H$, αφού έχουν τετριμμένη δράση στο $H$. Αυτοί 
	είναι $|G/H|=|G|$ το πλήθος, δηλαδή $g=1$.
\end{proof}

\noindent Είναι βοηθητικό να ταυτίζουμε τις $G,G^{\wedge\wedge}$ και έτσι έχουμε την απεικόνιση:
$$G\times G^{\wedge} \longrightarrow \mathbb{C}^\times$$
$$(g,\chi) \longmapsto \chi(g),$$ η οποία είναι \tl{non-degenerate}, δηλαδή αν $\chi(g) = 1$ για κάθε $\chi \in G^{\wedge}$ τότε $g=1$ και αντίστοιχα, 
αν $\chi(g)=1$ για κάθε $g \in G$, τότε φυσικά $\chi = 1$. Έστω $H$ υποομάδα της $G$. Θέτουμε:
$$H^\perp = \{ \chi \in G^{\wedge}: \ \chi(h) =1 \ \forall \ h \in H\}$$



\begin{lemma} Για $G$ πεπερασμένη αβελιανή ομάδα και $H$ υποομάδα της ισχύουν τα εξής:
	\begin{enumerate}
	\item $H^{\perp} \cong (G/H)^{\wedge}$.
	\item $H^{\wedge} \cong G^{\wedge}/H^{\perp}$.
	\item $H = (H^{\perp})^{\perp}$ (Ταυτίζοντας $G^{\wedge\wedge} = G$).
\end{enumerate}
\end{lemma}

\begin{proof}
	Για το $(1)$ έχουμε τον φυσιολογικό ισομορφισμό όπου ένα $\chi \in G^{\wedge}$ που δρα τετριμμένα σε όλο το $H$ θα αντιστοιχεί σε έναν χαρακτήρα 
	$\bar{\chi} \in (G/H)^{\wedge}$ με $\bar{\chi}(gH) = \chi(g)$. Για το $(2)$ έχουμε τον περιορισμό $G^{\wedge}\rightarrow H^{\wedge}$ με 
	$\chi\mapsto \chi|_H$ και πυρήνα $H^{\perp}$. Για το επί ισχύει ότι $|H^{\perp}| = |(G/H)^{\wedge}| = |G/H| = |G|/|H|$ και άρα $|H^{\wedge}| = |H| = |G|/|H^{\perp}| = |G^{\wedge}|/|H^{\perp}|$. Για το $(3)$ με όμοιο υπολογισμό δείχνουμε ότι οι τάξεις είναι ίσες και έχουμε ότι:
	$$(H^{\perp})^{\perp} = \{g \in G^{\wedge\wedge}: \  g(\chi) = 1 \ \forall \ \chi \in H^{\perp}\} = \{g \in G: \ \chi(g)=1 \ \forall \ \chi \in H^{\perp} \}$$
	από όπου φαίνεται ξεκάθαρα ότι $H\subseteq (H^{\perp})^{\perp}$.
\end{proof}


\noindent Θα θέλαμε να ενισχύσουμε την αντιστοιχία \tl{Galois} για πεπερασμένες επεκτάσεις, ώστε να αντιστοιχούν τα υποσώματα σε υποομάδες χαρακτήρων. 
Αρχικά, δείχνουμε ότι το $\Q(\zeta_n)$ αντιστοιχεί σε συγκεκριμένη ομάδα χαρακτήρων \tl{Dirichlet}.

\begin{prop}
	Το $\Q(\zeta_n)$ είναι το σώμα που αντιστοιχεί στην ομάδα χαρακτήρων $X= \{\chi: \Gal(\Q(\zeta_n)/\Q) \longrightarrow \mathbb{C}^\times\}$.
\end{prop}

\begin{proof}
	Είναι ξεκάθαρο ότι το σώμα που αντιστοιχεί στο $X$ θα είναι υπόσωμα του $\Q(\zeta_n)$, από τον ορισμό του. Αρκεί να δείξουμε ότι το $\mathcal{K}$, 
	δηλαδή η τομή όλων των πυρήνων των χαρακτήρων του $X$ είναι τετριμμένη. Έστω $g \in \mathcal{K}$, δηλαδή $\chi(g) =1 $ για κάθε $\chi \in X$. 
	Για $G=\Gal(\Q(\zeta_n))$ και $X= G^{\wedge}$, μιμούμενοι την απόδειξη του θεωρήματος \ref{chars_natural_theorem} παίρνουμε ότι $g=1$.
\end{proof}

\noindent Στην συνέχεια το $X$ θα είναι η ομάδα χαρακτήρων που αντιστοιχεί στο $\Q(\zeta_n)$. Έχουμε δηλαδή την \tl{nondegenerate} απεικόνιση:
$$\Gal(\Q(\zeta_n)/\Q) \times X \longrightarrow \mathbb{C}^\times$$

\noindent Έστω $K/\Q$ μια πεπερασμένη αβελιανή επέκταση. Από το θεώρημα \tl{Kronecker-Weber} έχουμε ότι υπάρχει $n$ τέτοιο ώστε $K\subset \Q(\zeta_n)$. 
Θέτουμε:
$$Y = \{\chi \in X: \ \chi(\sigma)= 1 \ \forall \ \sigma \in \Gal(\Q(\zeta_n)/K)\}$$ και παρατηρούμε ότι
\begin{align*}
	Y &= \Gal(\Q(\zeta_n)/K)^{\perp} \\
	&\cong (\Gal(\Q(\zeta_n)/\Q)/\Gal(\Q(\zeta_n)/K))^{\wedge} \\
	&=\Gal(K/\Q)^{\wedge}
\end{align*}

\noindent Άρα ξεκινώντας με ένα σώμα $K$ έχουμε συσχετίσει μια ομάδα χαρακτήρων \tl{Dirichlet} $Y$. Έστω τώρα ότι έχουμε $Y$ μια υποομάδα του $X$. 
Θέτουμε ως $K$ να είναι το σταθερό σώμα του
$$Y^{\perp} = \{\sigma \in \Gal(\Q(\zeta_n)/\Q): \ \chi(\sigma) =1 \ \forall \ \chi \in Y\}$$
Όπου έχουμε κάνει την ταύτιση $G=G^{\wedge\wedge}$. Αυτό είναι στην ουσία το $\mathcal{K}$ που αναφέραμε πριν. Εφόσον έχουμε θέσει 
$K:=\Q(\zeta_n)^{Y^{\perp}}$ η θεωρία \tl{Galois} μας λέει ότι $Y^{\perp} = \Gal(\Q(\zeta_n)/K)$. Έτσι έχουμε ότι

\begin{align*}
	Y&=(Y^{\perp})^{\perp} \\
	&=\Gal(\Q(\zeta_n)/\Q)^{\perp} \\
	&\cong \Gal(K/\Q)^{\wedge}
\end{align*}

\noindent Άρα έχουμε κατασκευάσει μια 1-1 και επί αντιστοιχία:


\begin{align*}\{\text{Υποομάδες του } X \} &\longleftrightarrow \{\text{ Υποσώματα του } \Q(\zeta_n) \} \\
	\Gal(K/\Q)^{\wedge} &\longleftrightarrow K \\
Y &\longleftrightarrow \Q(\zeta_n)^{Y^{\perp}}
\end{align*}

\noindent Δηλαδή, σχηματικά έχουμε ενισχύσει την αντιστοιχία \tl{Galois} ως εξής:

\begin{figure}[H]
	\centering
	\begin{tikzcd}
		\Q(\zeta_n)           & 1                                        & X=G^{\wedge}                                      \\
		K \arrow[u, no head]  & H=\Gal(\Q(\zeta_n)/K) \arrow[u, no head] & H^{\perp} \cong (G/H)^{\wedge} \arrow[u, no head] \\
		\Q \arrow[u, no head] & G \arrow[u, no head]                     & \{\chi_0\} = G^{\perp}  \arrow[u, no head]       
		\end{tikzcd}
\end{figure}
%(δεν έχω ορίσει $\chi_0$ \tl{principal} χαρακτήρα και αντίστροφο για να είναι ομάδα)

\noindent Αυτή η αντιστοιχία θα εννοείται όταν αναφερόμαστε στο σώμα που ανήκει σε μια ομάδα χαρακτήρων. Θα εισάγουμε τώρα τον χαρακτήρα 
\tl{Teichmüller}. Υποθέτουμε ότι $p$ είναι ένας περιττός πρώτος, ωστόσο μπορούν τα ίδια αποτελέσματα να γίνουν στην περίπτωση που $p=2$, 
αλλά θα χρειαζόταν να επιβαρύνουμε επιπλέον τον συμβολισμό. Χρειαζόμαστε το γνωστό λήμμα του \tl{Hensel} για τους $p$-αδικούς ακέραιους και ένα πόρισμα του. Την απόδειξη του λήμματος μπορεί να την βρει κανείς στο \cite{Gouv}.

\begin{lemma}[\tl{Hensel's lemma}]
	Έστω $f(x) \in \Z_p[x]$ και $a \in \Z_p$ που ικανοποιεί:
	$$f(a)\equiv 0 \mod p, \ f^{\prime}(a)\not\equiv 0 \mod p$$
	τότε υπάρχει μοναδικό $b\in \Z_p$ τέτοιο ώστε $f(b)=0$ στο $\Z_p$ και $a\equiv b \mod p$.
\end{lemma}

\begin{cor}
	Υπάρχουν ακριβώς $p-1$ διακεκριμένες $(p-1)$-οστές ρίζες της μονάδας στο $\Z_p$. Επιπλέον αυτές παραμένουν διακεκριμένες \tl{modulo} $p$.
\end{cor}

\begin{proof}
	Έστω $f(x) = x^{p-1} -1 \in \Z_p[x] \subseteq \Q_p[x]$. Το $f(x)$ μπορεί να έχει το πολύ $p-1$ ρίζες στο $\Q_p$. Έστω $a \in \Z_p$, $a\neq 0$. 
	Έχουμε ότι $f(a) = 0 (\operatorname{mod} p)$ και $f^{\prime}(a) \neq 0(\operatorname{mod}p)$. Από το λήμμα του \tl{Hensel} υπάρχει μοναδική ρίζα του $f(x)$ στο $\Z_p$ που είναι ισοδύναμη στο $a$ \tl{modulo} $p$. 
\end{proof}


\noindent Συνεπώς, για κάθε μη μηδενικό $a\in\Z_p$ αντιστοιχεί μια καλά ορισμένη ρίζα της μονάδας στο $\Z_p$. Έτσι, ορίζεται ο χακακτήρας \tl{Teichmüller} 
να είναι το $\omega:\Z_p^\times \longrightarrow \Z_p^\times$ όπου ως $\omega(a)$ θα είναι η ρίζα της μονάδας που αντιστοιχεί στο $a$. Στην παραπάνω 
κατασκευή κάθε $a\in \Z_p$ απεικονίζεται στο $\bar{a} \in \mathbb{F}_p$ και με το λήμμα του \tl{Hensel} αυτό γίνεται \tl{lift} στο $\Z_p$. Δηλαδή, εφόσον πετυχαίνουμε τις ρίζες τις μονάδας στο πεδίο τιμών, μπορούμε να βλέπουμε το $\omega$ σαν $p$-αδικό χαρακτήρα \tl{Dirichlet} με \tl{conductor} $p$:
$$\omega: \mathbb{F}_p \longrightarrow \Z_p^\times .$$
Θα μπορούσαμε να το βλέπουμε και σαν μιγαδικό χαρακτήρα \tl{Dirichlet}, αλλά αυτή η γραφή θα είναι πιο χρήσιμη στη συνέχεια.

\begin{prop} Έστω $Y = \{1,\omega^2,\omega^4,\ldots,\omega^{p-3}\}$. Το σώμα που αντιστοιχεί στο $Y$ είναι το $\Q(\zeta_p)^+ = \Q(\zeta_p + \zeta_p^{-1})$.
\end{prop}

\begin{proof}
	Το $X = \{1,\omega,\omega^2,\ldots,\omega^{p-2}\}$ είναι η ομάδα χαρακτήρων που αντιστοιχεί στο $\Q(\zeta_p)$, εφόσον αυτοί οι χαρακτήρες 
	είναι $p-1$ το πλήθος και διακεκριμένοι. Άρα αποτελούν όλο το $\Gal(\Q(\zeta_p)/\Q)^{\wedge}$. Επιπλέον, έχουμε ότι $|Y^\perp| = |X|/|Y| = 2$. 
	Καθώς $[\Q(\zeta_p):\Q(\zeta_p)^+]=2$ και η ομάδα $\Gal(\Q(\zeta_p)/\Q)$ είναι κυκλική, δηλαδή έχει μοναδική υποομάδα τάξης 2, παίρνουμε ότι 
	το $\Q(\zeta_p)^+$ είναι το σταθερό σώμα της $Y^\perp$. 
\end{proof}

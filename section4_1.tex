
Σε αυτό το κεφάλαιο θα έχουμε ως σκοπό να αποδείξουμε το θεώρημα του \tl{Iwasawa}, το οποίο 
περιγράφει έναν ρυθμό αύξησης του μεγέθους των ομάδων κλάσεων ιδεωδών για τις $\Z_p$-επεκτάσεις. 
Έχοντας ένα σώμα αριθμών $K$, μια $\Z_p$-επέκταση του $K$ είναι μια επέκταση $K_{\infty}/K$ για την οποία ισχύει 
$\Gal(K_\infty/K)\cong \Z_p$, όπου εδώ εννούμε την προσθετική ομάδα των $p$-αδικών ακέραιων. Καθώς όπως θα δούμε 
$K_\infty = \cup K_n$ για κάποιες ενδιάμεσες επεκτάσεις $K_n/K$, τα βήματα που θα ακολουθήσουμε για να αποδείξουμε το θεώρημα είναι τα εξής:
\begin{enumerate}
	\item Περιγραφή της άλγεβρας \tl{Iwasawa} $ \Lambda = \Z_p[[T]]$.
	\item Απόδειξη ενός θεωρήματος δομής για τα πρότυπα \tl{Iwasawa}.
	\item Χρήση θεωρίας κλάσεων σωμάτων για να παίρνουμε πληροφορίες στα διάφορα στρώματα με πεπερασμένη τάξη, δηλαδή στις $p$-\tl{Sylow} υποομάδες των ομάδων κλάσεων των $K_n$, καθώς ανεβαίνουμε τον πύργο $K_\infty$.
\end{enumerate}

\noindent Να σημειώσουμε ότι στα παρακάτω οι τοπολογίες που θα συναντάμε ταυτίζονται και οι ομάδες είναι ισόμορφες ως τοπολογικές. 
Δηλαδή, στις $\Z_p$-επεκτάσεις η τοπολογία \tl{Krull} της άπειρης ομάδας \tl{Galois}, με την επαγόμενη τοπολογία που παίρνει 
το αντίστροφο όριο και η την τοπολογία της προσθετικής ομάδας $\Z_p$ που επάγεται από την $p$-αδική νόρμα ταυτίζονται.

%(στα παρακάτω οι τοπολογίες ταυτίζονται: επαγόμενη στο \tl{inverse limit}, τοπολογία \tl{krull} και τοπολογία από $p$-αδική νόρμα)
Θα σταθεροποιήσουμε τώρα έναν περιττό πρώτο $p$, ωστόσο και η περίπτωση $p=2$ ακολουθεί τα ίδια αποτελέσματα, αλλά θα χρειαζόταν να επιβαρύνουμε 
τους συμβολισμούς. Για την πλήρη διατύπωση μπορεί να ανατρέξει κανείς στο \cite{Wash}. Θα ξεκινήσουμε με την ύπαρξη μιας $\Z_p$-επέκτασης για ένα σώμα αριθμών. Άρχικά για το $\Q$, υπενθυμίζουμε ότι:
$$\Gal(\Q(\zeta_{p^{n+1}})/\Q) \cong (\Z/p^{n+1}\Z)^\times \cong (\Z/p\Z)^\times \times \Z/p^n\Z.$$

$ $\newline
Έστω $\Q_n$ το σταθερό σώμα του $(\Z/p\Z)^\times$, δηλαδή $\Q_n = \Q(\zeta_{p^{n+1}})^{(\Z/p\Z)^\times}$. Από την θεωρία \tl{Galois} ξέρουμε ότι $\Gal(\Q_n/\Q)\cong \Z/p^n\Z$. Θέτουμε $\Q_\infty = \cup \Q_n$. Τότε έχουμε ότι
\begin{align*}
	\Gal(\Q_\infty/\Q) %&= \varprojlim_n \Gal(\Q_\infty/\Q)/\Gal(\Q_\infty/\Q_n) \\
	&\cong  \varprojlim_n \Gal(\Q_n/\Q)\\
	&\cong \varprojlim_n (\Z/p^n\Z) \\
	&\cong \Z_p ,
\end{align*}
εφόσον η απεικόνιση $\sigma \longmapsto (\sigma|_{\Q_n})_n$ είναι ισομορφισμός τοπολογικών ομάδων. Άρα υπάρχει $\Z_p$-επέκταση για τους ρητούς.

Για να δείξουμε την ύπαρξη για ένα τυχαίο σώμα αριθμών $K$, θέτουμε $K_\infty = K\Q_\infty$ το σύνθετο σώμα. Συνεπώς, έχουμε ότι 
$\Gal(K_\infty/K)\cong \Gal(\Q_\infty/K\cap \Q_\infty)$. Το $\Q_\infty\cap K$ είναι ενδιάμεση επέκταση της $\Q_\infty/\Q$ και άρα από την αντιστοιχία 
\tl{Galois} για τις άπειρες επεκτάσεις έχουμε ότι η $\Gal(\Q_\infty/K\cap\Q_\infty)$ είναι κλειστή υποομάδα της 
$\Gal(\Q_\infty/\Q)$. Άρα κάτω από τον ισομορφισμό, είναι της μορφής $p^n\Z_p$ για κάποιο $n \in \mathbb{N}$, αφού αυτή τη μορφή έχουν τα κλειστά υποσύνολα των $p$-αδικών ακεραίων. 
Πράγματι, έστω $H$ κλειστή υποομάδα του $\Z_p$ και $h \in H$ ένα στοιχείο με ελάχιστη εκτίμηση. Από την ιδιότητα της υποομάδας έχουμε 
ότι $h\Z \subseteq H$ και λόγω ότι η $H$ είναι κλειστή έχουμε ότι $h\Z_p\subseteq H$. Ωστόσο, καθώς επιλέξαμε το $h$ να έχει ελάχιστη εκτίμηση, πρέπει να είναι της μορφής $H=h\Z_p = p^n\Z_p$ για κάποιο $n$. Άρα έχουμε
$$\Gal(K_\infty/K) \cong p^n\Z_p \cong \Z_p$$ 
όπως θέλαμε, εφόσον και η απεικόνιση $x \in \Z_p \longmapsto p^n x \in p^n \Z_p$ είναι ισομορφισμός τοπολογικών ομάδων. Άρα έχουμε την ύπαρξη $\Z_p$-επεκτάσεων για τυχαία σώματα αριθμών. Η επέκταση που κατασκευάσαμε είναι γνωστή ως {\em κυκλοτομική}. Μπορεί για κάποιο $K$ να υπάρχουν και άλλες $\Z_p$-επεκτάσεις, αλλά έχουμε εξασφαλίσει ότι κάθε σώμα αριθμών έχει τουλάχιστον μία.


Έστω τώρα $K_\infty$ μια τυχαία $\Z_p$-επέκταση του $K$. Θέλουμε να δείξουμε ότι μπορούμε να σκεφτόμαστε το $K_\infty$ σαν ένωση από μια ακολουθία ενδιάμεσων σωμάτων:
$$K=K_0 \subset K_1 \cdots \subset K_\infty = \bigcup K_n$$
για τα οποία ισχύει 
$$\Gal(K_n/K)\cong \Z/p^n \Z.$$
Κοιτώντας πάλι την αντιστοιχία \tl{Galois} για τις άπειρες επεκτάσεις, ξέρουμε ότι οι κλειστές υποομάδες $p^n\Z_p$ θα απεικονίζονται σε ενδιάμεσα σώματα. Άρα θέτουμε ως $K_n$ τα σταθερά σώματα:
$$K_n := K_\infty^{p^n \Z_p},$$
όπου πράγματι 
$$\bigcup\limits_{n\geq 0}K_n = \bigcup\limits_{n\geq 0 } K_\infty^{p^n\Z_p} = K_\infty^{\{1\}} = K_\infty .$$


\noindent Μπορούμε πλέον να διατυπώσουμε το θεώρημα που θα αποδείξουμε στο τέλος του κεφαλαίου.

\begin{theorem}[\tl{Iwasawa}] \label{theorem4.1}
	Έστω $K_\infty/K$ μια $\Z_p$-επέκταση και $h_n$ να είναι η τάξη της ομάδας κλάσεων του $K_n$. Αν $h_n = p^{e_n}r$ με  $(r,p)=1$, τότε υπάρχουν 
	ακέραιοι $\lambda \geq 0, \mu \geq 0, \nu$ και $n_0$ έτσι ώστε
	$$e_n = \lambda n + \mu p^n + \nu$$ για κάθε $n\geq n_0$, όπου τα $\lambda,\mu,\nu$ είναι όλα ανεξάρτητα του $n$. 
\end{theorem}






\section{Δακτύλιοι Δυναμοσειρών}

Ένα από τα κύρια αντικείμενα της θεωρίας \tl{Iwasawa} είναι η \tl{Iwasawa} άλγεβρα $\Lambda = \Z_p[[T]]$, δηλαδή ο δακτύλιος τυπικών δυναμοσειρών στο $T$ 
με συντελεστές από το $\Z_p$. Θα περιγράψουμε στη συνέχεια γενικά αποτελέσματα για τις πεπερασμένες επεκτάσεις της $\Lambda$ που θα χρησιμοποιηθούν για την απόδειξη του θεωρήματος \ref{theorem4.1}.


Έστω $K/\Q_p$ μια πεπερασμένη επέκταση, $\mathcal{O}:= \mathcal{O}_K$ ο δακτύλιος ακεραίων του $K$, $\mathfrak{p}$ το μέγιστο ιδεώδες του $\mathcal{O}$ και $\pi$ ένας γεννήτορας του, δηλαδή $\mathfrak{p} = (\pi)$. Ξεκινάμε με το να αποδείξουμε έναν αλγόριθμο διαίρεσης για την άλγεβρα $\Lambda_{\mathcal{O}} := \mathcal{O}[[T]]$.

\begin{prop}
	Έστω $f,g \in \Lambda_{\mathcal{O}}$ με $f = a_0 + a_1 T + \cdots$ με $a_i \in \mathfrak{p}$ για κάθε $0\leq i \leq n-1$ και $a_n \in \mathcal{O}^\times$. Τότε υπάρχουν μοναδικά $q \in \Lo$ και $r \in \mathcal{O} [T]$ με βαθμό $\deg r\leq n-1$ έτσι ώστε

	$$g = qf + r$$
\end{prop}

\begin{proof}
	Αρχίζουμε την απόδειξη με το να ορίσουμε έναν \tl{shift} τελεστή $\tau_n:= \tau: \Lo \longrightarrow \Lo$ έτσι ώστε
	$$b_0 + b_1 T + b_2 T^2 + \cdots \longmapsto b_n + b_{n+1}T + b_{n+2}T^2 + \cdots$$
	δηλαδή 
	$$\tau\left(\sum\limits_{i=0}^\infty b_i T^i\right) = \sum\limits_{i=n}^\infty b_i T^{i-n}$$

	\noindent Ο τελεστής $\tau$ είναι προφανώς $\mathcal{O}$-γραμμικός. Επιπλέον, ισχύουν οι σχέσεις
	\begin{itemize}
		\item $\tau(T^n h(T)) = h(T)$.
		\item $\tau(h(T)) = 0$ αν και μόνο αν το $h$ είναι πολυώνυμο βαθμού $\leq n-1$.
	\end{itemize}

	\noindent Από την υπόθεση για το $f$ έχουμε ότι 
	\begin{equation}
		\label{eq4.1}
		f(T) = \pi P(T) + T^n U(T),
	\end{equation}
	όπου το $P(T)$ είναι πολυώνυμο βαθμού μικρότερου ή ίσου με $n-1$ και το $U(T)$ είναι αντιστρέψιμο στο $\Lo$, καθώς ο σταθερός όρος του $a_n$ είναι 
	αντιστρέψιμος. Θέτουμε
	$$q(T) = \frac{1}{U(T)} \sum\limits_{j=0}^\infty (-1)^j \pi^j \left( \tau \circ \frac{P}{U}\right)^j \circ \tau(g),$$ όπου για παράδειγμα έχουμε
	$$\left( \tau \circ \frac{P}{U}\right)^2 \circ \tau(g) = \tau \left(\frac{P}{U}\left(\tau \left(\frac{P}{U}\cdot \tau(g)\right)\right)\right)$$

	\noindent Καθώς στην $p$-αδική νόρμα που επάγεται στο $K$ ισχύει ότι μια σειρά συγκλίνει αν και μόνο αν η ακολουθία τείνει στο μηδέν, η ύπαρξη του $\pi^j$ αναγκάζει το $q$ να ανήκει στο $\Lo$. Χρησιμοποιώντας την σχέση \ref{eq4.1} έχουμε
	$$qf = \pi q P + T^n q U,$$
	Εφαρμόζοντας τον $\tau$ παίρνουμε 

	\begin{equation}
		\label{eq4.2}
		\tau(qf) = \pi \tau(qP) + \tau(T^nqU) = \pi \tau(qP)+qU
	\end{equation}
	και κοιτάμε τώρα το $\pi \tau(qP)$:

	\begin{align*}
		\pi \tau(qP) &= \pi \tau \left( \frac{P}{U} \sum\limits_{j=0}^\infty (-1)^j \pi^j \left(\tau \circ \frac{P}{U}\right)^j \circ \tau(g)\right) \\
		&= \sum\limits_{j=0}^\infty (-1)^j \pi^{j+1}\left(\tau\circ \frac{P}{U}\right)^{j+1}\circ \tau(g) \\
		&= \pi \left(\tau \circ \frac{P}{U}\right)\circ \tau(g) - \pi^2\left(\tau \circ \frac{P}{U}\right)^2 \circ \tau(g) + \cdots \\
		&=\tau(g) - \left(\tau(g) - \pi \left(\tau \circ \frac{P}{U}\right)\circ \tau(g) + \pi^2 \left(\tau \circ \frac{P}{U}\right)^2 \circ \tau(g) + \cdots \right) \\
		&= \tau(g) - Uq.
	\end{align*}

	\noindent Συνδυάζοντας το με την σχέση \ref{eq4.2} παίρνουμε ότι 
	$$\tau(qf) = \tau(g).$$

	\noindent Συνεπώς, έχουμε ότι τα $\tau(qf)$ και $\tau(g)$ διαφέρουν μόνο κατά ένα πολυώνυμο βαθμού μικρότερου ίσου του $n$. Για την μοναδικότητα, 
	υποθέτουμε ότι υπάρχουν $q_1,q_2,r_1$ και $r_2$ με $g= q_1f +r_1 = q_2 f +r_2$. Τότε έχουμε ότι $(q_1-q_2)f + (r_1-r_2) = 0$. Υποθέτουμε ότι 
	$q_1 \neq q_2$ και $r_1\neq r_2$. Άρα μπορούμε να υποθέσουμε ότι $\pi \nmid (q_1-q_2)$ ή $\pi \nmid (r_1-r_2)$. Κοιτώντας αυτά \tl{modulo} $\pi$, έχουμε 
	ότι $r_1 \equiv r_2 (\operatorname{mod} \pi)$, καθώς $\pi|a_i$ για τα $1\leq i \leq n-1$. Έτσι, $\pi | (q_1-q_2)f$. Ωστόσο, γνωρίζουμε ότι $\pi\not | f$ 
	καθώς $a_n \in \mathcal{O}^\times$, άρα καταλήγουμε στο άτοπο $\pi | (q_1 - q_2)$.
\end{proof}


\begin{defn} Έστω $P(T) = T^n + a_{n-1}T^{n-1} + \cdots + a_1 T + a_0 \in \mathcal{O}[T]$. Θα λέμε το $P(T)$ είναι \tl{distinguished} αν $a_i \in \p$ για τα $0\leq i \leq n-1$.
\end{defn}

\begin{theorem}[\tl{p-adic Weierstrass Preparation Theorem}]
	Έστω $f(T) = \sum\limits_{i=0}^\infty a_i T^i \in \Lo$ και υποθέτουμε ότι υπάρχει $n \in \mathbb{N}$ με $a_i\in \p$ για όλα τα $0\leq i \leq n-1$, ενώ $a_n \in \mathcal{O}^\times$. Τότε υπάρχει μοναδικό $U(T) \in \Lo$ αντιστρέψιμο και μοναδικό $P(T) \in \mathcal{O}[T]$ ένα \tl{distinguished} πολυώνυμο βαθμού $n$, έτσι ώστε
	$$f(T) = P(T)U(T).$$

	\noindent Αν το $f(T) \in \Lo$ είναι μη μηδενικό, τότε υπάρχει $\mu \in \Z, \mu \geq 0$ και $P(T) \in \mathcal{O}[T]$ \tl{distinguished} πολυώνυμο βαθμού 
	το πολύ $n$ και ένα αντιστρέψιμο $U(T)\in \Lo$ έτσι ώστε
	$$f(T) = \pi^m P(T) U(T).$$
\end{theorem}

\begin{proof} Ξεκινάμε εφαρμόζοντας τον αλγόριθμο διαίρεσης στο $g(T) = T^n$ και $f(T)$ ώστε να πάρουμε 
	$$q(T) = \sum\limits_{i=0}^\infty q_i T^i \in \Lo$$
	και $r(T) \in \mathcal{O}[T]$ με 
	\begin{equation}
		\label{eq4.3}
		T^n = f(T)q(T) + r(T),
	\end{equation}

	\noindent όπου $\deg r(T) \leq n-1$. Αν θεωρήσουμε αυτή την ισότητα \tl{modulo} $\pi$ βλέπουμε ότι 
	$$T^n \equiv q(T) (a_nT^n + a_{n+1}T^{n+1} + \cdots ) + r(T) (\operatorname{mod}\pi).$$

	\noindent Καθώς $\deg r(T)\leq n-1$, έχουμε ότι $r(T)\equiv 0 (\operatorname{mod}\pi)$ εφόσον είναι το μόνο στην παραπάνω σχέση που 
	έχει όρους με βαθμό μικρότερο του $n-1$. Συνεπώς, το $T^n-r(T)$ είναι \tl{distinguished} πολυώνυμο, το οποίο θέτουμε ως $P(T)$. Κοιτώντας την σχέση 
	\ref{eq4.3} και τους συντελεστές του $T^n$, έχουμε ότι $q_0 a_n\equiv 1 (\operatorname{mod}\pi)$. Συνεπώς, $\pi \nmid q_0$ το οποίο σημαίνει ότι 
	$q_0 \in \mathcal{O}^\times$, δηλαδή το $q(T)$ είναι αντιστρέψιμο. Άρα έχουμε
	$$T^n -r(T) = f(T)q(T)$$
	δηλαδή,
	$$f(T) = P(T)U(T),$$ όπου $U(T) = q(T)^{-1}$.

	Για τη μοναδικότητα, κάθε \tl{distinguished} πολυώνυμο γράφεται ως $P(T) = T^n - r(T)$. Έτσι, μετατρέπουμε την σχέση $f(T) = P(T) U(T)$ σε μια σχέση της μορφής $T^n = f(T)q(T)+r(T)$ και παίρνουμε την μοναδικότητα από την μοναδικότητα στον αλγόριθμο διαίρεσης. Για το δεύτερο επιχείρημα, απλά βγάζουμε κοινό παράγοντα την μεγαλύτερη δυνατή δύναμη του $\pi$.
\end{proof}


\begin{cor} Έστω $f(T) \in \Lo$ μη μηδενικό. Τότε υπάρχουν πεπερασμένα το πλήθος $z \in \mathbb{C}_p$ με $|z|_p < 1$ και $f(z) = 0$.
\end{cor}

\begin{proof}
	Γράφουμε $f(T) = \pi^m P(T)U(T)$. Καθώς το $U$ είναι αντιστρέψιμο έχουμε ότι $U(z)U^{-1}(z)=1$, δηλαδή $U(z)\neq 0$. Άρα αν $f(z)=0$, έπεται ότι $P(z)=0$. Ωστόσο, το $P$ είναι πολυώνυμο πάνω από το σώμα $K$, άρα έπεται το αποτέλεσμα.
\end{proof}

\begin{lemma} Έστω $P(T)\in \mathcal{O}[T]$ ένα \tl{distinguished} πολυώνυμο και $g(T) \in \mathcal{O}[T]$. Υποθέτουμε ότι $\frac{g(T)}{P(T)} \in \Lo$, τότε $\frac{g(T)}{P(T)} \in \mathcal{O}[T]$.
\end{lemma}

\begin{proof}
	Έστω $f(T) \in \Lo$ έτσι ώστε $\frac{g(T)}{P(T)} = f(T)$, δηλαδή $g(T) = f(T)P(T)$. Έστω $z$ μια ρίζα του $P(T)$. Τότε
	$$0 = P(z) =z^n + (\text{πολλαπλάσιο του } \pi)$$
	εφόσον το $P$ είναι \tl{distinguished}. Τότε καθώς $\pi | z$ έχουμε ότι $|z|_p < 1$. Άρα έχουμε ότι το $f(T)$ συγκλίνει στο $z$ και άρα $g(z) = 0$, δηλαδή $(T-z)|g(T)$. Συνεχίζοντας με το ίδιο επιχείρημα για τις υπόλοιπες ρίζες, όπου πιθανά επεκτείνουμε το $K$ να τις περιέχει όλες και κοιτάμε τον αντίστοιχο δακτύλιο ακεραίων, βλέπουμε ότι $P(T)|g(T)$ ως πολυώνυμα.
\end{proof}
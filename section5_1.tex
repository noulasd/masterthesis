\section{Εισαγωγή}

Σε αυτή την ενότητα θα δώσουμε τους απαραίτητους ορισμούς για να μπορέσουμε να διατυπώσουμε και να καταλάβουμε 
την κύρια εικασία του \tl{Iwasawa}. Θα την διατυπώσουμε με όρους πλήρως πραγματικών σωμάτων, ωστόσο ως εικασία αποδείχθηκε για πρώτη φορά 
το 1984 από τους \tl{A. Wiles} και \tl{B. Mazur} μόνο για το $\Q$ και τις αβελιανές του επεκτάσεις \cite{MW}. Στην συνέχεια το 1990 ο \tl{Wiles} την απέδειξε για όλα τα πλήρως πραγματικά σώματα \cite{Wiles2}.

Έστω $F$ ένα πλήρως πραγματικό σώμα και $F\subset F_1 \subset \cdots \subset F_\infty$ να είναι η κυκλοτομική $\Z_p$-επέκταση του $F$. Έστω $\gamma_0 \in \Gal(F_\infty/F)$ να είναι το στοιχείο που αντιστοιχεί στο $1 \in \Z_p$ μέσα από τον ισομορφισμό $\Gal(F_\infty/F)\cong \Z_p$, έτσι το $\gamma_0$ να είναι ένας τοπολογικός γεννήτορας του $\Gal(F_\infty/F)$. Παρατηρούμε ότι εδώ το $F_\infty$, άρα και κάθε $F_n$, διακλαδίζεται πλήρως στο $p$. Άρα $I_{\q} = \Gal(F_\infty/F)$, δηλαδή η ομάδα αδράνειας του μοναδικού πρώτου $\q$ πάνω από το $p$ είναι ολόκληρη η ομάδα \tl{Galois}.

%https://math.stackexchange.com/questions/2957093/correspondence-for-artin-characters
Έστω $\chi$ ένας $p$-αδικός \tl{Artin} χαρακτήρας του $F$, δηλαδή ένας συνεχής ομομορφισμός ομάδων με πεπερασμένη εικόνα
$$\chi : G_F = \Gal(F^{\operatorname{sep}}/F) \longrightarrow \overline{\Q}_p^\times ,$$  όπου $F^{\operatorname{sep}}$ είναι η διαχωρίσιμη θήκη του $F$. Καθώς ο $\chi$ είναι συνεχής, ο πυρήνας $\ker\chi$ θα είναι κλειστή υποομάδα, δηλαδή επιδέχεται την αντιστοιχία \tl{Galois} και άρα μπορούμε να ορίσουμε όπως στους χαρακτήρες \tl{Dirichlet} το σώμα $F^\chi$ που αντιστοιχεί στον χαρακτήρα $\chi$ να είναι το σταθερό σώμα του $\ker\chi$. Δηλαδή, ο χαρακτήρας παραγοντοποιείται μέσα από

$$\chi: \Gal(F^\chi/F) \longrightarrow \left< \zeta_n \right> \subseteq \overline{\Q}_p^\times .$$ 

\noindent Ακολουθώντας τον \tl{Wiles} \cite{Wiles2} λέμε ότι ο $\chi$ είναι τύπου \tl{S} αν $F^\chi \cap F_\infty = F$ και τύπου \tl{W} αν 
$F^\chi \subset F_\infty$. Στο βιβλίο του \tl{Washington} \cite{Wash} αυτά αναφέρονται ως τύπου 1 και 2 αντίστοιχα. Στο εξής θα 
υποθέτουμε ότι το $F^\chi$ είναι και αυτό πλήρως πραγματικό.

Υποθέτουμε επιπλέον ότι ο $\chi$ είναι τύπου \tl{S}. Θέτουμε $F^\chi_n = F_n F^\chi$, έτσι ώστε $$F^\chi_\infty  = F_\infty F^\chi = \bigcup\limits_n F^\chi_n.$$ Το γεγονός ότι ο $\chi$ είναι τύπου \tl{S} μας δίνει τους ισομορφισμούς

$$\Gamma = \Gal(F^\chi_\infty/F^\chi) \longrightarrow \Gal(F_\infty/F)\cong \Z_p$$ και 
$$\Delta = \Gal(F^\chi_\infty/F_\infty) \longrightarrow \Gal(F^\chi/F)$$ και όπως κάναμε στα προηγούμενα κεφάλαια, παίρνουμε όμοιους ισομορφισμούς αντικαθιστώντας τα $F^\chi_\infty$ και $F_\infty$ με τα $F^\chi_n$ και $F_n$ αντίστοιχα. Σκεφτόμαστε πλέον το $\gamma_0$ σαν στοιχείο του $\Gamma$. Έχουμε το ακόλουθο διάγραμμα σωμάτων με τις αντίστοιχες ομάδες

\begin{figure}[H]
    \centering
    \begin{tikzcd}
        & F^\chi_\infty                                      &                                         \\
F^\chi \arrow[ru, "\Gamma", no head] &                                                    & F_\infty \arrow[lu, "\Delta"', no head] \\
        & F \arrow[ru, "\Z_p"', no head] \arrow[lu, no head] &                                        
\end{tikzcd}
\end{figure}


\noindent Όπως στο κεφάλαιο 4 ορίζουμε ως $L_n$ να είναι η μέγιστη αδιακλάδιστη αβελιανή $p$-επέκταση του $F^\chi_n$. Θέτουμε $X_n = \Gal(L_n/F^\chi_n)$, οπότε το $X_n$ είναι ισόμορφο με την $p$-\tl{Sylow} υποομάδα της ομάδας κλάσεων του $F^\chi_n$. Έστω $L = \cup L_n F^\chi_\infty$ και παρατηρούμε ότι

\begin{align*}
    X &= \Gal(L/F^\chi_\infty) \\
    &\cong \varprojlim \Gal(L_nF^\chi_\infty/F^\chi_\infty) \\
    &\cong \varprojlim \Gal(L_n/F^\chi_n) \\ 
    &= \varprojlim X_n
\end{align*}
όμοια με την δουλειά στο κεφάλαιο 4. Έχουμε πλέον το ακόλουθο διάγραμμα σωμάτων με τις αντίστοιχες ομάδες:

\begin{figure}[H]
    \centering
    \begin{tikzcd}
        &                                                                          & L                                                                         &                                                          &                                         \\
        & L_n \arrow[ru, no head]                                                  &                                                                           & F^\chi_\infty \arrow[lu, "X"', no head]                  &                                         \\
L_0 \arrow[ru, no head] &                                                                          & F_n^\chi \arrow[lu, "X_n"', no head] \arrow[ru, "\Gamma^{p^n}"', no head] &                                                          & F_\infty \arrow[lu, "\Delta"', no head] \\
        & F^\chi \arrow[lu, "X_0", no head] \arrow[ru, "\Gamma/\Gamma^{p^n}"', no head] &                                                                           & F_n \arrow[ru, "p^n \Z_p"', no head] \arrow[lu, no head] &                                         \\
        &                                                                          & F \arrow[ru, "\Z/p^n\Z"', no head] \arrow[lu, no head]                    &                                                          &                                        
\end{tikzcd}
\end{figure}

\noindent Όπως προηγουμένως, έχουμε ότι $\Gal(F^\chi_\infty/F) \cong \Delta \times \Gamma$ και δρα στο $X$ με συζυγίες. Καθώς το $X$ 
είναι \tl{pro}-$p$ ομάδα, έχουμε μια φυσιολογική δράση του $\Z_p$ στο $X$. Συνεπώς, το $X$ είναι ένα 
$\Z_p [[\Delta \times \Gamma]]$-πρότυπο, άρα και ένα $\Z_p[[\Gamma]]$-πρότυπο. Χρησιμοποιούμε πάλι το γεγονός ότι 
$\Z_p[[\Gamma]] \cong \Lambda = \Z_p[[T]]$ μέσα από την απεικόνιση $\gamma_0 \longmapsto 1+T$. Υπενθυμίζουμε ότι στο κεφάλαιο 4 είδαμε το ακόλουθο

\begin{equation}\label{eq5.1}
    X \sim \left(\bigoplus\limits_i \Lambda/(p^{\mu_i})\right) \oplus \left(\bigoplus\limits_j \Lambda/(f_j(T)^{m_j})\right)
\end{equation} με τα $f_j$ να είναι ανάγωγα και \tl{distinguished} πολυώνυμα στο $\Z_p[T]$.

$ $\newline
Θέτουμε $V = X \otimes_{\Z_p} \overline{\Q}_p$. Με αυτό έχουμε ότι
$$V \cong \bigoplus\limits_{j}\overline{\Q}_p [T]/(f_j(T)^{m_j})$$ καθώς με το να πάρουμε το τανυστικό γινόμενο με το $\overline{\Q}_p$ μηδενίζουμε τον πυρήνα, τον συνπυρήνα και τα $\Lambda/(p^{\mu_i})$ στην εξίσωση \ref{eq5.1}. Άρα το $V$ είναι ένας διανυσματικός χώρος πεπερασμένης διάστασης. Θέτουμε
$$f_X(T) = \prod\limits_{j}f_j(T)^{m_j} .$$ 

\noindent Καθώς $T \longleftrightarrow \gamma_0 -1$ μέσα από τον ισομορφισμό $\Lambda \cong \Z_p[[\Gamma]]$, έχουμε ότι το $f_X(T)$ είναι το χαρακτηριστικό πολυώνυμο της δράσης του $\gamma_0-1$ στο $V$. Μπορούμε να πάρουμε επιπλέον πληροφορία με την δουλειά που κάναμε με τα ορθογώνια ταυτοδύναμα στοιχεία. Ο διανυσματικός χώρος $V$ είναι ένα $\overline{\Q}_p[\Delta]$-πρότυπο, άρα έχουμε 

$$V = \bigoplus\limits_{\psi \in \Delta^\wedge} \varepsilon_\psi V.$$ Υπενθυμίζουμε ότι μπορούμε να δούμε το $\chi$ σαν χαρακτήρα του $\Gal(F^\chi/F)\cong \Delta$. Ενδιαφερόμαστε για το
$$V^\chi := \varepsilon_\chi V = \{v\in V: \sigma v = \chi(\sigma)v \ \forall \sigma \in \Delta\}.$$ Είναι ξεκάθαρο ότι το $V^\chi$ παραμένει ένα $\Gamma$-πρότυπο, εφόσον το $\gamma_0$ δρα σαν $1+T$ όπως πριν. Θέτουμε $f_\chi(T)$ να είναι το χαρακτηριστικό πολυώνυμο της δράσης του $\gamma_0-1$ στο $V^\chi$. Παρατηρούμε ότι $f_\chi(T) \mid f_X(T)$. Το χαρακτηριστικό πολυώνυμο μας δίνει μόνο την μισή πληροφορία που χρειαζόμαστε για την διατύπωση της κύριας εικασίας. Πριν προχωρήσουμε στο δεύτερο μισό, αξίζει να σημειωθεί ότι όταν πήραμε το τανυστικό γινόμενο με το $\overline{\Q}_p$ χάσαμε όλη τη πληροφορία που περιείχε το $\mu = \sum \mu_i$ στην εξίσωση \ref{eq5.1}. Θα επιστρέψουμε σε αυτό και θα δούμε πώς θα αναδιατυπώσουμε το παραπάνω επιχείρημα για να διατηρηθεί αυτή η πληροφορία.

Πρέπει τώρα να επιστρέψουμε στις $p$-αδικές $L$-συναρτήσεις. Για το σώμα $\Q$, είδαμε στο κεφάλαιο 3 την παρεμβολή που 
κάνουν με τις κλασικές $L$-συναρτήσεις. Έστω $\psi$ ένας χαρακτήρας ενός πλήρως πραγματικού σώματος, έτσι ώστε το $F^\psi$ να είναι 
και αυτό πλήρως πραγματικό. Οι \tl{Deligne} και \tl{Ribet} \cite{DR} απέδειξαν την ύπαρξη μιας $p$-αδικής $L$-συνάρτησης 
$\mathcal{L}_p(s,\psi)$ που αντιστοιχεί στο $\psi$, γενικεύοντας εκείνη στην περίπτωση του σώματος $\Q$ με τις ίδιες ιδιότητες παρεμβολής. Ορίζουμε


$$H_\psi(T) = \begin{cases}
    \psi(\gamma_0)(1+T)-1, & \psi \text{ είναι τύπου } W \text{ ή τετριμμένο}, \\
    1, & \text{ διαφορετικά. }
\end{cases}$$

\noindent Οι \tl{Deligne} και \tl{Ribet} \cite{DR} απέδειξαν επίσης ότι υπάρχει $G_\psi(T) \in \mathcal{O}_\psi[[T]]$ έτσι ώστε
$$\mathcal{L}_p (1-s,\psi) =\frac{ G_\psi ((1+p)^s -1)}{H_\psi((1+p)^s -1)},$$ όπου $\mathcal{O}_\psi :=\Z_p[\psi]$ ο δακτύλιος πάνω από το $\Z_p$ που παράγεται από τις τιμές που παίρνει ο χαρακτήρας $\psi$. Θα χρησιμοποιούμε τον συμβολισμό $\Lambda_\psi$ για το $\mathcal{O}_\psi[[T]]$. Επιπλέον, αν $\rho$ είναι ένας χαρακτήρας τύπου \tl{W} τότε 

$$G_{\psi \rho}(T) = G_\psi(\rho(\gamma_0)(1+T)-1).$$

\noindent Έστω $\chi$ ένας περιττός χαρακτήρας και θέτουμε $\psi = \chi^{-1}\omega$, όπου $\omega$ είναι ο χαρακτήρας \tl{Teichmüller}. Καθώς ο $\psi$ είναι άρτιος χαρακτήρας έχουμε ότι υπάρχει για αυτόν η αντίστοιχη $p$-αδική $L$-συνάρτηση μαζί με το $G_\psi$. Από το θεώρημα προπαρασκευής του \tl{Weierstrass} για το $G_\psi((1+p)(1+T)^{-1}-1)$ έχουμε ότι 

$$G_\psi((1+p)(1+T)^{-1}-1) =  \pi^{\mu^{\operatorname{an}}_\chi} g_\psi(T)u_\psi(T),$$ όπου $\pi$ είναι ένας \tl{uniformizer} του $\mathcal{O}_\psi$, το $g_\psi(T)$ είναι ένα \tl{distinguished} πολυώνυμο και το $u_\psi(T)$ είναι αντιστρέψιμο στο $\Lambda_\psi$. Το $\operatorname{an}$ που προσθέσαμε στον συμβολισμό στο $\mu_\chi$ είναι για να το ξεχωρίσουμε ως το ((αναλυτικό)) $\mu$-αναλλοίωτο. Παρατηρούμε ότι αν το $\chi$ είναι τύπου \tl{S}, τότε $H_\psi(T)=1$ εκτός αν $\chi = \omega$.

\begin{theorem}[Κύρια Εικασία της Θεωρίας \tl{Iwasawa}]\label{conj5.1}
    Για $\chi$ περιττό χαρακτήρα τύπου \tl{S} και $p$ έναν περιττό πρώτο έχουμε

    $$f_\chi(T) = g_{\chi^{-1}\omega}(T) .$$
\end{theorem}

\noindent Για την περίπτωση όπου $p=2$ μπορεί να συμβουλευτεί κανείς το \cite{Wiles2}. Να σημειωθεί ότι δεν χρειαζόμαστε το $\chi$ να είναι τύπου \tl{S} στην περίπτωση των αβελιανών επεκτάσεων του $\Q$.

Όπως αναφέραμε πριν, καθώς φτιάξαμε έναν διανυσματικό χώρο $V$ από το $X$ παίρνοντας το τανυστικό γινόμενο με το $\overline{\Q}_p$, 
χάσαμε την πληροφορία που περιέχουν τα $\mu_\chi$ και $\mu^{\operatorname{an}}_\chi$ της κύριας εικασίας. 
Θα αναφέρουμε τώρα μια διαφορετική προσέγγιση με την οποία θα διατηρήσουμε αυτή τη πληροφορία. 
Υποθέτουμε ότι για τον χαρακτήρα $\chi$ εκτός από το να είναι περιττός και τύπου \tl{S} θέλουμε να έχει τάξη σχετικά πρώτη με 
το $p$. Γράφουμε ως $X^\chi$ για να συμβολίσουμε το $(X\otimes_{\Z_p}\mathcal{O}_\chi)^\chi$ για το οποίο ισχύει ότι

\begin{align*}
    (X\otimes_{\Z_p}\mathcal{O}_\chi)^\chi &= \{x \in X \otimes_{\Z_p} \mathcal{O}_\chi: \ \sigma x = \chi (\sigma) x \ \forall \ \sigma \in \Delta\} \\
    &\cong X \otimes_{\Z_p[\Delta]} \mathcal{O}_\chi,
\end{align*} όπου βλέπουμε το $\mathcal{O}_\chi$ σαν ένα $\Z_p[\Delta]$-πρότυπο μέσα από τον ομομορφισμό δακτυλίων που επάγεται από το $\chi$. Για να είναι καλά ορισμένο αυτό πρέπει να πάρουμε το τανυστικό γινόμενο με το $\Z_p[\chi]$, διαφορετικά ο όρος $\chi(\sigma)x$ δεν έχει νόημα. Έχουμε ότι το $X^\chi$ είναι ένα πεπερασμένα παραγόμενο $\Lambda_\chi$-πρότυπο στρέψης και έχει χαρακτηριστικό πολυώνυμο της μορφής $\pi^{\mu_\chi} f_\chi(T)$.

\begin{theorem}[$\mu$-αναλλοίωτη εικασία]\label{conj5.2}
    Έστω $p$ ένας περιττός πρώτος και $\chi$ ένας περιττός χαρακτήρας τύπου \tl{S} με τάξη σχετικά πρώτη ως προς το $p$. Τότε 
    $$\mu_\chi = \mu^{\operatorname{an}}_\chi .$$
\end{theorem}

%hyphenation
\noindent Η ισχυρή αυτή διατύπωση της $\mu$-αναλλοίωτης εικασίας επιπλέον μας λέει ότι αυτές οι αναλ-λοίωτες είναι $0$ για τις 
κυκλοτομικές $\Z_p$-επεκτάσεις. Για τις αβελιανές επεκτάσεις, είναι γνωστή αυτή η ισχυρή έκδοση της εικασίας, 
και αποδεικνύεται εύκολα ως το θεώρημα 7.15 στο \cite{Wash}. Αντιθέτως, υπάρχουν άλλες μη-κυκλοτομικές $\Z_p$-επεκτάσεις όπου οι $\mu$-αναλλοίωτες δεν είναι $0$. 
Ο \tl{Iwasawa} κατασκεύασε μια τέτοια μη-κυκλοτομική επέκταση με $\mu >0$ στο \cite{Iwa1}.

$ $\newline
Υπάρχει ωστόσο ένας πιο κομψός τρόπος να διατυπωθεί η κύρια εικασία, έτσι ώστε να περιέχεται το νόημα των θεωρημάτων \ref{conj5.1} και \ref{conj5.2} σε μια έκφραση. Έστω $M$ ένα πρότυπο έτσι ώστε

$$M \sim \left(\bigoplus\limits_i \Lambda/(p^{\mu_i})\right) \oplus \left(\bigoplus\limits_j \Lambda/(f_j(T)^{m_j})\right)$$ με τα $f_j(T)$ να είναι ανάγωγα και \tl{distinguished}. Το ιδεώδες που παράγεται από το χαρακτηριστικό πολυώνυμο

$$\operatorname{char}_\Lambda(M) = \left( p^{\sum \mu_i} \prod f_j(T)^m_j\right)$$ λέγεται το {\em χαρακτηριστικό ιδεώδες} του $M$. 
Η κύρια εικασία μπορεί πλέον να διατυπωθεί ως την ισότητα των ακόλουθων ιδεωδών στον δακτύλιο $\Lambda_\chi$:

$$\operatorname{char}_{\Lambda_\chi}(X^\chi) = \left(G_{\chi^{-1}\omega} ((1+p)(1+T)^{-1}-1)\right).$$

\noindent Να σημειωθεί ότι το $u_{\chi^{-1}\omega}(T)$ που προκύπτει όταν εφαρμόζουμε το θεώρημα προπαρασκευής του \tl{Weierstrass} είναι ένα αντιστρέψιμο στοιχείο και άρα δεν παίζει ρόλο στην παραπάνω ισότητα.

Συνοψίζοντας, η κύρια εικασία μας δίνει την επιπλέον πληροφορία για τις $p$-αδικές $L$-συναρτήσεις, ότι εκτός 
από την μια αλγεβρική κατασκευή τους από τον \tl{Iwasawa} με την χρήση των \tl{Stickelberger} στοιχείων, όπως γίνεται 
στο θεώρημα 7.10 στο \cite{Wash}, έχουμε ότι δεν είναι και τίποτα παραπάνω από χαρακτηριστικές δυναμοσειρές συγκεκριμένων δράσεων \tl{Galois} που εμφανίζονται μέσα στην θεωρία των $\Z_p$-επεκτάσεων.
%$ $\newline \textbf{\tl{Note}}
%συναρτήσεων,  και για αυτές τις συναρτήσεις η κύρια εικασία μας λέει ότι δεν 
%είναι 
%(εδώ να διατυπώσω λίγο όλο το παραπάνω καλύτερα με τις δύο προσεγγίσεις $p$-αδική $L$-συνάρτηση ενάντια σε χαρακτηριστικό πολυώνυμο, ότι οι δυναμοσειρές που προκύπτουν από τις δύο προσεγγίσεις είναι ίσες ως προς συντροφικότητα στην $\Lambda_\chi$. Να πω επιπλέον ότι $p$-αδικά των \tl{Deligne, Ribet}? προκύπτουν από \tl{Stickelberger} στοιχεία και τον δίσκο $D\subset \mathbb{C}_p$ που ισχύει η ισότητα με $p$-αδική $L$-συνάρτηση)

%$ $\newline
%(να αναφέρω μεθόδους απόδειξης?)

%Mazur and Wiles proved the Main Conjecture, showing that p-adic L-functions are essentially the characteristic power 
%series of certain Galois actions arising in the theory of Zp-extensions. 

%An especially timely work, the book is an introduction to the theory of p-adic L-functions originated by Kubota and Leopoldt in 1964 as p-adic analogues of the classical L-functions of Dirichlet.

%kubota leopoldt analytic first construction of p-adic l functions
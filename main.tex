\documentclass{report}

%%%%%%%%%%%%%%%%%%%%%%%%%%%%
\usepackage{amsthm}
\usepackage{amsmath}
\usepackage{amssymb}
%%%%%%%%%%%%%%%%%%%%%%%%%%%%%
\usepackage[greek,english]{babel}
\usepackage{alphabeta}
\usepackage[utf8]{inputenc}
\usepackage{mathtools}
\usepackage{blindtext}
\usepackage[T1]{fontenc}
\usepackage{titlesec}
\usepackage{sectsty}
\usepackage{verbatim}
\usepackage{multirow}
\chapternumberfont{\tiny} 
\chaptertitlefont{\Huge}
%ελληνικοι χαρακτηρες σε μαθ pdf utf-8
%%%%%%%%%%%%%%%%%%%%%%%%%%%%%%%%%
\usepackage{tikz-cd}



\usepackage{hyperref}



\usepackage{xcolor}
\usepackage{framed}%frames
\usepackage{float}
\usepackage{array}
\usepackage{pbox}
\usepackage{caption}
%%%%%%%%%%%%%%%%%%%%%%%%
\usepackage{tikz}
%%%%%%%%%%%%%%%%%%%%%%%%%%
%\usepackage{sagetex} needs sagetex.sty file
%%%%%%%%περιθώρια%%%%%%%%%%%%
\usepackage[a4paper,margin=3.5cm]{geometry}

\usepackage{graphicx,import}

\usepackage{titlesec}    
%\titleformat{\chapter}[display]
%{\normalfont%
    %\LARGE% %change this size to your needs for the first line
    %\bfseries}{\chaptertitlename\ \thechapter}{20pt}{%
    %\LARGE %change this size to your needs for the second line
    %}
\titleformat{\chapter}[display]
    {\normalfont\bfseries}{}{0pt}{\LARGE}


\usepackage[n, % o r lambda
advantage,
operators,
sets,
adversary,
landau,
probability,
notions,
logic,
ff,
mm,
primitives,
events,
complexity,
oracles,
asymptotics,
keys ]{cryptocode}

\newtheorem{theorem}{Θεώρημα}[chapter]
\newtheorem{defn}[theorem]{Ορισμός}
\newtheorem{prop}[theorem]{Πρόταση}
\newtheorem{cor}[theorem]{Πόρισμα}
\newtheorem{lemma}[theorem]{Λήμμα}
\newtheorem*{conj}{Εικασία}
%\newtheorem*{remark}{Παρατήρηση}
%\newtheorem{lemma}{Λήμμα}
%\newtheorem{example}{Παράδειγμα}
%\newtheorem{defn}{Ορισμός}
%\newtheorem{prop}{Πρόταση}
%\numberwithin{defn}{chapter}
%\numberwithin{prop}{chapter}
%\numberwithin{theorem}{chapter}
%\newtheorem{cor}{Πόρισμα}
\newtheorem*{remark}{Παρατήρηση}

\newcommand {\tl}{\textlatin}
%%%%%%%%%αριθμηση%%%%%%%%%%%%%%
\renewcommand{\theenumi}{\arabic{enumi}}
\renewcommand{\labelenumi}{{\rm(\theenumi)}}
\renewcommand{\labelenumii}{\roman{enumii}) }
%%%%%%%%%%%% New theorems %%%%%%%%%%%%%%%%%%%%%%%%

%%%%%%%%%%%%%%%%%%%%%%%%%%%%%%%%%%%%%%%%%%%%%%%%%%%
\newcommand{\Z}{\mathbb{Z}}
\newcommand{\Q}{\mathbb{Q}}
\newcommand{\Co}{\mathbb{C}}
\newcommand{\So}{\mathcal{S}}
\newcommand{\C}{\mathcal{C}}
\newcommand{\Gal}{\operatorname{Gal}}
\newcommand{\Frob}{\operatorname{Frob}}
\newcommand{\coker}{\operatorname{coker}}
\newcommand{\p}{\mathfrak{p}}
\newcommand{\q}{\mathfrak{q}}
\newcommand{\Lo}{\Lambda_\mathcal{O}}


\newcommand{\Sheaf}{(\So, \pi, X)}
\usepackage{listings}
\usepackage{color}

\newcommand{\el}{\lambda}
\newcommand{\eL}{\Lambda}
\newcommand{\presheaf}{(S(U),\rho^U_V)_{V\subseteq U\in \tau_X}}
\newcommand{\ds}{(A_{\lambda},\phi^{\lambda_1}_{\lambda_2})}
\newcommand{\dsb}{(B_{\lambda},\psi^{\lambda_1}_{\lambda_2})}
\newcommand{\dlimit}{(A,\phi_{\lambda})}
\newcommand{\openx}{\mathcal{N}^0_x}

\newcommand{\curly}{\mathrel{\leadsto}}

\newcommand{\defeq}{\mathrel{\stackrel{\makebox[0pt]{\mbox{\normalfont\tiny \text{ορσ}}}}{=}}}

\newcommand{\incfig}[1]{%
    \fontsize{12pt}{12pt}\selectfont %χρησιμοποιώ height 600 width 300 pixels
    \def\svgwidth{2in}
    \import{./figures/}{#1.pdf_tex}
}


\begin{document}
	
	
	\selectlanguage{greek}
	%\pagenumbering{roman}
	
	
	\begin{framed}	
		%\vspace{0.3truecm}
		\begin{center}
			\huge Θεωρία \tl{Iwasawa}
		\end{center}
		%\vspace{0.3truecm}
		\vspace{0.3truecm}
		\begin{center}
            Νούλας Δημήτριος\\
            \tl{dnoulas@math.uoa.gr}
\end{center}
		\vspace{0.3truecm}
	\end{framed}
	\vspace*{\fill}
	\begin{center}
	\includegraphics[width=0.5\textwidth]{C:/Users/dimit/Desktop/TeX/uoa_logo}
	%\includegraphics[width=0.5\textwidth]{/mnt/c/Users/dimit/Desktop/TeX/uoa_logo}
	\end{center}
\vspace{1cm}
\pagebreak

\tableofcontents
\pagebreak

\chapter{Εισαγωγή}

\chapter{Προαπαιτούμενα}

\section{Άλγεβρική Θεωρία Αριθμών}

\noindent Έστω $L/K$ μια πεπερασμένη επέκταση σωμάτων αριθμών με δακτύλιους ακεραίων $\mathcal{O}_L$ και $\mathcal{O}_K$ αντίστοιχα.

\begin{theorem}
	Κάθε γνήσιο μη-μηδενικό πρώτο ιδεώδες $\mathfrak{a} \subset \mathcal{O}_K$ έχει μοναδική παραγοντοποίηση:
	$$\mathfrak{a} = \mathfrak{p}_1^{e_1} \cdots \mathfrak{p}_r^{e_r}$$ με $e_i > 0$ και τα $\mathfrak{p}_i$ είναι πρώτα ιδεώδη.
\end{theorem}

Δοθέντος ενός πρώτου ιδεωδούς $\mathfrak{p} \subset \mathcal{O}_K$, μπορούμε να θεωρήσουμε το ιδεώδες $\mathfrak{p} \mathcal{O}_L$ στον δακτύλιο $\mathcal{O}_L$. Με βάση το προηγούμενο θεώρημα μπορούμε να το παραγοντοποιήσουμε σε γινόμενο πρώτων ιδεωδών:
\begin{equation}
	\label{eq2.1}
	\mathfrak{p}\mathcal{O}_L = \mathfrak{p}_1^{e_1} \cdots \mathfrak{p}_r^{e_r}
\end{equation}  με τα $\mathfrak{p}_i$ να είναι πρώτα ιδεώδη του $\mathcal{O}_L$.


\begin{defn}
	Σε μια παραγοντοποίηση όπως στην \ref{eq2.1}, λεμε το $e_i = e(\mathfrak{p}_i / \mathfrak{p})$ δείκτη διακλάδωσης του $\mathfrak{p}$ στο $\mathfrak{p}_i$. Θα λέμε ότι το πρώτο ιδεώδες $\mathfrak{p}$ διακλαδίζεται στο $L$ αν ισχύει $e_i >1$ για κάποιο $i$. Ο βαθμός αδράνειας $f_i = f(\mathfrak{p}_i/ \mathfrak{p})$ είναι η διάσταση του διανυσματικού χώρου $\mathcal{O}_L/\mathfrak{p}_i$ πάνω από το πεπερασμένο σώμα $\mathcal{O}_K/\mathfrak{p}$. 
\end{defn}%residue class field degree f_i

%από εδώ και πέρα μη μηδενικό πρώτο ιδεώδες... ή κατευθείαν σκέτος πρώτος
\begin{prop}
	Ένα πρώτο ιδεώδες $\mathfrak{p}$ στο $\mathcal{O}_K$ διακλαδίζεται στο $\mathcal{O}_L$ αν και μόνο αν $\mathfrak{p} \mid \operatorname{disc} (\mathcal{O}_L/\mathcal{O}_K)$.
\end{prop}
!Τι σημαίνει $disc(\mathcal{O}_L/\mathcal{O}_K)$; η διακρίνουσα ορίζεται για σώματα αριθμών. Λογικά:
$$disc_{\mathcal{O}_K} (\mathcal{O}_L) = \det (T_{L/K}(a_ia_j))$$ όπου $a_i$ βάση του $\mathcal{O}_L$ ως $\mathcal{O}_K$-πρότυπο, που σημαίνει τα $a_i$ είναι βάση του $L$ υπεράνω του $K$ (σωστό με βάση \tl{Milne})
%απόδειξη;
%theorem 24 number fields σελίδα 50 (περίπου)
%solid Milne: the primes that ramify σελίδα 53
%τα ορίζει σελίδα 26 ο Milne


\begin{theorem}
	Με βάση τα παραπάνω έχουμε:
	\begin{equation}
		\sum\limits_{i=1}^r e(\mathfrak{p}_i/\mathfrak{p}) f(\mathfrak{p}_i/\mathfrak{p}) = \sum\limits_{i=1}^r e_i f_i = [L:K]
	\end{equation}
\end{theorem}


Στο εξής θα θεωρούμε ότι η επέκταση $L/K$ είναι \tl{Galois}. Έτσι μπορούμε να απλοιποιήσουμε το προηγούμενο θεώρημα αρκετά. Ξεκινάμε με την ακόλουθη πρόταση.

\begin{prop}
	Η ομάδα $\Gal(L/K)$ δρα μεταβατικά στο σύνολο των πρώτων ιδεωδών $\mathfrak{p_i}$ του $\mathcal{O}_L$ που βρίσκονται υπεράνω του $\mathfrak{p}$.
\end{prop}

\begin{proof} Προς άτοπο, έστω ότι $\sigma (\p_i) \neq \p_j$ για κάθε $\sigma \in \Gal (L/K)$. Υπενθυμίζουμε ότι το $\sigma (\p_i)$ θα είναι και αυτό πρώτο ιδεώδες που θα στέκεται πάνω από το $\p$. Καθώς είμαστε σε περιοχές \tl{Dedekind} τα $\p_i$ και $\sigma(p_i)$ θα είναι μεγιστικά. Άρα $\p_i \not\subseteq \sigma(\p_i)$. Από το αντιθετοαντίστροφο του λήμματος αποφυγής πρώτων παίρνουμε ότι 
	$$\p_i \not\subseteq \bigcup\limits_{\sigma \in \Gal (L/K)} \sigma(\p_i)$$ δηλαδή, υπάρχει $x \in \p_i$ που αποφεύγει όλα τα $\sigma (\p_i)$. Για την νόρμα, παρατηρούμε ότι:
	$$N_{L/K}(x) = \prod\limits_{\sigma \in \Gal (L/K)} \sigma(x)$$ βρίσκεται μέσα στο $\p = \mathcal{O}_K \cap \p_i$, διότι η νόρμα θα βρίσκεται μέσα στο $\mathcal{O}_K$ καθώς και στο παραπάνω γινόμενο εμφανίζεται το $x$ που ανήκει στο ιδεώδες $\p_i$. Έχουμε ότι $x \not \in \sigma(p_i)$ και άρα $\sigma^{-1}(x) \not \in \p_i$ για κάθε $\sigma \in \Gal (L/K)$. Άρα $\prod \sigma^{-1}(x) = \prod \sigma(x) \not \in \p_i\cap \mathcal{O}_K = \p$, το οποίο είναι άτοπο.
\end{proof}
%Υπενθυμίζουμε: robert ash proposition 8.1.1 
%prime avoidance robert ash section 3.1 problems

\begin{cor}
	Έστω $L/K$ \tl{Galois} επέκταση και $0 \neq \p \subset \mathcal{O}_K$ πρώτο ιδεώδες. Τότε $e(\p_i/ \p) = e(\p_j/ \p) = e$ και $f(\p_i / \p) = f(\p_j /\p)=f$ για κάθε $i,j$ της εξίσωσης \ref{eq2.1}. Ειδικότερα, έχουμε $[L:K] = ref$.
\end{cor}
\begin{proof}
	Ο αυτομορφισμός $\sigma$ διατηρεί τις αλγεβρικές σχέσεις:
	$$\sigma (\p \mathcal{O}_L) = \prod\limits_{i=1}^r \sigma(p_i)^{e_i} = \prod\limits_{i=1}^r \p_i^{e_i} = \p \mathcal{O}_L$$ και συγκρίνουμε τους εκθέτες για να πάρουμε ότι είναι ίδιοι. Αν $\sigma(\p_i) = \p_j$ τότε παίρνουμε $f_i = f_j$ από τον ισομορφισμό πεπερασμένων σωμάτων:
	$$\mathcal{O}_L / \p_i \simeq \mathcal{O}_L / \p_j$$ από τον επιμορφισμό που επάγει ο $\sigma$:
	$$\mathcal{O}_L \longrightarrow \mathcal{O}_L / \p_j $$
	$$x \longmapsto \sigma(x) + \p_j$$
\end{proof}


Για $[L:K]=n$ υπενθυμίζουμε την ορολογία:
\begin{center}
\begin{tabular}{ |c|c|c|c|c| } 
	\hline
	 & $e$ & $f$ & $r$ \\
	\hline
	 αδρανές $\quad \quad \quad \quad \quad \quad$ & 1 & $n$ & 1\\ 
	 πλήρως διακλαδιζόμενο & $n$ & 1 & 1 \\ 
	 πλήρως διασπώμενο $\quad $& 1 & 1 & $n$ \\ 
	 \hline
	 \end{tabular}
	\end{center}

\begin{defn}
	Έστω $\q$ ένα πρώτο ιδεώδες του $\mathcal{O}_L$. Η υποομάδα $D_{\q} = \{\sigma \in \Gal (L/K): \sigma(\q) = \q\}$ λέγεται η ομάδα διάσπασης του $\q$ υπεράνω του $K$.
\end{defn}

Από την πρόταση 2.5 και το θεώρημα \tl{orbit-stabilizer} παίρνουμε το ακόλουθο πόρισμα.

\begin{cor}
	Για $L/K$ επέκταση όπως παραπάνω και $\p$ πρώτο ιδεώδες του $\mathcal{O}_K$ έχουμε:
	\begin{enumerate}
		\item $[\Gal(L/K) : D_{\q}] = r$ για κάθε $\q \mid \p$.
		\item $D_{\q} = 1$ αν και μόνο αν το $\p \mathcal{O}_L$ διασπάται πλήρως.
		\item $D_{\q} = \Gal(L/K)$ αν και μόνο αν το $\p \mathcal{O}_L$ διακλαδίζεται πλήρως, δηλαδή $\p \mathcal{O}_L = \q^{n}$ για $n=[L:K]$.
		\item $|D_{\q}| = ef$. 
	\end{enumerate}
\end{cor}

Έχουμε μια φυσική απεικόνιση:
$$D_{\q} \longrightarrow \Gal \left( (\mathcal{O}_L/\q) / (\mathcal{O}_K/\p)\right)$$ που ένα $\sigma \in D_{\q}$ εφόσον κρατάει σταθερό το $\q$ επάγει έναν $\mathcal{O}_L/\q$-αυτομορφισμό $\overline{\sigma}$ ο οποίος κρατάει σταθερό το υπόσωμα $\mathcal{O}_K/\p$, αφού ο $\sigma$ κρατάει σταθερό το $K$. Αποδεικνύεται ότι αυτή η απεικόνιση είναι επί \tl{(S. Lang ANT prop 14)}. %8.1.8 robert b ash

\begin{defn}
	Ο πυρήνας $I_{\q} \subseteq D_{\q}$ του παραπάνω ομομορφισμού λέγεται ομάδα αδράνειας του $\q$ υπεράνω του $K$. Ισχύει ότι:
	$$I_{\q} = \{s \in D_{\q}: \ \sigma(x) = x \mod \q \ \forall x \in L\}$$
\end{defn}

Από το πόρισμα 2.8 έχουμε ότι:
\begin{cor}
	Για $L/K$ επέκταση όπως παραπάνω έχουμε ότι $|I_{\q}| = e$.
\end{cor}


%θεωρία πεπερασμένων σωμάτων Μαλιάκας Galois κεφ 9.
Από την θεωρία πεπερασμένων σωμάτων, η ομάδα \tl{Galois} πεπερασμένου σώματος είναι κυκλική και ένας γεννήτορας είναι ο $\sigma(x) = x^q$, όπου $q$ είναι η τάξη του υποσώματος. Αυτός ο γεννήτορας είναι γνωστός ως ο αυτομορφισμός του \tl{Frobenius}. Στην περίπτωσή μας με $q = |\mathcal{O}_K/\p|$ και $\q \mid \p$ υπάρχει δηλαδή ένας αυτομορφισμός $\overline{\sigma_{\q}}$ του $\mathcal{O}_L/\q$ που σταθεροποιεί το $\mathcal{O}_K/\p$ που δίνεται από την σχέση $\overline{\sigma_{\q}}(x + \q) = x^q + \q$. Άρα από τον ισομορφισμό:
$$D_{\q}/ I_{\q} \simeq \Gal \left( (\mathcal{O}_L/\q) / (\mathcal{O}_K/\p) \right)$$


Έχουμε ότι κάποιο σύμπλοκο $\sigma_{\q} + I_{\q}$ θα αντιστοιχεί στον αυτομορφισμό του \tl{Frobenius}. Κάθε στοιχείο του συμπλόκου θα λέγεται αυτομορφισμός του \tl{Frobenius} στο $\q$ και θα συμβολίζεται με $\Frob_{\q}$. Αν η ομάδα αδράνειας $I_{\q}$ είναι τετριμμένη, δηλαδή $e=1$ και το $\p$ δεν διακλαδίζεται, τότε υπάρχει καλά ορισμένο στοιχείο $\Frob_{\q} \in D_{\q}$. Είναι σημαντικό να μπορούμε να συσχετίσουμε τα $\Frob_{\q_1}$ και $\Frob_{\q_2}$ για διαφορετικά πρώτα ιδεώδη $\q_i \mid \p$. Ξέρουμε ότι υπάρχει $\tau \in \Gal(L/K)$ με $\tau(\q_1) = \q_2$ και εύκολα φαίνεται ότι $D_{\q_2} = \tau D_{\q_1} \tau^{-1}$, καθώς και $\Frob_{\q_2} = \tau \Frob_{\q_1} \tau^{-1}$. Αν η $\Gal(L/K)$ είναι αβελιανή και το $\p$ δεν διακλαδίζεται στο $L$, τότε μπορούμε να ξεχωρίσουμε μοναδικό στοιχείο της $\Gal(L/K)$ που βρίσκεται στην $D_{\q}$ για κάθε $\q \mid \p$. Αυτό το στοιχείο θα το λέμε $\Frob_{\p}$.


%εύκολα: κοιτάω σ \in D_{\q_1} ότι τστ^{-1}(q_2) = ... = q_2 
%Frob: σ(χ) = χ^τάξη O_K/p mod q
% στ^{-1}(χ) = τ^{-1}(χ)^τάξη = (ομομορφισμός) τ^{-1}(χ^τάξη) και παίρνω τ στα δύο μέλη

\begin{prop}
	Έστω $L/K$ επέκταση \tl{Galois} και $\p$ πρώτο ιδεώδες του $\mathcal{O}_K$ και $\q$ πρώτο ιδεώδες του $\mathcal{O}_L$ με $\q \mid \p$. \tl{Completions}!
\end{prop}



\section{Κυκλοτομικά Σώματα}
%In this section we recall some basic facts about cyclotomic fields. These number
%fields will be the primary focus of classical Iwasawa theory, so we give some of
%the relevant details in this section.


\begin{defn}Μια πρωταρχική $n$-οστή ρίζα της μονάδας είναι ένας αριθμός $\zeta_n \in \mathbb{C}$ τέτοιος ώστε $\zeta_n^n = 1$ και $\zeta_n^m \neq 1$ για κάθε $0<m<n$. Το σώμα $\Q(\zeta_n)$ λέγεται το $n$-οστό κυκλοτομικό σώμα. 
\end{defn}
Ορίζουμε το $n$-οστό κυκλοτομικό πολυώνυμο $\Phi_n(x)$ ως εξής:
$$\Phi_n(x) = \prod\limits_{\substack{ 0 < m < n \\ \gcd(m,n)=1}} (x - \zeta_n^m)$$

Οι ρίζες του πολυωνύμου είναι ακριβώς οι πρωταρχικές $n$-οστές ρίζες της μονάδας. Έχουμε $\deg (\Phi_n) = \phi(n)$. Επιπλέον ισχύει ότι $\Phi_n(x) \in \Q[x]$. Αυτό φαίνεται από την σχέση

\begin{equation} \label{eq2.2}
	x^n-1 = \prod\limits_{d \mid n} \Phi_d(x)
\end{equation}

και με επαγωγή στο $n$. Αφού $\Phi_n(\zeta_n) = 0$ έχουμε ότι $[\Q(\zeta_n):\Q] \leq \phi(n)$. Έχουμε ότι η επέκταση $\Q(\zeta_n)/\Q$ είναι \tl{Galois} αφού το $\Phi_n$ διασπάται πλήρως στο $\Q(\zeta_n)$. Εφαρμόζοντας τον μετασχηματισμό \tl{Möbius} στην εξίσωση 2.3 παίρνουμε:
$$\Phi_n(x) = \prod\limits_{d\mid n} (x^d-1)^{\mu(n/d)}$$

\begin{lemma}Έστω $n=p^r$ όπου $p$ πρώτος. Τότε:
	\begin{enumerate}
	\item $[\Q(\zeta_{p^r}):\Q] = \phi(p^r) = p^r-p^{r-1}$.
	%ως ιδεώδη:
	\item $p \mathcal{O}_{\Q(\zeta_{p^r})} = (1-\zeta_{p^r})^{\phi(p^r)}$ και το $(1-\zeta_{p^r})$ είναι πρώτο ιδεώδες του $\mathcal{O}_{\Q(\zeta_{p^r})}$.
	\item $\mathcal{O}_{\Q(\zeta_{p^r})} = \Z[\zeta_{p^r}]$.
	\item $\Delta_{\Q(\zeta_{p^r})} = \pm p^{p^{r-1}(pr-r-1)}$. 
	\end{enumerate}
\end{lemma}

\begin{proof}
	Άρχικά έχουμε $\Z[\zeta_{p^r}]\subseteq \mathcal{O}_{\Q(\zeta_{p^r})}$ αφού τα στοιχεία του πρώτου είναι άθροισμα ακεραίων της μορφής $\sum\limits_{i=0}^{p^r-1} a_i \zeta_{p^r}^i$ και τα ακέραια στοιχεία αποτελούν δακτύλιο.
	Αν $\zeta_{p^r}^{\prime}$ είναι μια άλλη $p^r$ ρίζα της μονάδας, τότε υπάρχουν $s,t\in\Z$ με $p\not\mid st$ και $\zeta_{p^r} = (\zeta_{p^r}^{\prime})^t, \zeta_{p^r}^{\prime} = \zeta_{p^r}^s$. Έτσι, $\Q(\zeta_{p^r}) = \Q(\zeta_{p^r}^{\prime})$ και $\Z[\zeta_{p^r}] = \Z[\zeta_{p^r}^{\prime}]$. Επιπλέον,
	
	$$\frac{1-\zeta^{\prime}_{p^r}}{1-\zeta_{p^r}} = \frac{1-\zeta^s_{p^r}}{1-\zeta_{p^r}} = 1 + \zeta_{p^r} + \cdots + \zeta_{p^r}^{s-1} \in \Z[\zeta_{p^r}]$$ και όμοια, $(1-\zeta_{p^r})/(1-\zeta_{p^r}^{\prime}) \in \Z[\zeta_{p^r}]$. Αρα το $(1-\zeta^{\prime}_{p^r})$ είναι αντιστρέψιμο στο $\Z[\zeta_{p^r}]$ και άρα και στο $\mathcal{O}_{\Q(\zeta_{p^r})}$.

	$$\Phi_{p^r} (x) = \frac{x^{p^r}-1}{x^{p^{r-1}}-1} = \frac{t^p - 1}{t-1} = 1+t+\cdots + t^{p-1}, \ t = x^{p^{r-1}}$$ και $\Phi_{p^r}(1) = p$. Από τους ορισμούς φαίνεται ότι:
	\begin{align*}
		\Phi_{p^r}(1) &= \prod (1-\zeta^{\prime}_{p^r}) \\
		&= \prod \frac{1-\zeta^{\prime}_{p^r}}{1-\zeta_{p^r}}(1-\zeta_{p^r}) \\
		&= u (1-\zeta_{p^r})^{\phi(p^r)} 
	\end{align*}
	με $u$ αντιστρέψιμο στοιχείο του $\Z[\zeta_{p^r}]$. Άρα παίρνουμε ισότητα στα ιδεώδη του $\mathcal{O}_{\Q(\zeta_{p^r})}$, δηλαδή $p\mathcal{O}_{\Q(\zeta_{p^r})} = (1-\zeta_{p^r})^{\phi(p^r)}$. Συνεπώς, το ιδεώδες $p\mathcal{O}_{\Q(\zeta_{p^r})}$ έχει τουλάχιστον $\phi(p^r)$ πρώτους παράγοντες στο $\mathcal{O}_{\Q(\zeta_{p^r})}$. Άρα (?) παίρνουμε $[\Q(\zeta_{p^r}):\Q] \geq \phi(p^r)$ και συνεπώς 
	$$[\Q(\zeta_{p^r}):\Q] = \phi(p^r) = p^r - p^{r-1}$$
	Επιπλέον, το $(1-\zeta_{p^r})$ παράγει πρώτο ιδεώδες αλλιώς θα είχαμε παραπάνω από $\phi(p^r)$ πρώτους στην παραγοντοποίηση του $p\mathcal{O}_{\Q(\zeta_{p^r})}$. Για την διακρίνουσα, χρησιμοποιούμε τον τύπο με την παράγωγο από την βιβλιογραφία (π.χ. \tl{Milne ANT prop 2.33})
	
	$$disc(\Z[\zeta_{p^r}]/\Z) = \pm N_{\Q(\zeta_{p^r})/\Q} (\Phi^{\prime}_{p^r}(\zeta_{p^r}))$$. Έχουμε
	$$\Phi^{\prime}_{p^r}(\zeta_{p^r}) = \frac{p^r \zeta_{p^r}^{p^r-1}}{\zeta_{p^r}^{p^{r-1}}-1}$$ και
	$$N(\zeta_{p^r}) = \pm 1$$ αρα
	$$N(p^r) = (p^r)^{\phi(p^r)} = p^{r\phi(p^r)}$$ και ισχυριζόμαστε ότι:
	$$N(1-\zeta_{p^r}^{p^s}) = p^{p^s}, \ 0\leq s < r$$
	Πράγματι, το ελάχιστο πολυώνυμο του $1-\zeta_{p^r}$ είναι το $\Phi_{p^r}(1-x)$ που έχει σταθερό όρο $\Phi_{p^r}(1) = p$. Αρα $N(1-\zeta_{p^r})= \pm p$. Έστω $s<r$, το $\zeta_{p^r}^{p^s}$ είναι πρωταρχική $p^{r-s}$-οστή ρίζα της μονάδας, άρα ο ίδιος υπολογισμός για $r-s$ αντί για $r$ δίνει $N_{\Q(\zeta_{p^r}^{p^s})/\Q}(1-\zeta^{p^s}_{p^r}) = \pm p$. Χρησιμοποιώντας την προσεταιριστικότητα  της νόρμας, μαζί με $N_{M/L}(a) = a^{[M:L]}$ για σώματα $M\supset L$, παίρνουμε ότι:
	$$N_{\Q(\zeta_{p^r})/\Q} (1-\zeta^{p^s}_{p^r}) = p^a$$ όπου
	$$a = [\Q(\zeta_{p^r}):\Q(\zeta^{p^s}_{p^r})] = \phi(p^r)/\phi(p^{r-s}) = p^s$$

	Συνεπώς, $N(\Phi^{\prime}_{p^r}(\zeta_{p^r})) =\pm p^c$ όπου $c = p^{r-1}(pr-r-1)$. Άρα η διακρίνουσα του $\Z[\zeta_{p^r}]$ πάνω από το $\Z$ είναι δύναμη του $p$. Άρα και η διακρίνουσα του $\mathcal{O}_{\Q(\zeta_{p^r})}$ πάνω από το $\Z$ είναι δύναμη του $p$ από τον τύπο:
	$$disc(\mathcal{O}_{\Q(\zeta_{p^r})}/\Z) [\mathcal{O}_{\Q(\zeta_{p^r})}: \Z[\zeta_{p^r}]]^2 = disc(\Z[\zeta_{p^r}/\Z])$$
	(\tl{Milne remark 2.24})

	%τι πηλίκο είναι το παρακάτω; το p^M κάνει annihilate
	Επιπλέον, έχουμε ότι το $[\mathcal{O}_{\Q(\zeta_{p^r})}: \Z[\zeta_{p^r}]]$ είναι δύναμη του $p$, άρα $p^M (\mathcal{O}_{\Q(\zeta_{p^r})}/\Z[\zeta_{p^r}]) = 0$ για κάποιο $M$. Δηλαδή, $p^M \mathcal{O}_{\Q(\zeta_{p^r})} \subseteq \Z[\zeta_{p^r}]$. Το χρησιμοποιούμε αυτό για το ιδεώδες $\p = (1-\zeta_{p^r})$ και έχουμε $f(\p/p) =1$ και άρα η παρακάτω απεικόνιση είναι ισομορφισμός:

	$$\Z / p\Z \longrightarrow \mathcal{O}_{\Q(\zeta_{p^r})/(1-\zeta_{p^r})}$$
	Άρα $\Z + (1-\zeta_{p^r})\mathcal{O}_{\Q(\zeta_{p^r})} = \mathcal{O}_{\Q(\zeta_{p^r})}$ και άρα επίσης:
	\begin{equation} \label{eq2.3}
		\Z[\zeta_{p^r}] + (1-\zeta_{p^r})\mathcal{O}_{\Q(\zeta_{p^r})} = \mathcal{O}_{\Q(\zeta_{p^r})} 
	\end{equation}
	η οποία δίνει:
	\begin{equation} \label{eq2.4}
		(1-\zeta_{p^r}) \Z[\zeta_{p^r}] + (1-\zeta_{p^r})^2 \mathcal{O}_{\Q(\zeta_{p^r})} = (1-\zeta_{p^r}) \mathcal{O}_{\Q(\zeta_{p^r})}
	\end{equation}

	Έστω $a \in \mathcal{O}_{\Q(\zeta_{p^r})}$. Τότε από την εξίσωση \ref{eq2.3} παίρνουμε ότι $a=a^{\prime} + \gamma$ με $a^{\prime} \in (1-\zeta_{p^r})\mathcal{O}_{\Q(\zeta_{p^r})}$ και $\gamma \in \Z[\zeta_{p^r}]$. Η εξίσωση \ref{eq2.4} δίνει $a^{\prime} = a^{\prime\prime} + \gamma^{\prime}$ με $a^{\prime\prime} \in (1-\zeta_{p^r})^2 \mathcal{O}_{\Q(\zeta_{p^r})}$ και $\gamma^{\prime} \in \Z[\zeta_{p^r}]$. Άρα $a = (\gamma+\gamma^{\prime}) + a^{\prime\prime}$. Συνεπώς:
	$$\Z[\zeta_{p^r}]+ (1-\zeta_{p^r})^2 \mathcal{O}_{\Q(\zeta_{p^r})} = \mathcal{O}_{\Q(\zeta_{p^r})}$$

	Με επανάληψη, μπορούμε να πάρουμε $\Z[\zeta_{p^r}] + (1-\zeta_{p^r})^m \mathcal{O}_{\Q(\zeta_{p^r})} = \mathcal{O}_{\Q(\zeta_{p^r})}$ για $m \in \mathbb{N}$. Καθώς $(1-\zeta_{p^r})^{\phi(p^r)} = p\cdot u$, $u$ αντιστρέψιμο, έχουμε $\Z[\zeta_{p^r}]+ p^m \mathcal{O}_{\Q(\zeta_{p^r})} = \mathcal{O}_{\Q(\zeta_{p^r})}$ για κάθε $m\in\mathbb{N}$. Ωστόσο, για αρκετά μεγάλο $m$ έχουμε δείξει ότι $p^m \mathcal{O}_{\Q(\zeta_{p^r})} \subseteq \Z[\zeta_{p^r}]$. Άρα πράγματι $\Z[\zeta_{p^r}] = \mathcal{O}_{\Q(\zeta_{p^r})}$. Αυτό μαζί με τον υπολογισμό του $disc(\Z[\zeta_{p^r}]/\Z)$ ολοκληρώνουν την απόδειξη.  
\end{proof}

Μαζί με το ακόλουθο λήμμα, θα γενικεύσουμε την πρόταση για $n \in \mathbb{N}$.
%milne lemma 6.5
\begin{lemma}
	Έστω $K,L$ πεπερασμένες επεκτάσεις του $\Q$ με
	$$[KL:\Q] = [K:\Q]\cdot [L:\Q]$$ και έστω $d = \gcd \left( \operatorname{disc}(\mathcal{O}_K/\Z), \operatorname{disc}(\mathcal{O}_L/\Z)\right)$. Τότε
	$$O_{KL} \subset d^{-1} \mathcal{O}_K \mathcal{O}_L$$
\end{lemma}

\begin{prop}
	Έστω $\zeta_n$ μια πρωταρχική $n$-οστή ρίζα της μονάδας και $K= \Q(\zeta_n)$. Ισχύουν τα ακόλουα:
	\begin{enumerate}
		\item $[K:\Q] = \phi(n)$.
		\item $\mathcal{O}_K = \Z[\zeta_n]$.
		\item Ο πρώτος $p$ διακλαδίζεται στο $K$ αν και μόνο αν $p\mid n$ (εκτός αν $n=2\cdot$περιττός και $p=2$). Ειδικότερα, αν $n=p^r$ με $\gcd(p,m)=1$, τότε 
		$$p \mathcal{O}_K = (\p_1 \cdots \p_s)^{\phi(p^r)}$$ στο $K$ με τα $\p_i$ να είναι διακεκριμένοι πρώτοι στο $K$. %πρώτα ιδεώδη στο O_K αλλά το ίδιο είναι αν δω τα ιδεώδη που παράγονται από τους πρώτους
	\end{enumerate}
\end{prop}

\begin{proof}

	Με επαγωγή στο πλήθος των πρώτων που διαιρούν το $n$. Θεωρούμε τα σώματα:
	\begin{figure}[H]
		\centering
	\begin{tikzcd}
		& K = \Q(\zeta_n)                            &                                   \\
E=\Q(\zeta_{p^r}) \arrow[ru, no head] &                                            & F=\Q(\zeta_m) \arrow[lu, no head] \\
		& \Q \arrow[lu, no head] \arrow[ru, no head] &                                  
\end{tikzcd}
\end{figure}
και κοιτάμε πώς το $p$ παραγοντοποιείται στα $E,F$.
$p\mathcal{O}_E = \p^{\phi(p^r)}$ διακλαδίζεται πλήρως όπως δίνεται από το προηγούμενη πρόταση.
$p\mathcal{O}_F = \p_1\cdots \p_r$ δεν διακλαδίζεται καθώς το $p$ είναι σχετικά πρώτο με την διακρίνουσα.

Τώρα, κοιτάμε την παραγοντοίηση
\end{proof}

\section{Άπειρη Θεωρία \tl{Galois}}
\section{Θεωρία Κλάσεων Σωμάτων}

\end{document}


\section{Άπειρη Θεωρία \tl{Galois}}

Έστω $K/k$ μια \tl{Galois} επέκταση σωμάτων. Όπως συνηθίζεται, θα γράφουμε $\Gal(K/k)$ για το σύνολο των αυτομορφισμών του $K$ που 
διατηρούν το $k$ σταθερό κατά σημείο. Έστω $F$ μια ενδιάμεση πεπερασμένη επέκταση, δηλαδή $k\subseteq F \subseteq K$ με $[F:k]<\infty$. 
Ειδικότερα, η $\Gal(K/F)$ είναι υποομάδα πεπερασμένου δείκτη της $\Gal(K/k)$. Ορίζουμε μια τοπολογία στην $\Gal(K/k)$ όπου τα σύνολα 
$\Gal(K/F)$ σχηματίζουν μια βάση περιοχών του ουδέτερου στοιχείου της $\Gal(K/k)$. Αυτή η τοπολογία αναφέρεται στην βιβλιογραφία ως τοπολογία του \tl{Krull} και κάνει την ομάδα $\Gal(K/k)$ \tl{pro}-πεπερασμένη, δηλαδή σαν τοπολογικό χώρο την κάνει \tl{Hausdorff}, συμπαγή και πλήρως ασυνεκτική. Επιπλέον,

$$\Gal(K/k)\cong \varprojlim\limits_F \Gal(K/k)/\Gal(K/F) \cong \varprojlim\limits_F \Gal(F/k)$$ όπου το $F$ διατρέχει τις πεπερασμένες κανονικές επεκτάσεις $F/k$ ή κάθε υπακολουθία τους, έτσι ώστε $\cup F = K$. Η διάταξη που χρησιμοποιούμε στο αντίστροφο όριο είναι του περιέχεσθαι και ως απεικονίσεις είναι οι φυσιολογικές απεικονίσεις περιορισμού $\Gal(F_2/k) \longrightarrow \Gal(F_1/k)$ για $F_1\subseteq F_2$. Με τα παραπάνω, το θεμελιώδης θεώρημα της θεωρίας \tl{Galois} διατυπώνεται ως εξής.

\begin{theorem}[Θεμελιώδης Θεώρημα Θεωρίας \tl{Galois}]
    Έστω $K/k$ μια \tl{Galois} επέκταση. Υπάρχει μια 1-1 και επί αντιστοιχία μεταξύ των κλειστών υποομάδων $H$ της $\Gal(K/k)$ και των 
    ενδιάμεσων επεκτάσεων $k\subseteq F \subseteq K$ έτσι ώστε 
    $$H\longleftrightarrow K^H$$
    $$\Gal(K/L) \longleftrightarrow L$$ 
    Οι ανοιχτές υποομάδες αντιστοιχούν στις πεπερασμένες υποεπεκτάσεις.
\end{theorem}


\noindent Ένα σημαντικό παράδειγμα είναι η επέκταση $\Q(\zeta_{p^\infty})/\Q$. Την αποκτάμε με το να επισυνάψουμε όλες τις $n$-οστές 
ρίζες της μονάδας, όπου το $n$ διατρέχει τις δυνάμεις του $p$. Βασιζόμενοι σε αυτήν την επέκταση θα κατασκευάσουμε στα επόμενα κεφάλαια 
την λεγόμενη $\Z_p$-επέκταση του $\Q$. Γνωρίζουμε ότι ένα στοιχείο της $\Gal(\Q(\zeta_{p^\infty})/\Q)$ καθορίζεται πλήρως από την 
δράση του στις ρίζες της μονάδας που αναφέραμε. Έστω $n \in \mathbb{N}$. Για ένα $\sigma \in \Gal(\Q(\zeta_{p^\infty})/\Q)$ έχουμε 
$\sigma(\zeta_{p^n}) = \zeta^{\sigma_n}_{p^n}$ για κάποιο $\sigma_n \in \left(\Z/p^n\Z\right)^\times$. Παρατηρούμε ότι 
$\sigma_n \equiv \sigma_{n-1}(\operatorname{mod}p^{n-1})$ για κάθε $n\geq 1$. Ειδικότερα, αυτό μας δείχνει ότι παίρνουμε ένα 
στοιχείο της ομάδας

$$\Z_p^\times \cong \varprojlim\limits_n \left(\Z/p^n\Z\right)^\times \cong \varprojlim\limits_n \Gal(\Q(\zeta_{p^n})/\Q).$$

\noindent Αντίστροφα, είναι εύκολο να δούμε ότι για ένα $a \in \Z_p^\times$ έχουμε ένα στοιχείο $$\sigma \in \Gal(\Q(\zeta_{p^\infty})/\Q)$$
$$\sigma(\zeta_{p^n}) =\zeta^a_{p^n}$$ που καθορίζεται πλήρως από την παραπάνω δράση. Άρα έχουμε ότι
$$\Z^\times_p \cong \Gal(\Q(\zeta_{p^\infty})/\Q).$$ Επιπλέον, η κλειστή υποομάδα $1+p^n\Z_p$ αντιστοιχεί στο σταθερό της σώμα $\Q(\zeta_{p^n})$.


Έστω τώρα $K/k$ μια \tl{Galois} επέκταση, η οποία δεν είναι απαραίτητα πεπερασμένη. Έστω $\mathcal{O}_K$ και $\mathcal{O}_k$ να είναι οι 
δακτύλιοι ακεραίων των $K$ και $k$ αντίστοιχα. Θα θέλαμε να έχουμε κάτι αντίστοιχο με την διακλάδωση των πρώτων ιδεωδών όπως γίνεται στα 
σώματα αριθμών, ωστόσο δεν ισχύει γενικά ότι οι $\mathcal{O}_K$ και $\mathcal{O}_k$ είναι περιοχές \tl{Dedekind}, ώστε να έχουμε το 
μονοσήμαντο της παραγοντοποίησης. Για παράδειγμα, στο σώμα $K = \Q(\zeta_{p^\infty})$ που αναφέραμε πριν, ο δακτύλιος $\mathcal{O}_K$ 
δεν είναι περιοχή του \tl{Dedekind}. Για να το δει κανείς, αρκεί να θεωρήσει την $p$-οστή δύναμη του πρώτου ιδεωδούς 
$\p = (\zeta_p-1,\zeta_{p^2}-1,\ldots)$.

Εφόσον δεν γίνεται να ορίσουμε την διακλάδωση των πρώτων ιδεωδών ως προς την παραγοντοποίησή τους, πρέπει να το κάνουμε με άλλο τρόπο. 
Υπενθυμίζουμε ότι στις πεπερασμένες επεκτάσεις, αν $e$ είναι ο δείκτης διακλάδωσης ενός πρώτου ιδεωδούς $\q$ που στέκεται πάνω από το $\p$, τότε έχουμε ότι $e = |I_\q|$ με $I_\q$ να είναι η ομάδα αδράνειας. Με βάση αυτό θα ορίσουμε την διακλάδωση στις άπειρες επεκτάσεις.

Όπως στην πεπερασμένη περίπτωση, ορίζουμε την ομάδα διάσπασης $D_\q$ ενός πρώτου ιδεωδούς $\q \subset \mathcal{O}_K$ 
που στέκεται πάνω από το $\p \subset \mathcal{O}_k$ ως εξής.
$$D_\q = \{ \sigma \in \Gal (K/k): \ \sigma(\q) = \q \}$$ και την ομάδα αδράνειας ως 
$$I_\q = \{ \sigma \in D_\q: \ \sigma(a) \equiv a(\operatorname{mod}\q) \quad \forall \ a \in \mathcal{O}_K\}$$ Συνεπώς, ορίζουμε τον 
δείκτη διακλάδωσης $e = e(\q \mid \p)$ να είναι η τάξη, όχι απαραίτητα πεπερασμένη, της ομάδας $I_\q$.

Η αντίστοιχη κατάσταση για τις άπειρες θέσεις, δηλαδή τους αρχιμήδειους πρώτους, διαφέρει ελάχιστα. Σε αυτή τη περίπτωση, το $e$ παίρνει 
μόνο τις τιμές $1$ ή $2$. Συγκεκριμένα, το $I_\q$ είναι μη-τετριμμένο στην άπειρη επέκταση μόνο όταν το $\p$ είναι πραγματικό και το $\q$ 
μιγαδικό. Ειδικότερα, η ομάδα αδράνειας παράγεται από την απεικόνιση του μιγαδικού συζυγή.